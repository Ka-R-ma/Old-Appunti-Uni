\documentclass{subfiles}
\begin{document}
Brevemente, sintetizzando per i diversi autori:
\begin{itemize}
    \item \textbf{Freud, Skinner, Dollard e Miller}: motivazione di tipo edonistico, cioè la tendenza a ricercare il piacere e a evitare il dolore.

    \item \textbf{Rogers, Maslow, Jung, Horney}: auto-realizzazione, ossia la possibilità di esprimere appieno le proprie potenzialità.

    \item \textbf{Kelly, May}: prospettiva cognitiva, ciò che spinge le persone è la ricerca del significato di ciò che le circonda e la riduzione dell'incertezza.

    \item \textbf{Bandura, Mischel}: ottica cognitivo-sociale, il comportamento umano è mosso da una motivazione autodiretta, definita \textit{"autoregolazione"},
          cioè la capacità delle persone di stabilire per se stesse degli obiettivi.
\end{itemize}
\end{document}