\documentclass{subfiles}
\begin{document}
È possibile individuare due posizioni, riduzionistiche, che anche se diverse, ritengono che il comportamento
sia determinato da forze estranee.

Il riferimento è al \textbf{riduzionismo biologico}, in cui prevale la posizione secondo
la quale molte condotte sono governate da fattori genetici, somatici, istintuali, relativi al sistema
nervoso (teorie come la psicoanalisi).

Il \textbf{riduzionismo sociologico-ambientale}, nel quale il comportamento sarebbe
il risultato dei condizionamenti sociali, culturali, e di meccanismi legati ai rinforzi positivi, negativi e alle punizioni
(teorie di stampo comportamentista). La visione della persona come \textbf{agente libero} e capace di autodeterminazione è invece
sostenuta dagli indirizzi cognitivo-sociali, dove l'uomo è visto come soggetto attivo nella sua interazione con l'ambiente.
\end{document}