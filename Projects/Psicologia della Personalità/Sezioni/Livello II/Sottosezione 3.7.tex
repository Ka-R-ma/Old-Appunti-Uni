\documentclass{subfiles}
\begin{document}

I problemi hanno molte possibili origini, una tra quale sono le esperienze infantili. 

Il disagio psichico può nascere dalle fissazioni rimaste nel passaggio ad un altro stadio dello 
sviluppo psicosessuale. In una fissazione molto forte, la preoccupazione lascia alla persona 
poca energia per qualsiasi attività dell' Io.

Un'altra fonte di disagio psichico è una forte repressione o rimozione delle pulsioni di base. 
Se un Super-Io eccessivamente punitivo o un ambiente troppo rigido fanno si che troppe pulsioni 
siano da nascondere, la personalità ne risulterà distorta e disfunzionale. 
Tutto ciò porta a un costante dispendio di energia che altrimenti sarebbe disponibile per l'Io.

Una terza fonte di problemi sono i traumi sepolti, soprattutto nel periodo della prima infanzia.\\

\subsubsection{Tecnica Psicoanalitica}

La pratica psicoanalitica si focalizza sul trasformare il 
\textbf{contenuto dell'Es in contenuto dell'Io}, permettendo al paziente di prendere 
consapevolezza dei conflitti inconsci e di integrarli nella sua coscienza.

Inizialmente, Freud sperimenta con l'ipnosi ma poi adotta la tecnica delle 
\textbf{`associazioni libere'}, in cui il paziente esprime liberamente tutto ciò che gli 
passa per la mente, facilitando l'emergere dei contenuti inconsci, spesso simbolici, come nei sogni.

Avvolte le persone in terapia lottano attivamente contro la presa di coscienza dei conflitti e 
degli impulsi. Questa lotta viene detta \textbf{resistenza} e può essere conscia o inconscia ma 
è fondamentale per superare il conflitto.

Un elemento importante della terapia psicoanalitica è il \textbf{transfert}, processo per cui 
i sentimenti verso le persone significative del paziente (amore o odio) sono spostate sul terapeuta. 
Questo permette di interpretare i `personaggi`del conflitto attraverso l'\textbf{insight}, che 
non è solo una comprensione intellettuale ma la ri-esperienza emotiva dei conflitti 
repressi, delle memorie e delle pulsioni inconsce.

Affinché una rielaborazione cognitiva sia utile, deve avvenire nel contesto di una 
\textbf{catarsi emotiva}, una liberazione di energia repressa. Sebbene efficace, l'obiettivo 
principale non è ridurre l'angoscia ma produrre l'insight, il che implica un processo lungo e 
continuo di lavoro psicoanalitico.

\end{document}