\documentclass{subfiles}
\begin{document}
La personalità è influenzata da una serie di elementi che determinano il modo di essere di un individuo.
Questi risultano essere:
\begin{itemize}
    \item \emph{fattori genetici:} si tratta di caratteristiche genetiche dipendenti dall'eredità genetica dell'individuo;
          tra queste figurano la spinta al successo, al tradizionalismo, ecc;

    \item \emph{fattori disposizionali:} spesso definiti \emph{tratti}, definiscono elementi costitutivi della personalità, costanti nel tempo,
          assicurando una coerenza \textbf{evolutivo-longitudinale} e \textbf{cross-situazionale};

    \item \emph{fattori socio-culturali:} si fa riferimento a caratteristiche quali la cultura di appartenenza, l'ordine di nascita, ecc;

    \item \emph{fattori d'apprendimento:} per alcuni teorici, la personalità è il frutto di un sistema di ricompense/punizioni;
          cioé si suppone sia possibile controllare lo sviluppo dell'individuo, manipolando opportunamente tale sistema;

    \item \emph{fattori esistenziali:} legati a domande di carattere esistenziale: ``qual é il senso della vita?', ecc;

    \item \emph{meccanismi inconsci:} ossia dei processi di cui l'individuo non è consapevole è che esistono indipendentemente dalla sua volontà.
          Le teorie psicoanalitiche studiano le modalità non consapevoli con cui le persone si esprimono, come sogni, lapsus o associazioni libere,
          esplorando ciò che si nasconde dietro la \textit{"maschera"} che ogni individuo esibisce nel corso della sua vita.

    \item \emph{processi cognitivi:} modalità con le quali il soggetto si interfaccia con l'ambiente; ossia ai modi in cui  l'individuo percepisce, trattiene,
          trasforma e traduce in azione le informazioni dell'ambiente.
\end{itemize}
\end{document}