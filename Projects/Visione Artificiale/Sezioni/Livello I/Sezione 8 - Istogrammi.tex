\documentclass{subfiles}
\begin{document}
Data un'immagine \(I\), un suo \emph{istogramma} è un vettore contenente informazioni circa una caratteristica dell'immagine,
banalmente i livelli di luminanza della stessa. Da questo segue che la definizione popolare di istogramma, oltre che erronea,
indica semplicemente la rappresentazione grafica dei vettori prima citati.

Parlando delle applicazioni pratiche degli istogrammi, questi permettono di comprendere se un immagine è stata modificata in qualche modo.
A scopo didattico si considerano le operazioni di \emph{stretching \emph{e} dilatazione}.

\subsection{Stretching}
\subfile{Sotto Sezioni/Sottosezione 8.1 - Stretching.tex}

\subsection{Equalizzazione}
\subfile{Sotto Sezioni/Sottosezione 8.2 - Equalizzazione.tex}

\end{document}