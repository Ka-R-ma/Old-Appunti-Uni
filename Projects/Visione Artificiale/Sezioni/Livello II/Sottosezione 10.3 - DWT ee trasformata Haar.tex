\documentclass{subfiles}
\begin{document}
Partendo da definire il concetto di \emph{wavelet}; queste sono funzioni il cui ``moto'' è di tipo oscillatorio.
Dal concetto di funzione wavelet nascono le trasformate wavelet.
\\ \\
Da un punto di vista matematico, una trasformata wavelet permette di rappresentare una funzione integrabile per quadrati mediante una determinata serie ortonormale generata da una wavelet,
soddisfacendo proprietà che non risultano di interesse. Dal punto di vista pratico, ciò comporta la possibilità di decomporre il segnale tramite l'uso di filtri passa-alto e passa-basso,
definiti in funzione della wavelet, e successivamente ripristinarlo. Tale ricomposizione è effettuata moltiplicando per opportuni valori, i cosiddetti \emph{coefficienti wavelet},
le cosiddette funzioni madre, che definiscono la portante del segnale, e funzioni di scala.

Per quanto riguarda le DWT, queste sono un sottoinsieme delle trasformate wavelet in cui la funzione wavelet è campionata in modo discreto.

\begin{Note*}
    da un punto di vista teorico, le DWT sono in numero finito; tale numero è però così elevato da risultare infinito dal punto di vista pratico.
\end{Note*}

\subsubsection{Trasformata Haar}
Proposta per la prima volta nel 1909 da \emph{Alfred Haar}, da cui prende il nome, è la prima delle trasformate wavelet, la quale oltre a prestarsi bene per i segnali ad onda quadra,
risulta facilmente implementabile.

Essendo una trasformata wavelet, Haar possiede una propria funzione madre e una propria funzione di scala, a seguito riportate.
\[\underbrace{\varphi = \begin{cases}
            1, & \quad 0 \le t < 1       \\
            0, & \quad \text{altrimenti} \\
        \end{cases}}_{\text{funzione madre}} \qquad \underbrace{\psi = \begin{cases}
            1,  & \quad 0 \le t < \frac{1}{2} \\
            -1, & \quad \frac{1}{2} \le t < 1 \\
            0,  & \quad \text{altrimenti}
        \end{cases}}_{\text{funzione di scala}}\]

Come detto Haar è facilmente implementabile, motivo di ciò è il fatto che sfrutta unicamente semi-somme e semi-differenze\footnotemark[9], come si può vedere da \emph{Figura \ref{fig:10.1}}.
Per comprendere il funzionamento nel caso monodimensionale, l'estensione al caso bidimensionale è immediata poiché si ha separabilità delle variabili, si consideri quanto segue.

Si consideri \(v\) un certo vettore di numeri, la cui lunghezza è una qualche potenza di due, la trasformata si riduce a calcolare passo dopo passo semi-somme e semi-differenze\footnotemark[15],
fintanto che si ottengono vettori di lunghezza unitaria.

\subfile{../Figure/Firgure. 10.1 - Implementaione Haar monodimensionale.tex}

\footnotetext[9]{Data una coppia di numeri dicasi semi-somma, la media della somma dei due; analogamente dicasi semi-differenza, la media della differenza tra i due.}

\subsubsection{Filter bank}
Il termine "filter-bank" si riferisce a un insieme di filtri applicati a un segnale, o a un insieme di segnali,
con lo scopo principale di dividere tale segnale in diverse bande di frequenza o di estrarre particolari caratteristiche a diverse scale o orientamenti.
\\ \\
Nel contesto dei segnali audio o immagini, un filter-bank opera spesso nel dominio della frequenza.
Ciò significa che il segnale di ingresso viene scomposto nelle sue componenti di frequenza utilizzando vari filtri, in genere una successione di filtri passa-alto e passa-basso.
\\ \\
Una delle applicazioni più diffuse dei filter-bank sono le trasformazioni wavelet.
Una trasformata wavelet, come detto, utilizza una serie di filtri passa-alto e passa-basso per decomporre un segnale in diverse bande di frequenza.
Da ciò segue la possibilità di un'analisi multirisoluzione, è particolarmente utile per compiti come la compressione di immagini, il denoising e l'estrazione di caratteristiche.
\\ \\
In sintesi, un filter-bank è un insieme di filtri progettati per estrarre componenti di frequenza o caratteristiche specifiche da un segnale.
Nell'audio, nell'elaborazione delle immagini o nei sistemi di comunicazione, i filter-bank svolgono un ruolo cruciale nell'analisi,
nell'elaborazione e nell'estrazione di informazioni significative da segnali a frequenze o scale diverse.

\subsubsection{Decomposizione standard e non standard}
Quando si parla di decomposizione standard nel contesto dell'elaborazione dei segnali, in particolare nell'ambito delle trasformate wavelet,
di solito ci si riferisce a un'analisi gerarchica multirisoluzione di un segnale o di un'immagine.
\\ \\
Con la trasformata wavelet, un segnale o un'immagine viene decomposto in diverse sotto-bande o scale di frequenza. Questa decomposizione è gerarchica,
il che significa che è possibile continuare a scomporre ulteriormente le sottobande per ottenere informazioni di frequenza più dettagliate.
\\ \\
La decomposizione standard impiega in genere due filtri:
\begin{itemize}
    \item Filtro passa-basso (LPF): Noto anche come filtro di scala, questo filtro estrae le componenti a bassa frequenza del segnale o dell'immagine.
    \item Filtro passa-alto (HPF): Noto anche come filtro wavelet, questo filtro cattura le componenti ad alta frequenza, evidenziando i dettagli e i bordi del segnale o dell'immagine.
\end{itemize}

Dopo l'applicazione dei filtri, i coefficienti risultanti vengono spesso sotto-campionati (o decimati) di un fattore 2.
Questo sotto-campionamento riduce le dimensioni dei dati, preservando le informazioni di frequenza essenziali. I coefficienti declassati servono quindi come input per il livello di decomposizione successivo.
\\ \\
Una volta ottenuti i coefficienti di decomposizione ai vari livelli, è possibile ricostruire il segnale o l'immagine utilizzando la trasformata wavelet inversa.
Questa ricostruzione consente di ricostruire il segnale originale con vari livelli di dettaglio, da approssimazioni grossolane a dettagli fini.

\begin{Note*}
    Nel caso di immagini, se l'applicazione alternata dei filtri è effettuata lungo una direzione prima, e lungo l'altra poi, si parla di decomposizione standard;
    se si procede applicando i filtri alternativamente nelle due direzioni, si parla di decomposizione non standard.
\end{Note*}
\end{document}