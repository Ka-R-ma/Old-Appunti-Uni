\documentclass{subfiles}
\begin{document}

Lo spazio RGB non è l'unico spazio-colore esistente; un altro è infatti lo spazio HSV.
Questi, in \emph{Figura \ref{fig:3.1}} rappresenta il colore, \emph{hue}, tramite angoli: per convenzione gli zero gradi sono il rosso, i 120 il verde e i 240 il blu.
Il livello di saturazione è dipendente dal \emph{chroma}, mentre l'intensità dal \emph{value}.

Ulteriore spazio-colore è HSL. Questi è molto simile ad HSV, infatti può essere visto come un HSV in cui tutti i colori tendenti al bianco,
sono raggruppati in un unico punto.
\subfile{../Figure/Figure 3.1 - HSV color space.tex}
\end{document}