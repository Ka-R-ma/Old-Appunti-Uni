\documentclass{subfiles}
\begin{document}
Partendo col definire l'equalizzazione: sia \(s\) la soglia di equalizzazione: ossia il numero di pixel che si dovrebbe presentare nell'immagine per ciascun livello di grigio.
Siano \(g_{0}, g_{1}, \ldots, g_{255}\) tutti i possibili livelli di grigio e siano \(q_{0}, q_{1}, \ldots, q_{255}\) il numero di pixel per ciascuno dei livelli.

Da un punto di vista teorico, ciò che si dovrebbe verificare è che \(q_{0} = q_{1} = \cdots = q_{255}\): ossia si ha uno stesso numero di pixel per ciascun grigio.
Se si considera però il punto di vista pratico, tale risultato difficilmente è verificato.
Nella realtà dei fatti quel che si fa è accorpare sotto uno stesso livello di grigio, tanti livelli successivi in modo tale che sommando il numero di pixel con tali livelli,
la somma risulti prossima al valore soglia. Cioè
\[
    \text{per opportuni indici} i,j \in [0, 255], i < j, \sum\limits_{k = i}^{j}{q_{k}} \approx s
\]
con \(q_{k}\) numero di pixel con livello di grigio \(g_{k}\).

Passando al come stabilire il valore soglia, ciò risulta banale: si ha infatti che questi è dato dal rapporto tra l'area dell'immagine e la variazione massima tra i livelli di grigio.
Cioè,
\begin{wrapfigure}{l}{0.425\textwidth}
    \centering
    \includegraphics[scale = 0.325]{../Images/Airplane/Equalised Airplane.png}
    \caption{Equalizzazione di \emph{Figura \ref{fig:4.8}}.}
    \label{fig:8.3}
\end{wrapfigure}
supposte \(h \text{e} w\) rispettivamente altezza e larghezza dell'immagine, il valore di soglia \(s\) è definito come
\[
    s = \floor{\frac{h * w}{255}}
\]

Per dare un esempio di equalizzazione, si supponga di applicare quanto descritto a \emph{Figura \ref{fig:4.8}}, quel che ne deriva è mostrato in \emph{Figura \ref{fig:8.3}}.
\begin{Note*}
    dall'istogramma non è possibile stabilire l'effettiva equalizzazione di un immagine, se ne può solamente sospettare la presenza. Esistono immagini,
    che presentano di base un istogramma piatto.
\end{Note*}

\clearpage

\subsubsection{Threshold}
Parlando più in generale del valore di sogliatura \(s\), questi in generale, dipende, oltre che dal valore di grigio del singolo pixel, anche dalla posizione di quest'ultimo.
Segue così la possibilità di effettuare una sogliatura, o threshold, globale e locale.
Per comprendere la differenza tra i due, si assuma che \(s\) sia determinato da una funzione \(f(x)\).
Con tale notazione si ha che
\begin{enumerate}
    \item nel caso globale \(s = f(g(\vb{p}))\), ove \(g(\vb{p})\) rappresenta il livello di grigio del pixel \(\vb{p}\);
    \item nel caso locale \(s = f(g(\vb{p}), x, y)\), ove \(g(\vb{p})\) rappresenta il livello di grigio del pixel \(\vb{p}\) e \(x, y\) le sue coordinate
\end{enumerate}

Si considerano ora alcune tipologie di threshold.
\begin{itemize}
    \item \textbf{Threshold geometrico:} si assume che l'immagine abbia un istogramma bimodale:
          ossia è possibile separare le componenti dell'immagine in \emph{foreground \emph{e} background}.
          L'effettivo calcolo del valore di sogliatura è ottenuto congiungendo i due picchi del segnale e, tracciando dei segmenti ortogonali al segmento così formatosi,
          si considera il piede di quello la cui lunghezza risulta essere maggiore.

    \item \textbf{Threshold adattivo:} potrebbe capitare, come accade nella maggior parte dei casi, che l'istogramma non sia bimodale,
          oppure lo è ma è mascherato dal rumore. In questi casi si può determinare un livello di threshold per un intorno di pixel.

    \item \textbf{Threshold iterativo:} si stabilisce arbitrariamente un valore di soglia, per ciascun sottoinsieme si calcola la rispettiva media,
          siano queste \(\mu_{1}, \mu_{2}\), di queste se ne calcola la media \(\mu\) e si riapplica il medesimo processo usando \(\mu\) come valore di soglia.
          Il processo si interrompe quando si è calcolato un \(\mu\) precedentemente calcolato, ponendo questi come effettivo livello di sogliatura.
\end{itemize}


\end{document}