\documentclass{subfiles}
\begin{document}
Legata alla \emph{trasformata di Fourier}, la \emph{discrete-cosine-transform} è una tecnica utilizzata nell'analisi dei segnali e nella compressione di dati.
Di tale trasformata si distingue il caso monodimensionale e quello bidimensionale. Si dimostra inoltre che il caso bidimensionale gode di separabilità delle variabili,
indi per cui, una DCT bidimensionale può essere ottenuta come prodotto tensoriale di due DCT monodimensionali.

Parlando del funzionamento della trasformata: questa riceve in input una funzione \(f(x)\), che si può supporre rappresentare una sequenza finita di valori,
e di questa procede a restituire una sua rappresentazione nello spazio delle frequenze, i cui termini \(C(u)\),
sono ottenuti come combinazione lineare di \(f(x)\) e una opportuna base di coseni. Cioè
\begin{equation}
    \begin{aligned}
        C(u) & = a(u) \sum\limits_{x = 0}^{N - 1}{f(x) \cos\left(\frac{(2x + 1) u\pi}{2N}\right)} \\
        f(x) & = \sum\limits_{u = 0}^{N - 1}{a(u) C(u) \cos\left(\frac{(2x + 1) u\pi}{2N}\right)} \\
        a(u) & = \begin{cases}
                     \sqrt{\frac{1}{N}}, u = 0                \\
                     \sqrt{\frac{2}{N}}, u = 1, \ldots, N - 1 \\
                 \end{cases}
    \end{aligned}
\end{equation}
ove, considerata \(f(x)\) una sequenza di elementi, \(N\) rappresenta il numero degli stessi.
\\ \\
Come anticipato in \emph{Sezione \ref{sec:6.2}}, la DCT è utilizza nel processo di compressione del formato JPG.
Dando una maggiore descrizione del suo funzionamento in tale formato, si consideri \(B\) uno dei blocchi 8 x 8 ottenuti dalla prima fase della compressione con JPG.
In questo caso, poiché si opera di fondo con gli elementi di una matrice, la formulazione in \emph{Equazione \eqref{eq:3}} va estesa al caso bidimensionale,
da cui
\[\begin{aligned}
        C(u, v) & = a(u)a(v) \sum\limits_{x = 0}^{N - 1}\sum\limits_{y = 0}^{N - 1}{f(x, y) \cos\left(\frac{(2x + 1) u\pi}{2N}\right) \cos\left(\frac{(2y + 1) v\pi}{2N}\right)} \\
        f(x, y) & = \sum\limits_{u = 0}^{N - 1}\sum\limits_{v = 0}^{N - 1}{a(u)a(v) C(u, v) \cos\left(\frac{(2x + 1) u\pi}{2N}\right) \cos\left(\frac{(2y + 1) v\pi}{2N}\right)} \\
        a(u)    & = \begin{cases}
                        \sqrt{\frac{1}{N}}, u = 0                \\
                        \sqrt{\frac{2}{N}}, u = 1, \ldots, N - 1 \\
                    \end{cases} \qquad
        a(v) = \begin{cases}
                   \sqrt{\frac{1}{N}}, v = 0                \\
                   \sqrt{\frac{2}{N}}, v = 1, \ldots, N - 1 \\
               \end{cases}
    \end{aligned}\]
Sfruttando la formulazione di cui sopra, posto che \(f(x, y)\) rappresenti \(B\), si ottiene una rappresentazione di \(B\) nello spazio delle frequenze,
rappresentazione che verrà sfruttata nelle successive fasi di compressione del formato.
\end{document}