\documentclass{subfiles}
\begin{document}
L'acquisizione di immagini digitali, è effettuata tramite l'uso di CCD.
Questi, in breve, sono dispositivi che contano il numero di fotoni in una certa areola.
Per quanto appena descritto però i CCD sono capaci di acquisire immagini a scale di grigio, come si può allora acquisire quelle a colore?
L'idea più semplice sarebbe quella di usare un CCD per canale, dunque uno per il rosso, uno per il verde e uno per il blu.
Un'implementazione di questo tipo risulta però complessa e costosa, ragione per cui nella maggior parte dei casi si sfruttano i filtri di Bayer.
Si pensi a questi come delle maschere, di forma e dimensione variabile, che in un area 2x2 della stessa presenta due elementi verdi, uno rosso e uno blu\footnotemark[1].

\footnotetext[1]{Per comprendere  il motivo di tale disparità si veda l'osservazione fatta in \emph{Sezione \ref{sec:1}}}
\end{document}