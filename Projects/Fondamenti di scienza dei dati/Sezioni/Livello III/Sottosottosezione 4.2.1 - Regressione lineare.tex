\documentclass{subfiles}
\begin{document}
Come suggerito dal nome, è una tipologia di regressione utile quando, considerando il grafo di dispersione, i dati sono grossomodo distribuiti lungo una linea retta.
Si distinguono:
\begin{itemize}
    \item \emph{il caso semplice:} \(y\) dipende da un'unica variabile \(x\);
    \item \emph{il caso multiplo:} \(y\) è dipendente da due o più variabili \(\List{x}{1}{k}\).
\end{itemize}

Considerando unicamente il caso semplice (il caso multiplo si riduce ad una banale estensione), la relazione tra le variabili si può esprimere come
\[
    y = a + bx + \varepsilon
\]
da ciò, calcolare gli \(y_{i}\) si riduce a calcolare
\[
    y_{i} = a_{0} + b_{1}x_{i} + \varepsilon_{i}
\]
Ciò, comporta dover calcolare \(a_{0}, b_{1}\), in generale realizzato con il metodo dei minimi quadrati.

\paragraph*{Metriche di regressione lineare}
Se ne distinguono essenzialmente tre:
\begin{itemize}
    \item \textbf{maximum error:} \(\Max*{\Abs{f(x_{k}) - y_{k}}}[k \in \Set{1, \ldots, n}]\);
    \item \textbf{mean absolute error:} \(\Frac{1}{n} \Sum{\Abs{f(x_{k}) - y_{k}}}{k = 1}[n]\);
    \item \textbf{mean absolute error:} \(\sqrt{\Frac{1}{n} \Sum{\Abs{f(x_{k}) - y_{k}}^{2}}{k = 1}[n]}\);
\end{itemize}
\end{document}