\documentclass{subfiles}
\begin{document}
Si tratta di una tipologia di machine learning in cui i dati di addestramento sono ``etichettati'':
cioè presentano un campo che descrive la classe di appartenenza degli stessi.
Tra questi si distinguono
\begin{MarginNote}
    Entrambe le categorie saranno discusse nelle sezioni a seguire.
\end{MarginNote}
\begin{itemize}
    \item gli algoritmi di regressione;
    \item gli algoritmi di classificazione.
\end{itemize}
In generale, si utilizzano quando l'obbiettivo è quello di separare in classi i dati.
Si osserva però che affinche l'addestramento possa definirsi ``buono'', l'algoritmo deve minimizzare l'errore relativo i dati di addestramento:
il cosiddetto \emph{training error}, e l'errore relativo i dati di test: il cosiddetto \emph{test/generalization error}.
In fine, per quanto detto e quanto discusso in \emph{Sezione \ref{sec:2}}, si deve fare attenzione
\begin{itemize}
    \item all'\emph{over-fitting:} ossia un fenomeno per cui il modello si è troppo adattato ai dati, non riuscendo dunque a generalizzare;
    \item all'\emph{under-fitting:} fenomeno opposto all'over-fitting, è una condizione in cui il modello ha appreso poco dai dati,
          pertanto non ha le capacità sufficienti a generalizzare.
\end{itemize}
\end{document}