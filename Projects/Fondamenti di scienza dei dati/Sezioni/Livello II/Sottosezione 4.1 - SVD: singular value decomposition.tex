\documentclass{subfiles}
\begin{document}
Tecnica fondamentale per la riduzione della dimensionalità, permette di decomporre la matrice rappresentanti i dati in tre sotto-matrici.
Nel dettaglio, questa è definita come segue.
\begin{Definition*}
    sia $X \in \Real^{n,m}$, allora
    $$
        \exists U, \Sigma, V \Such X = U\Sigma V^{T}, \forall X \in \Real^{n, m}
    $$
    ove $U \in \Real^{n,n} , \Sigma \in \Real^{n,m} \text{e} V \in \Real^{m,m}$.
\end{Definition*}
Ciò equivale a dire, da un punto di vista numerico, che
$$
    X = \Sum{\sigma_{i}u_{i}v_{i}^{T}}{i = 1}[n]
$$
ove $u_{i} \in U, v_{i} \in V \text{e} \sigma_{i} \in \Sigma, \sigma_{i} \ge \sigma_{j}, \forall i < j < n$.\\
\begin{MarginNote}
    Si definiscono i $\sigma_{i}$ valori singolari.
\end{MarginNote}
Geometricamente segue che
\begin{itemize}
    \item $\Sigma$ è una matrice diagonale, nello specifico è data dagli autovalori di $XX^{T} \text{e} X^{T}X$;
    \item rispettivamente, $U \text{e} V$, sono le matrici date dagli autovettori di $XX^{T} \text{e} X^{T}X$.
\end{itemize}

Dalla definizione si è detto che la SVD esiste comunque scelta la matrice $X$. La dimostrazione di ciò è sostenuta dal seguente teorema.
\begin{Theorem*}
    Sia $C = X^{T}X \in \Real^{m,m}$, allora $C$ è diagonale, simmetrica e definita positiva.
    \begin{Proof*}
        da dimostrazione segue banale dal \emph{Teorema di decomposizione spettrale}.
        Per esso si ha che $C = V T V^{T}$, con $T = \Diagonal{\List{\lambda}{1}{m}} \text{e} r = \RankOf{X}$.
        Posto allora $\sigma_{i} = \sqrt{\lambda_{i}}$, segue che
        $$\Sigma = \begin{pmatrix}
                \Diagonal{\List{\sigma}{1}{r}} & 0_{r, (m - r)}   \\
                0_{r, (m - r)}                 & 0_{m - r, m - r} \\
            \end{pmatrix}$$
        e ponendo inoltre
        $$
            u_{i} = \Frac{X}{\sigma_{i}}v_{i}
        $$
        che si dimostrano ortogonali, segue, completando a base
        $$U = \begin{bmatrix}
                u_{1} & \cdots & u_{r} & u_{r + 1} & \cdots & u_{n}
            \end{bmatrix}$$
    \end{Proof*}
\end{Theorem*}
\clearpage

\subsubsection{Low-Rank approximation}
\subfile{../Livello III/Sottosottosezione 4.1.1 - LowRank approximation.tex}

\subsubsection{PCA: principal component analysis}
\subfile{../Livello III/Sottosottosezione 4.1.2 - PCA: principal component analysis.tex}

\end{document}