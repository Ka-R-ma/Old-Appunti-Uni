\documentclass{subfiles}
\begin{document}
Ora sebbene gli algoritmi di clustering siano utilizzati per il ML non-supervisionato, è necessario stabilire la qualità dell'algoritmo,
e per far ciò si definiscono delle misure relativamente ai cluster. Tali misure si dividono in
\begin{itemize}
    \item \emph{interne:} si tratta di misure dipendenti unicamente dai dati del cluster;
    \item \emph{esterne:} misure da utilizzare, nel caso siano comunque presenti delle etichette.
\end{itemize}
\begin{MarginNote}
    Banalmente minore è SSE, migliore è la qualità del clustering.
\end{MarginNote}
Tra le pù diffuse ed utilizzate la \emph{sum of square errors (SSE)}, definita come
$$
    SSE = \Sum{\Sum{dist(x, y_{j})}{x \in C_{j}}}{j = 1}[m]
$$

\subsubsection{Misure interne: coesione e separazione}
\subfile{../Livello III/Sottosottosezione 5.2.1 - Misure interne: coerenza e coesione.tex}

\subsubsection{Misure interne: WSS}
\subfile{../Livello III/Sottosottosezione 5.2.2 - Misure interne: WSS.tex}

\subsubsection{Stima del numero di cluster}
\subfile{../Livello III/Sottosottosezione 5.2.3 - Stima del numero di cluster.tex}
\end{document}