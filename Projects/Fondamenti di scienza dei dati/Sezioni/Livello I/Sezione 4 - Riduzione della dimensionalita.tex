\documentclass{subfiles}
\begin{document}
Come anticipato in \emph{Sezione \ref{sec:3}}, la riduzione della dimensionalità è una tecnica utilizzata dalla'apprendimento non-supervisionato.
L'idea alla base è quella di ridurre la quantità dei dati che si deve analizzare, riducendoli al minimo.
In generale, alla base di una qualsiasi tecnica di riduzione vi è la SVD, nel seguito descritta.

\subsection{SVD: singular value decomposition}
\subfile{../Livello II/Sottosezione 4.1 - SVD: singular value decomposition.tex}
\clearpage

\subsection{Regressione}
\subfile{../Livello II/Sottosezione 4.2 - Regressione.tex}
\clearpage
\end{document}