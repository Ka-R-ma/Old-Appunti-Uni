\documentclass{subfiles}
\begin{document}
Il concetto di clustering fa riferimento ad una classe di algoritmi di machine learning non-supervisionato,
generalmente utilizzati per il data-mining o per la ricerca degli outliers\footnotemark[1].

In generale, a decidere i cluster è l'utente, scelta che deve però minimizzare la seguente funzione
$$
    O = \Sum{\Min*{dist\left(X_{i}, Y_{j}\right)}[j]}{i = 1}[n]
$$
ove $Y_{j}$ è il rappresentante del cluster $C_{j}$.
Più nel dettaglio la definizione di clustering è la seguente.
\begin{Definition*}
    sia $X \in \Real^{n}$, allora un \emph{m-clustering} di $X$ è un partizione $\List{C}{1}{m}$ tale che
    \begin{itemize}
        \item $C_{i} \neq \varnothing, \forall i \in \Set{1, \dots, m}$;
        \item $\Bigcup{C_{i}}{i = 1}[m] = X$;
        \item $C_{i} \cap C_{j} = \varnothing, \text{per} i \neq j$.
    \end{itemize}
\end{Definition*}
Come sarà discusso a seguire se $dist\left(X_{i}, Y_{j}\right) = \Norm{X_{i} - Y_{j}}[2]^{2}$, si parlera di \emph{k-mean} clustering.
Circa il processo di clustering, questi si compone di cinque fasi, quali
\begin{itemize}
    \item selezione della soglia di clustering;
    \item selezione della metodologia di clustering;
    \item scelta del modello di clustering;
    \item validazione dei risultati;
    \item interpretazione dei risultati.
\end{itemize}

\subsection{Tipologie di clustering}
\subfile{../Livello II/Sottosezione 5.1 - Tipologie di clustering.tex}

\subsection{Misura della qualità di clustering}
\subfile{../Livello II/Sottosezione 5.2 - Misure di qualita di clustering.tex}

\subsection{K-mean clustering}
\subfile{../Livello II/Sottosezione 5.3 - K-mean clustering.tex}

\footnotetext[1]{Si tratta di osservazioni del dataset, che hanno poca correlazione con tutti le altre.}
\footnotetext[2]{Si raggiunge la convergenza se vi è una ripetizione di medie, o se si è definito un qualche criterio di arresto.}
\clearpage
\end{document}