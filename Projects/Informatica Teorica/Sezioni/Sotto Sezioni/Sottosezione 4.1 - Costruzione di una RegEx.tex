\documentclass{subfiles}
\begin{document}
Sia \(e\) una RegEx. La costruzione di \(e\) è di tipo ricorsivo.
\begin{Base*}
    \begin{itemize}
        \item []
        \item \(\varepsilon \text{e} \emptyset\) sono espressioni regolari, ove \(L(\varepsilon) = \Set{\varepsilon}, L(\emptyset) = \Set{}\).
        \item Se \(\alpha\) è un simbolo, allora questi è una RegEx, ove \(L(\alpha) = \Set{\alpha}\).
    \end{itemize}
\end{Base*}
\begin{Induction*}
    \begin{itemize}
        \item []
        \item Siano \(e \text{ed} f\) due RegEx. Allora \(e + f\) è un'espressione regolare.
        \item Siano \(e \text{ed} f\) due RegEx. Allora \(ef\) è un'espressione regolare.
        \item Sia \(e\) RegEx. Allora \(e^{*}\) è un'espressione regolare.
        \item Sia \(e\) RegEx. Allora \((e)\) è un'espressione regolare.
    \end{itemize}
\end{Induction*}
\end{document}