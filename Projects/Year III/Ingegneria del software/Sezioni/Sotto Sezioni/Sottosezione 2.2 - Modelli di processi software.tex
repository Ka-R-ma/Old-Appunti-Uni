\documentclass{subfiles}
\begin{document}
Un modello di processo software è una rappresentazione semplificata di un processo software, da una particolare prospettiva;
si forniscono dunque solo informazioni parziali sul processo.
\\ \\
Nel seguito saranno trattati i seguenti modelli
\begin{itemize}
    \item a cascata;
    \item sviluppo incrementale;
    \item integrazione e configurazione.
\end{itemize}

\subsubsection{Modello a cascata}
\'E uno dei modelli di processo guidato da piani.
Attività principali sono, nell'ordine: la \emph{definizione dei requisiti}, la \emph{progettazione del sistema e del software}, l'\emph{implementazione e i test d'unità},
l'\emph{integrazione e i test di sistema} e la verifica dell' \emph{operatività e la manutenzione}.
\end{document}