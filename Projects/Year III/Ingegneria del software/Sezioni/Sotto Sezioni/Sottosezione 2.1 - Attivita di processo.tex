\documentclass{subfiles}
\begin{document}
I processi software sono sequenze intricate di attività tecniche, collaborative e manageriali, cui obbiettivo specificare, progettare,
implementare e testare un sistema software.
La metodologia con cui tali attività sono svolte è dipendente da vari fattori, come il tipo di software da realizzare e l'esperienza dei programmatori.

\subsubsection{Specifica del software}
La creazione delle specifiche del software è uno stadio critico del processo, atto a capire e definire i servizi richiesti dal sistema.
\\ \\
Tale attività, porta alla produzione di un documento delle caratteristiche concordate, che specifica un sistema che soddisfa le richieste degli \emph{stakeholder\footnotemark[1]}
\\ \\
Fasi principali sono
\begin{itemize}
    \item la deduzione e l'analisi dei requisiti;
    \item la specifica dei requisiti;
    \item la convalida dei requisiti.
\end{itemize}


\footnotetext[1]{Tutte le figure coinvolte nella realizzazione del progetto. (e.g. investitori, ecc.)}

\subsubsection{Progettazione e implementazione}
Lo stadio di implementazione del software è il processo di conversione delle specifiche, in un sistema eseguibile e consegnabile al cliente.
\\ \\
La progettazione del software, può essere vista come una descrizione della struttura del software che si deve implementare, dei modelli e delle strutture dati usate.
\\ \\
Attività che possono far parte del processo, sono: la \emph{progettazione dell'architettura}, la \emph{progettazione del DB}, quella delle interfacce.

\subsubsection{Convalida del software}
Più genericamente, verifica e convalida (V\&V), è intenta a mostrare che un sistema sia conforme e soddisfi le specifiche del cliente.
\\ \\
Tecnica principale pwe la convalida risultano essere i test dei programmi.

\begin{Remark*}
    a meno che per piccoli programmi, i sistemi non dovrebbero essere testati come entità monolitiche.
\end{Remark*}

\noindent Fasi della fase di test sono
\begin{itemize}
    \item i \emph{test di unità}: i diversi componenti sono testati individualmente;
    \item i \emph{test di sistema}: i componenti sono integrati per formare il sistema completo;
    \item i \emph{test del cliente}: il cliente testa il software con i propri dati.
\end{itemize}

\subsubsection{Evoluzione del software}
Risulta banale che un buon software sia modificabile qual'ora necessario, questo perché, se comparato col modificare un componente hardware, risulta più economico.
\\ \\
Conseguenza di ciò è la necessità di metodi per far fronte a eventuali cambiamenti.
Tra queste la prototipazione: al posto di creare un sistema completo, si procede col creare dei programmi che inizialmente soddisfano le attività strettamente necessarie al funzionamento del sistema,
procedendo, di versione in versione, ad aggiungere e migliorare nuove funzionalità.
\end{document}