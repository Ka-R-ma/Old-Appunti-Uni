\documentclass{subfiles}
\begin{document}
Come detto un processo software è un'insieme di attività atte alla realizzazione di un prodotto software.
Nonostante ne esistano diversi, tutti includono le attività descritte nel seguito.

\begin{Remark*}
    quando si descrivono i processi, è importante descrivere anche le persone coinvolte, come e cosa viene prodotto.
\end{Remark*}

\begin{Remark*}
    lo sviluppo può essere realizzato secondo un piano: ogni attività è pianificata a priori;
    secondo un modello agile: la pianificazione è incrementale e procede per tutto lo sviluppo.
\end{Remark*}

\subsection{Attività di processo}
\subfile{Sotto Sezioni/Sottosezione 2.1 - Attivita di processo.tex}
\clearpage

\subsection{Modelli di processi software}
\subfile{Sotto Sezioni/Sottosezione 2.2 - Modelli di processi software.tex}
\end{document}