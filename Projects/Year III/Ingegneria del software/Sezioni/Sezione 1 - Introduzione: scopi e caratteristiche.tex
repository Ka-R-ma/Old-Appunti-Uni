\documentclass{subfiles}
\begin{document}
Obbiettivo dell'ingegneria del software è quello di supportare lo sviluppo software professionale.
Questa include tecniche che risultano utili negli aspetti di definizione delle specifiche, alla progettazione del software e l'evoluzione dello stesso.

\begin{Remark*}
    in genere con software si tende ad identificare i programmi per computer. Nel campo dell'ingegneria del software, essa include altri aspetti, quali:
    la documentazione, le librerie, i siti web e i dati di configurazione necessari a rendere utili tali programmi.
\end{Remark*}

\noindent Considerando l'approccio utilizzato dall'ingegneria del software, anche detto \emph{processo software}, questi si compone delle seguenti attività,
approfondite nel seguito.
\begin{itemize}
    \item Specifiche del software.
    \item Sviluppo del software.
    \item Convalida del software.
    \item Evoluzione del software.
\end{itemize}
\end{document}