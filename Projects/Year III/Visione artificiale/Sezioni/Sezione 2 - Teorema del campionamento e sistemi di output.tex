\documentclass{subfiles}
\begin{document}
Sia assunto che l'immagine ammette frequenze massime \(v_{x} \text{e} v_{y}\).
Supponendo di dover campionare l'immagine, è così possibile determinare l'ampiezza campionante ad intervalli spaziali, dati dalle seguenti espressioni.
\[
    \Delta_{x} = \frac{1}{2v_{x}} \qquad \qquad \Delta_{y} = \frac{1}{2 v_{y}}
\]
\noindent Nel caso di pixel quadrati si impone \(\Delta = \min{\Delta_{x}, \Delta_{y}}\).

Parlando di effettivo campionamento, si identificano principalmente tre casi, casi dai quali banalmente dipende la qualità del segnale. Questi sono
\begin{itemize}
    \item \emph{sotto-campionamento}: il numero di campioni del segnale da campionare, non è sufficiente a ricostruire il segnale di partenza;
    \item \emph{campionamento critico}: si campiona il segnale con un numero sufficiente di campioni, permettendo di ripristinare il segnale;
    \item \emph{sovra-campionamento}: il segnale è perfettamente ricostruibile, ma il numero di campioni è eccessivo.
\end{itemize}

\subsection{Sistema di output a scala di grigio}
\subfile{Sotto Sezioni/Sottosezione 2.1 - Sistema di output a scala di grigio.tex}

\subsection{Sistemi di output a colori}
\subfile{Sotto Sezioni/Sottosezione 2.2 - Sistemi di output a colori.tex}
\clearpage
\end{document}