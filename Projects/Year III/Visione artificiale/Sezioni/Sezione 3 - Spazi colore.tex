\documentclass{subfiles}
\begin{document}
Uno \emph{spazio-colore} è la combinazione di un modello di colore e di una appropriata funzione di mappatura di questo modello.
Un modello di colore, infatti, è un modello matematico astratto che descrive un modo per rappresentare i colori come combinazioni di numeri,
tipicamente come tre o quattro valori detti componenti colore.
Tuttavia questo modello è una rappresentazione astratta, per questo viene perfezionato da specifiche regole adatte all'utilizzo che se ne andrà a fare,
creando uno spazio dei colori.

\subsection{Spazio-colore RGB}
\subfile{Sotto Sezioni/Sottosezione 3.1 - Spazio colore RGB.tex}

\subsection{Spazio colore RGB: CCD e filtro di Bayer}
\subfile{Sotto Sezioni/Sottosezione 3.2 - Spazio RGB: CCD e filtro di Bayer.tex}

\subsection{Spazio colore HSL/HSV}
\subfile{Sotto Sezioni/Sottosezione 3.3 - Spazio colore HSL e HSV.tex}

\subsection{Spazio colore YUV}
\subfile{Sotto Sezioni/Sottosezione 3.4 - Spazio colore YUV.tex}

\subsection{Altre nozioni sugli spazi colore}
\subfile{Sotto Sezioni/Sottosezione 3.5 - Altre nozioni sugli spazi colore.tex}
\clearpage
\end{document}