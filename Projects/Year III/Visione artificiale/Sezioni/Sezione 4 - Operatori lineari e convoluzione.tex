\documentclass{subfiles}
\begin{document}
Prima di parlare di convoluzione è necessario fare alcune puntualizzazioni.
Per prima cosa si farà una distinzione tra operazione matriciale e puntuale. Si considerino le seguenti matrici
\[
    \begin{pmatrix}
        a_{11} & a_{12} \\
        a_{21} & a_{22} \\
    \end{pmatrix}
    \qquad
    \begin{pmatrix}
        b_{11} & b_{12} \\
        b_{21} & b_{22} \\
    \end{pmatrix}
\]
per il prodotto matriciale si avrebbe
\[
    \begin{pmatrix}
        a_{11} & a_{12} \\
        a_{21} & a_{22} \\
    \end{pmatrix}
    \cp
    \begin{pmatrix}
        b_{11} & b_{12} \\
        b_{21} & b_{22} \\
    \end{pmatrix}
    =
    \begin{pmatrix}
        a_{11}b_{11} + a_{12}b_{21} & a_{11}b_{12} + a_{12}b_{22} \\
        a_{21}b_{11} + a_{22}b_{21} & a_{21}b{12} + a_{22}b_{22}  \\
    \end{pmatrix}
\]
per quello punto-punto risulta invece
\[
    \begin{pmatrix}
        a_{11} & a_{12} \\
        a_{21} & a_{22} \\
    \end{pmatrix}
    \cp
    \begin{pmatrix}
        b_{11} & b_{12} \\
        b_{21} & b_{22} \\
    \end{pmatrix}
    =
    \begin{pmatrix}
        a_{11}b_{11} & a_{12}b_{12} \\
        a_{21}b_{21} & a_{22}b_{22} \\
    \end{pmatrix}
\]
Ossia nel prodotto punto-punto ad essere moltiplicati sono gli elementi i cui indici coincidono, pertanto le matrici coinvolte devono avere le medesime dimensioni

\begin{Note*}
    per il resto del documento saranno considerate operazioni punto-punto, se non espressamente specificato.
\end{Note*}

\noindent Ulteriore nozione è quella di operatore lineare. Si ricorda che \(H\) è un operatore lineare se
\begin{equation}
    H[\alpha f(x,y) + \beta f(x,y)] = \alpha H[f(x,y)] + \beta H[f(x, y)]
\end{equation}

Di interesse per l'utilizzo di MATLAB, risultano essere le seguenti operazioni lineari.
\[\begin{gathered}
        I^{(0)} = \set{(i,j,g)}{g = 0} \implies \text{Immagine completamente nera.} \\
        I^{(255)} = \set{(i,j,g)}{g = 255} \implies \text{Immagine completamente bianca.} \\
        kI = \set{(i,j,g)}{g = \min{G - 1, \floor{k \cdot g}}} \implies \text{Eventuale saturazione a G} \\
        k + I = \set{(i,j,g)}{g = \min{G - 1, \floor{k + g}}} \implies \text{Eventuale saturazione a G} \\
        \min{I_{1}, I_{2}} = \set{(i,j,g)}{g = \min{g_{1}, g_{2}}} \implies \text{Immagine risultante è più scura} \\
        \max{I_{1}, I_{2}} = \set{(i,j,g)}{g = \max{g_{1}, g_{2}}} \implies \text{Immagine risultante è più chiara} \\
        I_{1} + I_{2} = \set{(i,j,g)}{g = \min{G - 1, g_{1} + g_{2}}} \implies \text{Eventuale saturazione a G} \\
        I_{1} \cp I_{2} = \set{(i,j,g)}{g = \floor{(g_{1} \cdot g_{2})/G - 1}} \implies \text{Eventuale saturazione a G} \\
    \end{gathered}\]
\clearpage

\subsection{Convoluzione}
\subfile{Sotto Sezioni/Sottosezione 4.1 - Convoluzione.tex}

\subsection{Filtro di convoluzione blur}
\subfile{Sotto Sezioni/Sottosezione 4.2 - Filtro di convoluzione blur.tex}
\clearpage

\subsection{Filtro mediano}
\subfile{Sotto Sezioni/Sottosezione 4.3 - Filtro mediano.tex}
\clearpage

\subsection{Filtro di sharpening}
\subfile{Sotto Sezioni/Sottosezione 4.4 - Filtro di sharpening.tex}
\clearpage

\subsection{Filtro gradiente}
\subfile{Sotto Sezioni/Sottosezione 4.5 - Filtro gradiente.tex}
\clearpage

\subsection{Filtri Prewitt e Sobel}
\subfile{Sotto Sezioni/Sottosezione 4.6 - Filtri Prewitt ee Sobel.tex}
\clearpage
\end{document}