\documentclass{subfiles}
\begin{document}
Si considerano ora alcune operazioni geometriche che possibile applicare alle immagini.
Sebbene ne esistono molte altre, di interesse risultano essere le operazioni di traslazione, rotazione, ridimensionamento e interpolazione.

\begin{Note*}
    poiché molto elementare, la traslazione sarà brevemente trattata in questo paragrafo, le altre operazioni saranno analizzate nel dettaglio nei paragrafi successivi.
\end{Note*}

Considerando dunque l'operazione di traslazione: si supponga \(I\) un'immagine, e sia \(I(i, j)\) un certo pixel.
Quel che si fa con l'operazione di traslazione è spostare \(I(i,j)\) in \(I(i', j')\), ove \((i',j') = (i, j) + (h, k)\).

\subsection{Rotazione}
\subfile{Sotto Sezioni/Sottosezione 9.1 - Rotazione.tex}
\end{document}