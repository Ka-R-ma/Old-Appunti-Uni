\documentclass{subfiles}
\begin{document}
L'occhio umano possiede una visione tri-cromatica, permessa come detto in precedenza dai recettori conici.
Tramite rappresentazione RGB, a ciascun pizel è associata una terna\footnotemark[3] di byte, potendo definire \(2^{24}\) colori distinti.
La rappresentazione di tali colori è facilmente gestibile a livello hardware.

\footnotetext[3]{Ad oggi esiste una rappresentazione che fa uso di un quarto bit, per la trasparenza il cosiddetto \emph{alpha channel}.}

\end{document}