\documentclass{subfiles}
\begin{document}
Il sistema a scala di grigi che si considera è il \emph{tubo catodico}.
Questi si compone di un tubo di vetro, mantenuto a bassissima pressione, alle cui estremità sono posti due elettrodi collegati ad un generatore di corrente.
\\ \\
Quando la differenza di potenziale tra gli elettrodi è elevata, e inoltre la pressione scende sotto le \(\SI{e-6}{\atmosphere}\), il vetro di fronte emette luminescenza.
Grazie agli elettronica, con l'uso di magneti è possibile far cambiare direzione al flusso degli elettroni, secondo un percorso \emph{raster\footnotemark[2]};
\\ \\
L'utilizzo di tale tecnologia non permetteva a volte di trasmettere a 25 fotogrammi al secondo, quantità minima di frame affinché l'immagine risulti fluida.
In questi casi si procedeva con una trasmissione interlacciata: si trasmettevano cioè prima tutte le righe dispari, poi quelle pari.
Motivo di tale scelta è il fatto che ad illuminarsi non è unicamente il pixel, quanto più un'areola leggermente più ampia;
facendo così dunque si illuminava anche parte dei pixel delle righe pari.

\footnotetext[2]{L'immagine viene visualizzata a partire dal pixel più in alto a sinistra, procedendo per l'intera riga, e iniziando nuovamente dal pixel più a sinistra della riga successiva.}
\end{document}