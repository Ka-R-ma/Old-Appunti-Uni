\documentclass{subfiles}
\begin{document}
La convoluzione è un operatore lineare, e in quanto tale soddisfa l'\emph{Equazione \ref{eq:1}}.
Essa può essere utilizzata in vari modi, ma tutti sono accomunati da un elemento comune il \emph{kernel}.
In maniera sintetica, si pensi al kernel come una matrice i cui valori fanno da pesi alla convoluzione.
\end{document}