\documentclass{subfiles}
\begin{document}
Come ormai ovvio, i computer del passato non erano in grado di mostrare a video immagini con lo stesso numero di colori di un computer moderno.
In questa sezione si parlerà del metodo con cui il \emph{Commodore Amiga}, nonostante fosse progettato per mostrare solo 16 colori, riusciva a mostrarne 4096.

Parlando di tale tecnica, la \emph{Hold-and-Modify} (o anche HAM6): ciascun colore è rappresentato con 6 bit, divisi come mostrato in \emph{Figura \ref{Fig:5.6}}.
\subfile{../Figure/Figure 5.6 - Rappresentazione codifica colore commodore.tex}
Con questa codifica i bits di controllo indicavano quale dei tre colori modificare, considerando come valori degli altri due, quelli presenti nel pixel adiacente.
Quindi si aveva qualcosa del tipo: control bits posti a 00, allora i bit data rappresentato uno dei colori della tavolozza;
control bits posti a 01, allora modifica il blue, ecc.

\begin{Remark*}
    Il pixel in alto a sinistra necessariamente presenta i bit di controllo posti a 00, ciò non implica che sia l'unica pixel che possa presentare tale situazione.
\end{Remark*}

C'è da notare che tale tecnica presenta un problema: il poter modificare un colore per volta, nel caso di cambi netti di colore può portare a strisciamenti dell'immagine.
Tale problematica fu parzialmente risolta con SHAM, per la quale la palette di 16 colori non era per l'intera immagine, quanto più per ogni linea di pixel presente in essa.

\begin{Note*}\label{Note: }
    per quel che concerne HAM8, questi è una versione successiva dell'HAM6: la tecnica funziona allo stesso modo, solamente che il numero di data bits è 6,
    permettendo così oltre 262.000 colori.
\end{Note*}


\end{document}