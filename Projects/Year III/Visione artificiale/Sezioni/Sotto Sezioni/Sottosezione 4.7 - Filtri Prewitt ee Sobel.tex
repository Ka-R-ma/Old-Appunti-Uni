\documentclass{subfiles}
\begin{document}
Considerando il filtro gradiente sin ora descritto, si osserva che i kernel utilizzati per il calcolo delle due derivate sono molto restrittivi.
Nel caso dei filtri Prewitt e Sobel, anch'essi filtri che effettuano il gradiente di un immagine, il kernel è realizzato cosi da permettere una maggior flessibilità,
sebbene ciò comporta quello che si può definire un ``doppio bordo''.

Per una questione di sinteticità: i due filtri sono analoghi del filtro gradiente precedentemente visto, differendo da questi e reciprocamente,
per la matrice utilizzata come kernel. Queste risultano essere
\[\underbrace{\begin{bmatrix}
            -1 & 0 & 1 \\
            -1 & 0 & 1 \\
            -1 & 0 & 1 \\
        \end{bmatrix}}_{Kernel \ Prewitt} \qquad \text{e} \qquad \underbrace{\begin{bmatrix}
            -1 & 0 & 1 \\
            -2 & 0 & 2 \\
            -1 & 0 & 1 \\
        \end{bmatrix}}_{Kernel \ Sobel}\]
Ovviamente i kernel di cui sopra sono utilizzati per calcolare la componente Dx del rispettivo filtro, le relative trasposte quella Dy.

Si osserva inoltre che i due kernel sono a variabili separabili: cioè ottenibili come prodotto colonna-riga di opportuni vettori;
nel caso dei kernel di Prewitt e Sobel si ha quanto segue.
\[\begin{aligned}
        \begin{bmatrix}
            -1 & 0 & 1 \\
            -1 & 0 & 1 \\
            -1 & 0 & 1 \\
        \end{bmatrix} & = \begin{bmatrix}
                              1 \\
                              1 \\
                              1 \\
                          \end{bmatrix} \otimes \begin{bmatrix}
                                                    -1 & 0 & 1 \\
                                                \end{bmatrix} \text{Prewitt} \\
        \begin{bmatrix}
            -1 & 0 & 1 \\
            -2 & 0 & 2 \\
            -1 & 0 & 1 \\
        \end{bmatrix} & = \begin{bmatrix}
                              1 \\
                              2 \\
                              1 \\
                          \end{bmatrix} \otimes \begin{bmatrix}
                                                    -1 & 0 & 1 \\
                                                \end{bmatrix} \text{Sobel}
    \end{aligned}
\]

Considerando pertanto l'applicazione dei due filtri, se ne riporta a seguire l'effetto su \emph{Figura \ref{fig:4.8}},
più precisamente si riportano in \emph{Figura \ref{fig:4.12}} le componenti dx dell'immagine, a seguito dell'applicazione dei filtri.
\subfile{../Figure/Figure 4.12 - Componendti dx con filtri Sobel e Prewitt.tex}

Come precedentemente detto i filtri differiscono minimamente, risultato di ciò è il fatto che le immagini risultanti appaiono essere pressoché uguali.
\begin{Note*}
    come per il filtro di Gauss, anche per i filtri di Prewitt e Sobel si è effettuata una normalizzazione: si è passati da un range di [-1020, 1020] ad uno di 255,
    per mezzo della segue formula \lstinline[language = MATLAB]{img = 255*(img + 1020)/(2 * 1020)}.
\end{Note*}
\end{document}