\documentclass{subfiles}
\begin{document}
Si supponga \(I\) una certa immagine e sia \(\alpha\) un certo angolo: come suggerisce il nome, l'operazione di rotazione ruota l'immagine di \(\alpha\) gradi.
In generale, a meno di casi particolari\footnotemark[7], l'immagine \(I'\) ottenuta dalla rotazione, risulterà avere dimensioni maggiori rispetto \(I\).

Nascono così due problemi:
\begin{enumerate}
    \item come calcolare la nuova posizione di ciascun pixel in \(I\);
    \item come stabilire le dimensioni di \(I'\).
\end{enumerate}

Per quel che riguarda il calcolo delle nuove coordinate, ciò è realizzato semplicemente utilizzando il seguente sistema lineare
\begin{equation}
    \begin{cases}
        X = \floor{x \cos(\alpha) - y \sin(\alpha)} \\
        Y = \floor{x \sin(\alpha) + y \cos(\alpha)} \\
    \end{cases}
\end{equation}
ove \((X, Y)\) rappresentano le coordinate dell'immagine ruotata.

\begin{Remark*}
    in realtà utilizzando il sistema \eqref{eq:2}, l'immagine risultate presenterà dei puntini neri quando visualizzata, dati dall'approssimazione.
    Per tale ragione si preferisce optare per una mappatura inversa. Cioè supposto \(I'(X, Y)\) un pixel nell'immagine ruotata,
    si cerca ricercano le coordinate \((x, y)\) del pixel corrispondente in \(I\). Ciò è effettuato sfruttando la relazione inversa a quella sopra, data in \eqref{eq:3}.
    \begin{equation}
        \begin{cases}
            x = \floor{X \cos(\alpha) + Y \sin(\alpha)}  \\
            y = \floor{Y \cos(\alpha) -  X \sin(\alpha)} \\
        \end{cases}
    \end{equation}
\end{Remark*}

Considerando ora il come stabilire le dimensioni della nuova immagine, si osserva che indipendentemente dall'immagine, in un certo qual senso,
a determinare le dimensioni sono le posizioni dei quattro pixel agli angoli. Da tale osservazione segue che per stabilire le nuove dimensioni,
assumendo di aver posto il riferimento dell'immagine al centro della stessa, che queste sono a loro volta dipendenti dalla posizione ruotata dei pixel agli angoli.
Più precisamente, per una questione di semplicità espositiva, posti \(A, B, C, D\) i pixel agli angoli, secondo un senso orario,
le nuove dimensioni sono definite come \([2 \abs{A_{x}}, 2 \abs{B_{y}}]\), con \(A_{x} \text{e} B_{y}\) rispettivamente le posizione in \(I'\) di \(A, B\) lungo i due assi.
\end{document}