\documentclass{subfiles}
\begin{document}
Si supponga \(I\) una certa immagine e sia \(\alpha\) un certo angolo: come suggerisce il nome, l'operazione di rotazione, procede a ruotare l'immagine di \(\alpha\) gradi.
In generale, a meno di casi particolari\footnotemark[7], l'immagine \(I'\) ottenuta a seguito della rotazione, risulterà avere dimensioni maggiori rispetto \(I\).
Nascono così due problemi:
\begin{enumerate}
    \item come calcolare la nuova posizione di ciascun pixel in \(I\);
    \item come stabilire le dimensioni di \(I'\).
\end{enumerate}

Per quel che riguarda il calcolo delle nuove coordinate, ciò è realizzato semplicemente utilizzando il seguente sistema lineare
\begin{equation}
    \begin{cases}
        X = \floor{x \cos(\alpha) - y \sin(\alpha)} \\
        Y = \floor{x \sin(\alpha) + y \cos(\alpha)} \\
    \end{cases}
\end{equation}
ove \((X, Y)\) rappresentano le coordinate dell'immagine ruotata.

\begin{Remark*}
    è giusto puntualizzare che utilizzando l'\emph{Equazione \eqref{eq:2}}, una volta visualizzata,
    l'immagine risultante presenterà dei mancamenti di colore, mancamenti che si presentano come puntini neri.
    Il presentarsi di tali puntini è banalmente spiegato dal fatto che le nuove coordinate sono ottenute per arrotondamento, dunque introducendo un errore.
\end{Remark*}

Dall'osservazione appena fatta, deriva la necessità di poter calcolare le nuove coordinate in altra maniera.
Per una questione di semplicità espositiva, in breve, quel che si fa è partire dall'immagine ruotata, e per ciascun pixel di coordinate \((X, Y) \text{in} I'\),
si ricerca in \(I\) il pixel di coordinate \((x, y)\) tali da soddisfare l'\emph{Equazione \eqref{eq:2}}, applicando una relazione inversa a quest'ultima a seguito riportata.
\[\begin{cases}
        x = \floor{X \cos(\alpha) + Y \sin(\alpha)}  \\
        y = \floor{Y \cos(\alpha) -  X \sin(\alpha)} \\
    \end{cases}\]
Passando ora a considerare come stabilire le dimensioni della nuova immagine, si parta col fare un'osservazione in apparenza banale.
Considerando una qualsiasi immagine, sia questa quadrata o rettangolare, si ha che, denominando \(A, B, C, D\) i quattro pixel agli angoli secondo una disposizione oraria,
e ponendo il riferimento dell'immagine al suo centro, la larghezza dell'immagine è data dalla differenza, considerata in modulo, delle coordinate in \emph{x} tra \(A \text{e} B\),
equivalentemente per \(C \text{e} D\).
Analogamente l'altezza è data dalla differenza, anch'essa da considerare in modulo, tra le coordinate in \emph{y} di \(A \text{e} D\), equivalentemente per \(B \text{e} C\).
Segue spontaneo chiedersi dunque se quanto detto, valga anche nel caso di immagini ruotate. La risposta a tale domanda risulta essere affermativa.
Da quanto detto segue che per determinare le nuove dimensioni dell'immagine ruotata, basta calcolare le posizioni dei pixel \(A, B, C, D\) e applicare quanto detto in precedenza.

\footnotetext[7]{Si consideri il caso di un immagine quadrata ruotata di 90 gradi, banalmente questa mantiene le medesime dimensioni. %
    Discorso simile vale nel caso di immagini rettangolari ruotate di 90 gradi, queste semplicemente scambieranno altezza con larghezza e viceversa.}
\end{document}