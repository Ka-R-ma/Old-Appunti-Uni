\documentclass{subfiles}
\begin{document}

Il formato GIF venne sviluppato per la prima volta nel 1987. La nascita di tale formato, di cui a seguito si spiegherà il funzionamento,
era dovuto al fatto che il formato BMP, allora utilizzato per la trasmissioni di immagini, non era sufficientemente prestate:
poiché internet si stava sempre più diffondendo era necessario che la trasmissione delle immagini avvenisse quanto più velocemente possibile.

\begin{Note*}
    premettendo che il formato trasmette immagini indicizzate, quest'ultime possono essere a 256, 16 o 2 colori.
    Nel resto della discussione sarà considerato il caso di immagini a 256 colori, ma un ragionamento analogo lo si puo applicare per gli altri casi.
\end{Note*}

Analizzando pertanto il formato, alla base di questi vi è l'algoritmo di Lempel-Ziv-Welch.
Quello che fa l'algoritmo è, a partire dagli indici dell'immagine da trasmettere, crea un dizionario (lo si pensi ad un array bidimensionale).
Partendo dal pixel in alto a sinistra, si legge l'indice a cui questi fa riferimento, essendo ovviamente questi uno dei 256 indici di partenza,
si trasmette il pixel, ripetendo lo stesso ragionamento per il pixel alla sua destra. Letti questi due indici,
l'algoritmo suppone che la successione dei due si ripeterà nell'immagine, è crea così un nuovo indice, il 257-esimo, che conterrà gli indici ai due pixel appena trasmessi.
Procede con il leggere l'indice del pixel successivo, verifica a se questi rientra nei primi 256, e lo trasmette.
Dopo la trasmissione di quest'ultimo, si presentano due possibilità: un'indice \(I\) del dizionario contiene già gli indici del pixel tramesso e del suo successivo,
in questo caso si ``annulla'' la trasmissione del l'indice del pixel e si trasmette l'indice \(I\);
nel dizionario non è presente nessun indice che faccia riferimento all'indice del pixel e al suo successivo,
si crea allora un indice \(I'\) che conterrà gli ultimi due indici trasmessi.

\begin{Example*}
    Con il mero scopo di dare un'illustrazione grafica dell'algoritmo precedentemente descritto, si consideri \emph{Figura \ref{fig:6.1}}.
    \subfile{../Figure/Figure 6.1 - Dizionario GIF.tex}

    Come detto i due pixel in alto a sinistra sono trasmessi appena letti, così anche il pixel 15, poiché non esiste ancora un'indice che lo contenga.
    Successivamente letto nuovamente il pixel 33, che dovrebbe essere trasmesso, si verifica che il pixel che lo segue non dia vita ad un indice del dizionario,
    poiché lo fa, anziché trasmettere 33 si trasmette 256.
\end{Example*}

Quanto sinora descritto è noto come \emph{GIF 87a}: denominato così per via dell'anno di rilascio e poiché si supponeva di rilasciarne una nuova versione nello stesso anno.
Tale release però non avvenne, infatti la successiva versione del formato richiese ulteriori due anni di sviluppo, anni che portarono all'ormai consolidato \emph{GIF 89a}.

\begin{Remark*}
    GIF 87a presenta un sostanziale problema, via via che si creano nuovi indici, questi richiedono sempre più bit per essere rappresentati,
    venendo così meno la compressione. Per risolvere tale problema vi sono due possibilità: la prima consiste nell'ignorarlo; la seconda prevede, raggiunta una soglia limite,
    di trasmettere una segnale con il quale si resetta il dizionari.
\end{Remark*}
\hfill\\

Analizzando la nuova versione, GIF 89a introdusse le seguenti funzionalità.
\begin{itemize}
    \item Trasmissione interlacciata: era necessario poter visualizzare le immagini sempre più velocemente, si penso allora di trasmettere le immagini secondo quanto segue.
          Si trasmette la prima riga per altre 15, poi la 17-esima per altre 15 righe e così via fino a giungere alla fine dell'immagine.
          Giunti alla fine, a partire dalla prima ripetizione, si trasmette la riga che nell'immagine originale equivale a quella di mezzo della ripetizione.

    \item Trasparenza: da non confondere con il fattore alpha, questi stabiliva se nella trasmissione dell'immagine si dovesse o meno visualizzare un determinato pixel.
    \item Animazioni: si penso sfruttando la trasparenza, di includere delle animazioni.
          L'idea nasceva dal fatto che di supponeva che i fotogrammi componenti l'immagine differissero per pochi dettagli,
          dunque sfruttando la trasparenza si poteva trasmettere solo quell'insieme di pixel che formavano il dettaglio.
\end{itemize}
\clearpage
\end{document}