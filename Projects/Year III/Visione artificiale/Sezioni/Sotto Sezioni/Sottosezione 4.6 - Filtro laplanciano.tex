\documentclass{subfiles}
\begin{document}
Similarmente a quanto fatto con il filtro gradiente, si consideri la definizione matematica di operatore di Laplace.
Questa, nel caso di funzioni a due variabili, da un punto di vista analitico risulta essere
\[
    \laplacian{I} = \pdv[2]{I}{x} + \pdv[2]{I}{y}
\]
Considerando ora le componenti della sua controparte discreta, per quanto detto precedentemente circa le derivate parziali nel discreto, questi risulta essere
\[\begin{aligned}
        \pdv[2]{I}{x} & \approx \pdv{(I(i, j + 1) - I(i,j))}{x} = \cdots = I(i, j + 2) - 2I(i, j + 1) + I(i, j) \\
        \pdv[2]{I}{y} & \approx \pdv{(I(i + 1, j) - I(i,j))}{y} = \cdots = I(i + 2, j) - 2I(i + 1, j) + I(i, j) \\
    \end{aligned}\]
Si osserva però che il kernel, se realizzato cosi non è centrato in I(i,j) come supposto, per tale ragione ciò che si fa è traslare lo stesso in I(i,j),
secondo quanto segue.
\[\begin{aligned}
        \pdv[2]{I}{x} & \approx I(i, j + 2) - 2I(i, j + 1) + I(i, j) = \Delta^{2}_{x}I = \begin{bmatrix}
                                                                                             1 & -2 & 1 \\
                                                                                         \end{bmatrix} \\
        \pdv[2]{I}{y} & \approx I(i + 2, j) - 2I(i + 1, j) + I(i, j) = \Delta^{2}_{y}I = \begin{bmatrix}
                                                                                             1  \\
                                                                                             -2 \\
                                                                                             1  \\
                                                                                         \end{bmatrix} \\
    \end{aligned}\]
\end{document}