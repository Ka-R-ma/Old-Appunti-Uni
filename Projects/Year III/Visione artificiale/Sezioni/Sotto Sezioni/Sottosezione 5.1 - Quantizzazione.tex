\documentclass{subfiles}
\begin{document}
La quantizzazione è un processo che permette, a partire da un'immagine true color, di ottenere un immagine riconducibile all'originale,
il cui numero di colori è però significativamente ridotto.

Per quanto visto sinora circa gli spazi colore, le immagini true color possono essere pensate come un parte dello spazio RGB, che come noto contiene oltre 16 milioni di colori.
Si osserva però che, a meno di risoluzione elevate, le immagini non presenteranno mai tutti i colori: considerando ad esempio un immagine 512 x 512,
questa ammette al più circa 262 mila colori distinti,
quindi rappresentare le stesse come immagini RGB comporta uno spreco significativo di memoria.
Tale spreco ad oggi può risultare trascurabile date le capacità dei moderni sistemi di storage,
si pensi però che la quantizzazione nasce in un periodo storico in cui le memorie avevano dimensioni molto limitate, qualche centinaio di kB, qualche MB nei casi migliori.
\`E dunque ovvio che all'epoca era necessario ottimizzare quanto più possibile l'uso della memoria, da cui segue la nascita della quantizzazione.

Parlando del processo di quantizzazione in se\footnotemark[5]: punto di partenza è il cubo RGB, che si assume essere l'intera immagine;
il cubo è tagliato lungo il punto medio dell'asse più lungo, assicurando che un eguale numero di colori sia assegnato a ciascuno dei due nuovi cubi.
Si applica ricorsivamente la procedura ai due nuovi cubi, fintanto che si ottengo 256 cubi. Fatto ciò, per ciascun cubo si considera il centroide (in alcune varianti i medoide),
che rappresenterà uno dei 256 colori da utilizzare nella tavolozza.

\footnotetext[5]{Sebbene sarà discussa la \emph{median-cut color quantization}, questa non è l'unica tecnica applicabile.%
    Ulteriori tecniche sono il \emph{nearest color algorithm} e soluzioni basate su gli \emph{octrees}.}
\end{document}