\documentclass{subfiles}
\begin{document}
Partendo da definire il concetto di \emph{wavelet}; queste sono funzioni il cui ``moto'' è di tipo oscillatorio.
Dal concetto di funzione wavelet nascono le trasformate wavelet.
\\ \\
Da un punto di vista matematico, una trasformata wavelet permette di rappresentare una funzione integrabile per quadrati mediante una determinata serie ortonormale generata da una wavelet,
soddisfacendo proprietà che non risultano di interesse. Dal punto di vista pratico, ciò comporta la possibilità di decomporre il segnale tramite l'uso di filtri passa-alto e passa-basso,
definiti in funzione della wavelet, e successivamente ripristinarlo. Tale ricomposizione è effettuata moltiplicando per opportuni valori, i cosiddetti \emph{coefficienti wavelet},
le cosiddette funzioni madre, che definiscono la portante del segnale, e funzioni di scala.

Per quanto riguarda le DWT, queste sono un sottoinsieme delle trasformate wavelet in cui la funzione wavelet è campionata in modo discreto.

\begin{Note*}
    da un punto di vista teorico, le DWT sono in numero finito; tale numero è però così elevato da risultare infinito dal punto di vista pratico.
\end{Note*}

\subsubsection{Trasformata Haar}
Proposta per la prima volta nel 1909 da \emph{Alfred Haar}, da cui prende il nome, è la prima delle trasformate wavelet, la quale oltre a prestarsi bene per i segnali ad onda quadra,
risulta facilmente implementabile.

Essendo una trasformata wavelet, Haar possiede una propria funzione madre e una propria funzione di scala, a seguito riportate.
\[\underbrace{\varphi = \begin{cases}
            1, & \quad 0 \le t < 1       \\
            0, & \quad \text{altrimenti} \\
        \end{cases}}_{\text{funzione madre}} \qquad \underbrace{\psi = \begin{cases}
            1,  & \quad 0 \le t < \frac{1}{2} \\
            -1, & \quad \frac{1}{2} \le t < 1 \\
            0,  & \quad \text{altrimenti}
        \end{cases}}_{\text{funzione di scala}}\]

Come detto Haar è facilmente implementabile, motivo di ciò è il fatto che sfrutta unicamente semi-somme e semi-differenze\footnotemark[9], come si può vedere da \emph{Figura \ref{fig:10.1}}.
Per comprendere il funzionamento nel caso monodimensionale, l'estensione al caso bidimensionale è immediata poiché si ha separabilità delle variabili, si consideri quanto segue.

Si consideri \(v\) un certo vettore di numeri, la cui lunghezza è una qualche potenza di due, la trasformata si riduce a calcolare passo dopo passo semi-somme e semi-differenze\footnotemark[15],
fintanto che si ottengono vettori di lunghezza unitaria.

\subfile{../Figure/Firgure. 10.1 - Implementaione Haar monodimensionale.tex}

\footnotetext[9]{Data una coppia di numeri dicasi semi-somma, la media della somma dei due; analogamente dicasi semi-differenza, la media della differenza tra i due.}



\end{document}