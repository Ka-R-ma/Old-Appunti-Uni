\documentclass{subfiles}
\begin{document}
Si parta ora con il considerare le due operazione alla base della morfologia, quali l'\emph{erosione} \(\varepsilon\) e la \emph{dilatazione} \(\delta\).
Tali operazioni sono definite rispettivamente come
\[
    \varepsilon_{S}(I(\vb{p})) = \min\limits_{\vb{b} \in S}{I(\vb{p} + \vb{q})} \qquad \delta_{S}(I(\vb{p})) = \max\limits_{\vb{b} \in S}{I(\vb{p} + \vb{q})}
\]
ove \(S\) rappresenta è la matrice che rappresenta l'elemento strutturale da ricercare nell'immagine,
\(\vb{p}, \vb{q}\) sono vettori: nel caso di \(\vb{p}\) questi rappresenta le coordinate di un pixel nell'immagine;
\(\vb{q}\) invece rappresenta le coordinate di un elemento in \(S\).

\begin{Note*}
    per una mera questione di semplicità per il resto della sezione saranno considerati elementi strutturali di forma circolare.
    In generale si dovrebbero considerare tutte le orientazioni che l'elemento potrebbe assumere nell'immagine.
\end{Note*}

Per comprendere gli effetti delle due operazioni, sia \(I\) l'immagine di \emph{Figura \ref{fig:4.1}}.
Quanto ne risulta dall'applicazione delle due operazioni è mostrato in \emph{Figura \ref{fig:7.1}}.
\subfile{../Figure/Figure 7.1 - Esempio erosione e dilatazione.tex}

Analizzando brevemente gli effetti delle due operazioni, quel che accade con l'erosione è l'eliminazione dei dettagli le cui forme e dimensioni sono somigliano a quelle di \(S\);
viceversa con la dilatazione tali dettagli vengono ``allargati''.
\end{document}