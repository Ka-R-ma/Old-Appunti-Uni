\documentclass{subfiles}
\begin{document}
Dalle due operazioni di erosione e dilatazione si definiscono altre due operazioni: l' \emph{apertura} e la \emph{chiusura}.
Queste, definite come segue, sono operazioni che nascono dal fatto che erosione e dilatazione \underline{\textbf{non}} sono l'una inversa all'altra.
\[
    \gamma_{S}(I) = \delta_{S}(\varepsilon_{S}(I)) \qquad \varphi_{S}(I) = \varepsilon_{S}(\delta_{S}(I))
\]
Quanto appena detto è difatti sostenuto dal fatto che se con l'erosione dei dettagli vengono completamente rimossi, la dilatazione non può ovviamente rigenerarli.
L'unico legame che si dimostra esservi tra erosione e dilatazione è il fatto che, l'una equivale al completamento dell'altra. Cioè
\[
    \varepsilon_{S}(I) = \overline{\delta_{S}(I)} \qquad \delta_{S}(I) = \overline{\varepsilon_{S}(I)}
\]

Tornando a parlare delle operazioni di apertura e chiusura: a livello pratico l'apertura è utilizzata per rimuovere dall'immagine piccoli dettagli,
mentre la chiusura è utilizzata per chiudere piccoli buchi.
\subfile{../Figure/Figure 7.2 - Esempio apertura e chiusura.tex}

In \emph{Figura \ref{fig:7.2}} sono mostrate rispettivamente l'applicazione dell'operazione di apertura e di chiusura a \emph{Figura \ref{fig:4.1}}.

\begin{Remark*}
    si tenga presente che l'esecuzione di entrambe le operazioni applicano la seguente logica: l'operazione interna è effettuata normalmente,
    quella esterna invece è effettuata ribaltando lungo ambo gli assi l'elemento strutturale.
\end{Remark*}

Vale in generale la seguente relazione d'ordine
\[
    \varepsilon_{S}(I) \le \gamma_{S}(I) \le I \le \varphi_{S}(I) \le \delta_{S}(I)
\]
per la quale è possibile definire le operazioni descritte nella seguente sezione.

\begin{Note*}
    per quel che riguarda il codice MATLAB per i due operatori, e per gli operatori \(\varepsilon \text{e} \delta\), sebbene sia possibile implementarli secondo quanto detto,
    in MATLAB sono gia definite rispettivamente tramite le funzioni \lstinline[language = MATLAB]{imerode, imdilate, imopen, imclose}.
\end{Note*}
\end{document}