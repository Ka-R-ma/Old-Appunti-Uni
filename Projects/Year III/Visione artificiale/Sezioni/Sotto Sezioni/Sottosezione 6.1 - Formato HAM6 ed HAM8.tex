\documentclass{subfiles}
\begin{document}
Come ormai ovvio, i computer del passato non erano in grado di mostrare a video immagini con lo stesso numero di colori di un computer moderno.
In questa sezione si parlerà del metodo con cui il \emph{Commodore Amiga}, sebbene possedeva una tavolozza di soli 16 colori, riusciva a mostrarne immagini con 4096.

Tale sproporzione nel numero di colori era permesso dal formato grafico \emph{Hold-and-Modify}, anche noto come HAM6.
Con tale formato ciascun colore era codificato con 6 bits: due, detti \emph{control bits}, e i restanti quattro detti \emph{data bits}.
Funzione dei control bits era quella di stabilire cosa rappresentassero i data bits, permettendo le seguenti quattro possibilità:
\begin{enumerate}
    \item i bit di controllo sono posti a 00: ciò sta ad indicare che il colore di quel pixel è uno dei 16 colori della tavolozza,
          allora si trattano i data bits come indice nella palette;
    \item i bit di controllo sono posti a 01: modifica il blu, i valori di verde e rosso sono quelli del pixel precedente;
    \item i bit di controllo sono posti a 10: modifica il rosso, i valori di verde e blue sono quelli del pixel precedente;
    \item i bit di controllo sono posti a 11: modifica il verde, i valori di rosso e verde sono quelli del pixel precedente.
\end{enumerate}

Tale formato presenta un grave problema: nel caso di cambi netti di colore si verificano delle ``strisciate'' nel colore,
questo ovviamente perché può cambiare un solo colore per volta.
Per risolvere tale problema, o quantomeno ridurre la possibilità che questi si verificasse, fu sviluppato SHAM6.
Con tale formato la tavolozza non era più da utilizzare per l'intera immagine, quanto più di riga:
cioè per ogni riga dell'immagine si creava una tavolozza colori propria della riga,
secondo l'idea che verosimilmente buona parte dei pixel della riga avrebbero avuto un colore presente nella tavolozza.
\\ \\
Con l'evolversi della tecnologia si passo da SHAM6 ad HAM8.
Questi adoperando analogamente ad HAM6, semplicemente aggiungendo due data bits, permetteva da una tavolozza di 64 colori, di mostrarne oltre 262 mila.

\end{document}