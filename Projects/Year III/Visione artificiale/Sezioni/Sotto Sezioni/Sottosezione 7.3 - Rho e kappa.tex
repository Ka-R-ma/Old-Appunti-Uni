\documentclass{subfiles}
\begin{document}
Come precedentemente anticipato, per la relazione di cui sopra, si definiscono gli operatori \(\rho \text{e} \kappa\).
Partendo con il considerare \(\rho\), definito come segue, può essere utilizzato per definire i contorni dell'immagine.
\[
    \varrho_{S} = \gamma_{S}(I) - \varepsilon_{S}(I)
\]
Per quanto detto, segue che l'operatore morfologico \(\varrho\) è simile al filtro gradiente, precedentemente studiato.
L'effettiva similarità è dimostrata, in maniera esemplificativa, in \emph{Figura \ref{fig:7.3}}.
\subfile{../Figure/Figure 7.3 - Similarita tra operatore rho e filtro gradietne.tex}

Considerando l'implementazione in MATLAB, l'immagine di \emph{Figura \ref{fig:7.3}.a} è ottenuta tramite l'esecuzione del seguente codice.
\begin{center}
    \begin{lstlisting}[language = MATLAB]
        % caricamento di LenaGS
        struct = [0 1 1 1 0; 1 1 1 1 1; 1 1 1 1 1; 1 1 1 1 1; 0 1 1 1 0];
        rho = LenaGS - imopen(LenaGS, struct);
        figure; imshow(rho, []);
    \end{lstlisting}
\end{center}

\begin{Remark*}
    per quanto simili, l'operatore rho e il filtro gradiente possono differire. Tale differenza banalmente può essere data dal livello di threshold dato al filtro gradiente,
    che si ricorda essere la lunghezza minima dei vettori che compongono il gradiente, oppure dalla forma e dalle dimensioni dell'elemento strutturale usato.
\end{Remark*}

Considerando ora l'operatore \(\kappa\), questi definito come segue, può essere utilizzato per aumentare il contrasto di un immagine
\[
    \kappa = \max\Set{0, \min\Set{255, I + th - bh}}
\]
ove \(th \text{e} bh\) sono i due operatori \emph{top-hat \emph{e} bottom-hat}: si pensi a questi come dei filtri passa alto,
secondo una definizione simile a quanto fatto con il filtro di sharpening. Definendo pertanto tali operatori si ha
\[
    th = I - \gamma_{S}(I) \qquad bh = \varphi_{S}(I) - I
\]
\begin{Note*}
    gli operatori \(th\) e \(bh\), se usati in maniera indipendente, evidenziano dettagli chiari su sfondo scuro per quel che riguarda \(th\),
    viceversa per \(bh\).
\end{Note*}
\clearpage

Con le considerazioni fatte per l'operatore \(\varrho\), risulta evidente la similarità tra l'operatore \(\kappa\) ed un filtro di sharpening.
Come per l'operatore \(\varrho\), di seguito si riporta un esempio di tale similarità.
\subfile{../Figure/Figure 7.4 - Similarita tra operatore kappa e filtro di sharpening.tex}

Per quel che concerne il codice MATLAB, questi di seguito riportato, segue banalmente la definizione stessa di \(\kappa\).
\begin{center}
    \begin{lstlisting}[]
        % caricamento di LenaGS
        struct = [0 1 1 1 0; 1 1 1 1 1; 1 1 1 1 1; 1 1 1 1 1; 0 1 1 1 0];
        top = imtophat(LenaGS, struct);
        bottom = imbothat(LenaGS, struct);
        kappa = max(0, min(255, LenaGS + top - bottom))
        figure; imshow(LenaGS, []);
    \end{lstlisting}
\end{center}

\begin{Note*}
    Per una questione di velocità nella scrittura del codice si è deciso di utilizzare le funzioni \lstinline[language = MATLAB]{imtophat, imbothat} pre-definite in MATLAB.
    Inoltre, tale scelta è dovuta al fatto che, in generale, queste risultano maggiormente ottimizzate rispetto quelle definite dall'utente.
\end{Note*}
\end{document}