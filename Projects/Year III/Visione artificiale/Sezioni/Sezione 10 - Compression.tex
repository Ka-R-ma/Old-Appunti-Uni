\documentclass{subfiles}
\begin{document}
In generale quando si deve comprimere un file, oltre che a tener conto del fattore di compressione,
gli algoritmi di compressione tengo anche conto dell'errore introdotto dall'operazione di compressione stessa.
Indipendentemente dall'algoritmo utilizzato, nel caso di compressione di immagini digitali, due misure importanti risultano essere
\[\begin{aligned}
        \text{\underline{bit per pixel}}         & \equiv \frac{C}{N}  \\
        \text{\underline{ratio di compressione}} & \equiv \frac{kN}{C}
    \end{aligned}\]
ove \(C\) indica il file compresso espresso in bit, \(N\) il numero di pixel e \(k\) il numero di bit per pixel nell'immagine originale.

\begin{Note*}
    le informazioni aggiuntive date dall'header, in generale, non sono considerate.
\end{Note*}

\subsection{Misure di qualità}
\subfile{Sotto Sezioni/Sottosezione 10.1 - Misure di qualita.tex}


\end{document}