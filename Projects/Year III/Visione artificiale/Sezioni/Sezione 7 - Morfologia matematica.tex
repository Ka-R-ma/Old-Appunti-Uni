\documentclass{subfiles}
\begin{document}
La \emph{morfologia matematica} nasce in Francia nella metà degli anni `60.
Questa inizialmente concepita per l'analisi di immagini in bianco e nero, successivamente si evolse con lo scopo di applicarla anche ad immagini a scale di grigio.
Circa l'utilità della morfologia, questa può essere utilizzata per evidenziare (o eliminare) alcune strutture nell'immagine.
\\ \\
Considerando il modello di funzionamento della morfologia, questi a partire da un immagine \(I\) e un \emph{elemento strutturale\footnotemark[6]},
crea una nuova immagine \(I'\) contenente il risultato dell'operazione di morfologia.

\footnotetext[6]{Si tratta di opportune matrici binarie che definiscono forma e dimensioni dei dettagli da ricercare.}

\subsection{Erosione e dilatazione}
\subfile{Sotto Sezioni/Sottosezione 7.1 - Erosione e dilatazione.tex}
\clearpage

\subsection{Apertura e chiusura}
\subfile{Sotto Sezioni/Sottosezione 7.2 - Apertura e chiusura.tex}
\clearpage

\subsection{Rho e kappa}
\subfile{Sotto Sezioni/Sottosezione 7.3 - Rho e kappa.tex}
\clearpage
\end{document}