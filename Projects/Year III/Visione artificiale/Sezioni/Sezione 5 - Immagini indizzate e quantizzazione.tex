\documentclass{subfiles}
\begin{document}
Considerando le immagini sinora viste, queste sono immagini cosiddette \emph{true color}, convertite in scala di grigio.
Si osserva che la tecnologia odierna non è la stessa di 20/30 anni fa: all'epoca non era possibile rappresentare immagini con gli oltre 16 milioni di colori,
che un'immagine true color permette oggigiorno. Per ovviare a tale problema si sono sviluppate le \emph{immagini indicizzate}.
Queste sono immagini con un ridotto numero di colori\footnotemark[4], in cui ciascun elemento della matrice corrispondente all'immagine non indica un livello di grigio,
bensì funge da indice per una seconda matrice, la \emph{tavolozza colori}, che specifica l'effettiva terna di colori RGB da utilizzare.

Nasce da ciò un problema: come selezionare i 256 colori in modo che il risultato sia riconducibile all'originale?
Risposta a tale quesito è la \emph{quantizzazione}, spiegata a seguire.

\footnotetext[4]{Sarà considerato il caso di immagini indicizzate a 256 colori.}

\subsection{Quantizzazione}
\subfile{Sotto Sezioni/Sottosezione 5.1 - Quantizzazione.tex}
\clearpage

\subsection{Dithering}
\subfile{Sotto Sezioni/Sottosezione 5.2 - Dithering.tex}
\clearpage

\subsection{Cenni ad HAM6 ed HAM8}
\subfile{Sotto Sezioni/Sottosezione 5.3 - Cenni ad HAM6 ed HAM8.tex}
\clearpage
\end{document}