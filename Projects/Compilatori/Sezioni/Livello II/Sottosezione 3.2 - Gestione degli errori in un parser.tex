\documentclass{subfiles}
\begin{document}
Affinché la compilazione sia corretta, è necessario che un parser sia in grado di scoprire, diagnosticare e correggere efficientemente gli errori,
così da riprendere l'analisi quanto prima.
Sia supposto un parser che abbia rilevato un errore, resta il problema di come procedere per risolverlo.
In generale si utilizza una delle seguenti tecniche.
\begin{itemize}
    \item \textbf{Panic mode:} l'idea è quella di saltare simboli fintantoché non si legge un token di sincronizzazione (eg. begin-end).
          L'efficacia dipende fortemente dalla scelta dei token, scelta che può essere effettuata euristicamente.

    \item \textbf{Phrase level:} si fa in modo che il parser proceda a correzioni locali.
          Per far ciò, inevitabilmente si procederà ad alterare lo stack.
\end{itemize}
\end{document}