\documentclass{subfiles}
\begin{document}
Un parser ascendente (o bottom-up), è un parser che a partire dalla stringa di input, tenta di risalire all'assioma.
La definizione di ``bottom-up'', deriva dal fatto che la costruzione dell'albero sintattico procede dalle foglia verso la radice.

\begin{Remark*}
    Tutti i parser bottom-up procedono cercando  derivazioni right-most della grammatica.

    \begin{Example*}
        sia supposta la seguente grammatica
        \[\begin{aligned}
                 & \Grammar{E}{E}[E + T] \\
                 & \Grammar{T}{int}[(E)]
            \end{aligned}\]
        e si supponga di dover analizzare \(int  + (int + int + int)\).
        Di interesse risulta verificare che tale stringa sia generabile con la grammatica di cui sopra.
        Osservando la stringa in input si nota che
        \[\begin{aligned}
                  & int + (int + int + int) \\
                = & E + (int + int + int)   \\
                = & E + (E + int + int)     \\
                = & E + (E + T + int)       \\
                = & E + (E + int)           \\
                = & E + (E + T)             \\
                = & E + (E)                 \\
                = & E + T                   \\
                = & E
            \end{aligned}\]
        per cui si conclude, essendo risaliti all'assioma, che la stringa è derivabile.
    \end{Example*}
\end{Remark*}


\end{document}