\documentclass{subfiles}
\begin{document}
Come detto Earley accetta qualsiasi CFG, più nello specifico: data \(\List{x}{1}{n}\) una stringa,
scandendo la stessa da sinistra a destra, per ogni \(x_{i}\) si costruiscono stati \(S_{j}\), stati del tipo \lstinline{(dotted_rule, address)}.
Qui \lstinline{dotted_rule} sta ad indicare una produzione della grammatica alla cui destra è posto un punto,
per tenere traccia della posizione della ``sotto-stringa'' esaminata. Con \lstinline{address} si indica invece la posizione del punto.

\begin{Example*}
    si supponga uno stato \((\Grammar{A}{\alpha . \beta}, i)\). Ciò sta ad indicare che si è esaminata la sotto-stringa \(\alpha\).
\end{Example*}

\noindent Si riporta di seguito lo pseudo codice per implementare Earley.
\subfile{../../Figure/Tikz Figure/Figure 2 - Pseudo-codice Earley.tex}

\subsubsection{Scanner}
\subfile{../Livello III/Sottosottosezione 3.1.1 - Scanner.tex}

\subsubsection{Predictor}
\subfile{../Livello III/Sottosottosezione 3.1.2 - Predictor.tex}

\subsubsection{Completed}
\subfile{../Livello III/Sottosottosezione 3.1.3 - Completed.tex}

\subsubsection{Complessità di Earley}
\subfile{../Livello III/Sottosottosezione 3.1.4 - Complessita di Earley.tex}
\end{document}