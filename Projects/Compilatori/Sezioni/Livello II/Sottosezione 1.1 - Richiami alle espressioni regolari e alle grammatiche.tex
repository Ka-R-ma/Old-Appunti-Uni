\documentclass{subfiles}
\begin{document}
Poiché i concetti di Regex e grammatiche CF sono alla base della definizione di un compilatore,
si riprende a seguito la definizione delle stesse.
\begin{Definition*}
    Si definisce \emph{espressione regolare}, (o RegEx), la descrizione algebrica delle stringhe di un dato linguaggio.
\end{Definition*}
\noindent In particolare, la costruzione di una RegEx \(e\) è di tipo ricorsivo. Si ha infatti che
\begin{itemize}
    \item se \(\varepsilon \text{e} \varnothing\) sono espressioni regolari, ove \(L(\varepsilon) = \Set{\varepsilon}, L(\varnothing) = \Set{}\);
    \item se \(\alpha\) è un simbolo, allora questi è una RegEx, ove \(L(\alpha) = \Set{\alpha}\);
\end{itemize}
da queste
\begin{itemize}
    \item se \(e \text{ed} f\) sono due RegEx. Allora \(e + f\) è un'espressione regolare;
    \item se \(e \text{ed} f\) sono due RegEx. Allora \(ef\) è un'espressione regolare;
    \item se \(e\) è una RegEx. Allora \(e^{*}\) è un'espressione regolare;
    \item se \(e\) è una RegEx. Allora \((e)\) è un'espressione regolare.
\end{itemize}

\begin{Definition*}
    Dato \(T\) un certo alfabeto, si definisce la seguente quadrupla
    \[
        G = (T, N, S, P)
    \]
    grammatica. Nello specifico: \(T\) indica l'insieme dei simboli terminali della grammatica, \(N\) quell dei non terminali,
    \(S\) è l'assioma e \(P\) l'insieme delle regole di produzione.
\end{Definition*}
\end{document}