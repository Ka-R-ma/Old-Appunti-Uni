\documentclass{subfiles}
\begin{document}
Si tratta di un parser che fa uso di un automa a pila, pila che inizialmente contiene unicamente il simbolo $\$$
ad indicare appunto la pila vuota. Ricevuto un input, anch'esso contenente $\$$ ad indicarne la fine,
sulla base dello stesso effettua quattro operazioni.
\begin{itemize}
    \item \textbf{Shift:} un terminale della stringa in input, è spostato nella pila.
    \item \textbf{Reduce:} una stringa $\alpha$ in cima alla pila è sostituita con un $A \in N$ secondo $\Grammar{A}{\alpha}$.
    \item \textbf{Error:} comunica la presenza di un errore sintattico.
    \item \textbf{Accept:} la pila è vuota (si ha solo il simbolo $\$$) e la lettura è terminata,
          procede ad accettare la stringa.
\end{itemize}

Il parser shif-reduce descritto precedentemente, potrebbe non essere sufficiente a riconoscere alcune grammatiche.
Per tale ragione, in generale, si preferisce utilizzare i parser LR(k), con ``k'' numero di caratteri successivi di cui tenere traccia ad ogni passo.
Saranno trattati nel dettaglio i parser LR(0). ALtri parser SR (\textbf{eg:} SLR e/o LALR(1)) saranno solamente accennati.

\begin{Remark*}
    Se una grammatica è ambigua, non è LR(k) per nessun k.
\end{Remark*}
\end{document}