\documentclass{subfiles}
\begin{document}
Come mostrato in \emph{Figura \ref{fig:1}}, l'analisi lessicale è la prima fase della compilazione.
I suoi compiti sono sintetizzati a seguire.
\begin{enumerate}
    \item Il sorgente è scansionato e da questi si compongono i \emph{lessemi}: sequenze di caratteri con un determinato significato.
          \begin{Example*}
              un lessema per la gestione dei dati sarà del tipo: \lstinline{t_dataType}.
          \end{Example*}

    \item Per ciascuno dei lessemi, un analizzatore sintattico genera dei token della forma \lstinline{(token_name, address)},
          successivamente gestiti dall'analisi sintattica.

          Qui \lstinline{token_name} identifica un lessema, mentre \lstinline{address} è un puntatore alla cosiddetta \emph{symbol table}.
          Quest'ultima, in breve, contiene le diverse proprietà di un istanza di un lessema.
\end{enumerate}

Per quel che riguarda lo scanner questi ha essenzialmente due compiti:
\begin{itemize}
    \item costruire la symbol table;
    \item semplificare il sorgente.
\end{itemize}
Prima che ciò possa essere fatto però, è necessario, a meno che non sia stata eseguita una fase di precompilazione, che:
\begin{itemize}
    \item vengano rimossi i commenti: come ovvio sono utili al solo programmatore, dunque, per alleggeri l'eseguibile, si procede alla loro rimozione;
    \item si effettui una case conversion: se il linguaggio non distingue tra maiuscole e minuscole, allora si converte il sorgente interamente in minuscolo;
    \item si rimuovano gli spazi: per motivi analoghi ai commenti, si elimina gli spazi superflui;
    \item si deve tenere traccia del numero di linea: ciò è utile per la segnalazione di eventuali errori.
\end{itemize}

\begin{MarginNote}
    Sebbene in \emph{Figura \ref{fig:1}} sia mostrata come fase precedente l'analisi sintattica, più correttamente,
    l'analisi lessicale è da intendere come una sua sub-routine.
\end{MarginNote}
L'implementazione della symbol table, sebbene realizzabile diversamente, in generale è realizzata tramite hash-table.

\subsection{Gestione degli input}
\subfile{../Livello II/Sottosezione 2.1 - Gestione degli input.tex}
\clearpage
\end{document}