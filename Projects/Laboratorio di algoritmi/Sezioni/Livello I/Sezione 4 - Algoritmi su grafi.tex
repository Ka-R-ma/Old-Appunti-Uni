\documentclass{subfiles}
\begin{document}
\begin{Definition*}
    siano \(V, E\), rispettivamente, insieme di nodi e di archi. Dicasi la coppia \((V, E)\) grafo.
\end{Definition*}

Nel seguito della discussione sarà utilizzata una terminologia rigorosa, della quale a seguito si riporta una breve sintesi.
\begin{itemize}
    \item \emph{Cammino:} insieme di due o più vertici connessi da archi.
    \item \emph{Grado (degree) di un vertice:} numero di archi connessi al nodo.
    \item \emph{Ciclo:} un cammino che inizia e finisce in uno stesso nodo.
    \item \emph{Connessione di vertici:} con ciò ci si riferisce all'esistenza di un cammino tra due nodi.
\end{itemize}

In generale, i grafi si distinguono in \emph{diretti \emph{e} non diretti}.

\subsection{Grafi non orientati}
\subfile{../Livello II/Sottosezione 4.1 - Grafi non orientati.tex}

\subsection{Esplorazione di un grafo}
\subfile{../Livello II/Sottosezione 4.2 - Esplorazione di un grafo.tex}
\clearpage

\subsection{Grafi orientati}
\subfile{../Livello II/Sottosezione 4.3 - Grafi non orientati.tex}

\subsection{Cicli euclidei}
\subfile{../Livello II/Sottosezione 4.4 - Cicli euclidei.tex}

\subsection{Cicli hamiltoniani}
\subfile{../Livello II/Sottosezione 4.5 - Cicli hamiltoniani.tex}
\end{document}