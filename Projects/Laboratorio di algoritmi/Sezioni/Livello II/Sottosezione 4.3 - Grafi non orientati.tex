\documentclass{subfiles}
\begin{document}
Dato \(G\) un grafo, questi si dice essere orientato se, per ogni coppia di nodi \(u, v\) tali per cui esista \((u, v) \in E\),
non è detto che esista \((v, u) \in E\). La rappresentazione di questi grafi è analoga a quella dei grafi non orientati.

\subfile{../Livello III/Sottosottosezione 4.3.1 - Ordinamento topologico di un digrafo.tex}
\end{document}