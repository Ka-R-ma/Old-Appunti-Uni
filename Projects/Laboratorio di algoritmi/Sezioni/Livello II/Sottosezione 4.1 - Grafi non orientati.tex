\documentclass{subfiles}
\begin{document}
Sia dato \(G = (V, E)\) un grafo. Si dice che \(G\) è non orientato se,
per ogni coppia di nodi \(u, v\) tali per cui esista \((u, v) \in E\), esiste anche \((v, u) \in E\).

Per quanto concerne la loro rappresentazione, in generale si opta per una delle seguenti soluzioni.
\begin{itemize}
    \item \emph{\textbf{matrice di adiacenza:}} può essere intesa come una matrice booleana,
          in cui l'elemento di posizione \(a_{i,j} = 1 \iff \exists (u, v) \in E\).
          Tale rappresentazione è utile quando si ha un'elevato numero di archi, viceversa, assunto \(n\) il numero dei nodi nel grafo,
          il costo spaziale di \(\OrderOf{n^{2}}\) non sarebbe giustificato.

          Per quanto riguarda le operazioni, con tale rappresentazione l'inserimento di un nodo (\lstinline{insert}),
          e di verifica dell'esistenza di connessione con un'altro nodo (\lstinline{adjacent}) risultano efficienti, essendo \(\OrderOf{1}\).

    \item \emph{\textbf{lista di adiacenza:}} questa è da intendere come una lista contenente i nodi,
          in cui per ogni elemento si memorizza un vettore contenente i nodi a cui esso è connesso.
          Questa richiede spazio \(\OrderOf{n + m}\), posti \(n\) il numero di nodi e \(m\) il numero di archi.
          Le operazioni \lstinline{insert} e \lstinline{adjacent} richiedono, rispettivamente \(\OrderOf{1} \text{e} \OrderOf{deg(v)}\).
\end{itemize}
\end{document}