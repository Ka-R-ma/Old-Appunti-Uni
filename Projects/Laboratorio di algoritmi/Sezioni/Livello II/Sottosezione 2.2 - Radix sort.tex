\documentclass{subfiles}
\begin{document}
Sebbene nato per l'ordinamento di interi, il radix sort, per il modo in cui opera,
può essere inteso come un'algoritmo per l'ordinamento di stringhe.

Si distinguono
\begin{itemize}
    \item \emph{MSD radix sort:} per cui l'ordinamento è effettuato a partire dalla cifra più significativa;
    \item \emph{LSD radix sort:} con cui si ordina a partire dalla cifra meno significativa.
\end{itemize}

\begin{Remark*}
    Tutte le stringhe/numeri hanno lo stesso numero di cifre/caratteri.
\end{Remark*}

\subsubsection{LSD radix sort}
\subfile{../Livello III/Sottosottosezione 2.2.1 - LSD radix sort.tex}

\subsubsection{MSD radix sort}
\subfile{../Livello III/Sottosottosezione 2.2.2 - MSD radix sort.tex}
\end{document}