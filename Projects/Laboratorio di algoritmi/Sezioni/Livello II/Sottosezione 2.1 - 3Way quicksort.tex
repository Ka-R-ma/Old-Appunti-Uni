\documentclass{subfiles}
\begin{document}
Prima di descrivere il 3-way-quicksort, si procede a ricordare il quicksort nella sua versione base.
Supposto \(S\) un insieme di dati da ordinare, dati sui quali esiste una relazione d'ordine,
si procede scegliendo ricorsivamente uno degli elementi (il pivot) e sulla base di questi si suddivide l'insieme in due sottoinsiemi;
il primo contenente quegli elementi di \(S\) tali che questi risultino minori o al più uguali al pivot, il secondo contenente quegli elementi che risultano invece maggiori.
Si procede applicando la procedura per ciascuno dei sottoinsiemi, terminando quando i sottoinsiemi contengono un solo elemento.

Essendo una sua evoluzione, 3-way-quicksort procede similarmente al classico quicksort con una sola differenza: anziché formare due sole partizioni,
se ne costruiscono tre, identificando pertanto i sottoinsiemi di elementi minori, uguali e maggiori al pivot scelto.
Da un punto di vista implementativo, lo pseudo-codice è il seguente.
\subfile{../../Figure/Tikz Figure/Figure 5 - Pseudo codice 3-Way quicksort.tex}
\begin{Remark*}
    Qui con \(\Abs{R}\) si indica il numero di stringhe rimaste.

\end{Remark*}
\end{document}