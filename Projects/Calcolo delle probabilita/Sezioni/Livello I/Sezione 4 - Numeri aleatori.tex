\documentclass{subfiles}
\begin{document}
Dato $\Omega$ uno spazio di probabilità, un numero aleatorio può essere inteso come una funzione a variabili reali definita su $\Omega$ stesso.
Sia dunque $X$ un numero aleatorio, e sia $x \in \Real$ un valore assumibile da $X$.
Allora posto considerato l'evento $(X = x)$, a questi si associa la probabilita $\Prob{X = x}$.

In genere, si distinguono due ``classi'' di numeri aleatori:
\begin{itemize}
    \item \textbf{discreti:} se il numero di valori assumibili dal numero aleatorio è finito o al più numerabile;
    \item \textbf{continuo:} se non discreto.
\end{itemize}
\begin{Remark*}
    banalmente, si deve verificare che
    $$
        \Sum{\Prob{X = x_{i}}}{i = 1}[n] = \Sum{p_{i}}{i = 1}[n] = 1
    $$
    affinche l'assegnazione di probabilità risulti coerente.
\end{Remark*}

\subsection{Numeri aleatori semplici}
\subfile{../Livello II/Sottosezione 4.1 - Numeri aleatori semplici.tex}
\clearpage

\subsection{Numeri aleatori discreti}
\subfile{../Livello II/Sottosezione 4.2 - Numeri aleatori discreti.tex}
\end{document}