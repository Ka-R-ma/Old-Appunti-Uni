\documentclass{subfiles}
\begin{document}
Siano $X, Y$ due numeri aleatori. La covarianza misura il grado di variabilità reciproca tra i due.
Ossia quanto $X$ varia in funzione di $Y$ e viceversa.
Segue dunque dalla definizione di varianza
$$\begin{aligned}
        \Var{X + Y} & = \Expected*{(X + y) - (\mu_{X} + \mu_{Y})}                      \\
                    & = \Var{X} + \Var{Y} + 2\Expected*{(X - \mu_{X}) + (Y - \mu_{Y})} \\
                    & = \Var{X} + \Var{Y} + 2 \Cov{X}{Y}
    \end{aligned}$$
\begin{Remark*}
    Dalla definizione di valore atteso, segue
    $$\begin{aligned}
            \Cov{X}{Y} & = 2\Expected*{(X - \mu_{X}) + (Y - \mu_{Y})} \\
                       & =\Expected{XY} - \Expected{X}\Expected{Y}
        \end{aligned}$$
\end{Remark*}
\begin{MarginNote}
    Se $\Cov{X}{Y} = 0$ allora si dirà che $X, Y$ sono incorrelati.
\end{MarginNote}

\subsubsection{Coefficiente di correlazione}
\subfile{../Livello III/Sottosottosezione 6.2.1 - Coefficiente di correlazione.tex}
\end{document}