\documentclass{subfiles}
\begin{document}
Sia $X$ un numero aleatorio continuo, e sia la sua funzione di ripartizione definita come
$$
    f(x) = N(x) = \Frac{1}{\sqrt{2\sigma^{2}\pi}}e^{- \Frac{(x - \mu)^{2}}{2 \sigma^{2}}}   , x \in \Real
$$
Si dimostra che $\Expected{X} = \mu \text{e} \Var{X} = \sigma$.
Circa la funzione di ripartizione, questa può solo essere tabulata\footnotemark[2].
\footnotetext[2]{Si osserva banalmente che la funzione $F(x)$ è riconducibile all'integrale di Gauss, per cui non esiste una primitiva semplice.}
\end{document}