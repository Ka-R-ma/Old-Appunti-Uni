\documentclass{subfiles}
\begin{document}
\begin{Definition*}
    dato $X$ un numero aleatorio discreto, questi si dice avere distribuzione geometrica di parametro $p \in \Range*{0}{1}$ se
    $$
        \Prob{X = n} = (1 - p)^{n - 1} p, \forall n \in \Natural
    $$
\end{Definition*}
Più in generale, siano \(\List{E}{1}{n}\) eventi equiprobabili e indipendenti. Sia
$$
    X = \text{``Numero di prove fino al primo successo''}
$$
ossia $X = \Min{n}[\Abs{E_{n}} = 1]$. Segue che se $X$ è discreto si ha
$$
    (X = n) = \begin{cases}
        E_{1}, n = 1 \\
        \List{{E^{C}}}{1}{n - 1}E_{n}, \forall n \ge 2
    \end{cases}
$$
da cui
$$
    \Prob{X = n} =(1 - p)^{n - 1}p, \forall n \in \Natural
$$
Considerando valore atteso e varianza si ha che, poiché $X \in \Natural$ e nello specifico $X = \Abs{X > 0} + \Abs{X > 1} + \cdots$,
segue, posto $(1 - p) = q$, che
$$
    \Expected{X} = \Sum{\Prob{X > n}}{n = 0}[\infty]  = 1 + q + q^{2} + \cdots = \Sum{q_{i}}{i = 0}[\infty] = \Frac{1}{1 - q}
$$
Per calcolare la varianza, sia considerato $X^{2}$, poiché
$$
    X^{2} = \Abs{X > 0} + 3 \Abs{X > 1} + \cdots
$$
segue $\Expected{X^{2}} = 1 + 3(1 - p) + \cdots = \cdots = \tfrac{2 - p}{p^{2}}$, da cui
$$
    \Var{X} = \Expected*{(X - \mu)^{2}} = \Frac{1 - p}{p^{2}}
$$
\end{document}