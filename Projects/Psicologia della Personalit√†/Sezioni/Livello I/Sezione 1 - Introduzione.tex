\documentclass{subfiles}
\begin{document}
Il concetto di \emph{personalità} è un qualcosa complesso da rappresentare.
Nell'accezione cumune, il termine personalità è utilizzato in due contesti
\begin{enumerate}
    \item per dare un'idea di \textbf{coerenza} e \textbf{continuità} che caratterizza una persona:
          è questo il caso di espressioni come ``è sempre stato cosi'' oppure ``è uguale a suo nonno'';
    \item per trasmettere un'idea di causalità legato alle azioni dell'individuo.
\end{enumerate}

\noindent In tal senso, la personalità può essere utilizzata per fare previsioni riguardo al comportamento delle persone.
Se si presuppone che le persone si comportino in maniera coerente nel corso del tempo e nei diversi contesti in cui si trovano ad agire, in virtù della personalità,
allora è possibile prevedere e anticipare come gli individui si comporteranno in futuro in una data situazione.

Da un punto di vista strettamente psicologico, il termine personalità è utilizzato per definire ciò che è \textbf{rappresentativo e distintivo} dell'individuo.
\begin{framed}
    Secondo Allport:
    \begin{quote}
        ``la personalità è un'organizzazione dinamica, entro l'individuo, di sistemi psicofisici che determinano i pattern di comportamento,
        di pensiero e di emozioni tipici di ciascun individuo.''
    \end{quote}
\end{framed}

\subsection{Cosa strudia la psicologia della personalità?}
\subfile{../Livello II/Sottosezione 1.1 - Cosa studia la psicologia della personalita?.tex}

\subsection{Dalle teorie ingenue alle teorie scientifiche}
\subfile{../Livello II/Sottosezione 1.2 - Dalle teori ingenue alle teorie scientifiche.tex}

\subsection{Fattori determinanti della personalità}
\subfile{../Livello II/Sottosezione 1.3 - Fattori determinanti della personalita.tex}

\subsection{Importanza del passato, del presente e del futuro}
\subfile{../Livello II/Sottosezione 1.4 - Importanza del passato, del presente e del futuro.tex}

\subsection{I ``Quattro Quadranti'' della personalità}
\subfile{../Livello II/Sottosezione 1.5 - I quadranti della personalita.tex}

\clearpage

\end{document}