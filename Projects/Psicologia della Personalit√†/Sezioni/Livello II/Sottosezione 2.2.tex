\documentclass{subfiles}
\begin{document}

\subsubsection*{Misure della personalità}

    E' possibile distinguere: 
    \begin{itemize}
        \item \textbf{Misure fisse o nomotetiche}: a tutti i soggetti vengono somministrate le stesse 
        misure e vengono calcolati a tutti i punteggi nello stesso modo. I \textbf{limiti} sono dovuti 
        a item irrilevanti per il soggetto e dal fatto che gli item non colgono l'unicità del soggetto. 
        \item \textbf{Misure flessibili e ideografiche}: vengono somministrate misure diverse ai 
        soggetti come questionari con domande fisse, di cui i soggetti scelgono poi quelle più 
        pertinenti, oppure anche questionari con domande aperte.
    \end{itemize}

    \begin{mdframed}
    \textbf{Assessment di personalità}: La valutazione della personalità intesa come una procedura 
    standard volta a raccogliere informazioni su un individuo o su un gruppo di individui di una 
    data popolazione.
    \end{mdframed}

    Sulla base del modello teorico di riferimento si scelgono i dati da raccogliere.
    E' possibile identificare 4 tipi di valutazioni differenti: 
    \begin{enumerate}
        \item \textbf{Comportamento nella media}: si registrano le tendenze comportamentali medio o 
        tipiche delle persone; in media quanto sono calmo?
        \item \textbf{Variabilità del comportamento}: si raccolgono le variazioni del comportamento 
        di uno stesso individuo; reazione calma/ansiosa nelle varie situazioni.
        \item \textbf{Pensieri coscienti}: flusso cosciente di pensieri, azioni e sentimenti. 
        \item \textbf{Eventi mentali inconsci}
    \end{enumerate}

    A prescindere dalla tipologia di dati scelta, si deve necessariamente fare uso di procedure e 
    misurazioni ce siano attendibili e valide. 

    \subsubsection*{Attendibilità delle misurazioni}
        Le osservazioni condotte, o i testi somministrati consentono di ottenere, nel tempo, le stesse 
        misure?

        Un'osservazione con alta attendibilità ha un alto grado di coerenza e ripetibilità. 
        Invece una bassa attendibilità significa che ciò che è stato misurato è poco coerente, che 
        la misurazione non riflette ciò che effettivamente voleva essere misurato, questo include 
        un'alta quota di causalità chiamata \textbf{errore}.

        Tutte le procedure di misurazione sono fonti di errore, che può essere ridotto ma non 
        eliminato del tutto.
        Il modo in cui vengono formulati gli item può essere una fonte di errore, oppure un 
        osservatore può essere una fonte di errore causa delle variazioni nel modo in cui presta 
        attenzione a ciò che osserva. 

        Una \textbf{soluzione} può essere ripetere la misura più di una volta, cambiando punti 
        di vista e modificando leggermente la tecnica di misurazione. 

    \subsubsection*{Coerenza interna}
        Per controllare la quota di errore va misurato il grado di \textbf{coesione degli item} che 
        compongono la scala.
        Per misurare una certa dimensione si dovrebbero usare più item con significato simile ma 
        diversa formulazione, facendo in modo di pareggiare l'errore.

        Esistono diversi metodi per indagare la coerenza interna, ognuno quali analizza le correlazioni 
        tra le risposte delle persone tra i diversi item.
        Si può anche misurare la correlazione di due sottoinsiemi di item, di solite item pari e item 
        dispari, sommare i punteggi delle persone per ogni sottoinsieme e correlare tra loro i due 
        subtotali. Questo viene chiamato \textbf{split-half}.

    \subsubsection*{Attendibilità inter-rater}
        Viene fatta una osservazione finale confrontando le valutazioni di più osservatori. 
        Si procede a una suddivisione in sottoinsiemi del questionario, e quando il giudizio di un 
        valutatore correla fortemente con quello degli altri, in seguito a ripetute osservazioni, 
        allora esiste un'alta \textbf{attendibilità inter-rater}.

    \subsubsection*{Stabilità nel tempo}
        Misuro il grado in cui i \textbf{punteggi} ottenuti da un soggetto ad uno stesso test, in 
        momenti diversi, sono simili. 
        Si suppone ce la personalità si astabile e coerente nel tempo e quindi lo stesso soggetto 
        deve, a distanza di tempo, ottenere gli stessi punteggi allo stesso test.

    \subsubsection*{Validità delle misurazioni}
        La validità indica il grado in cui il test misura aggettivamente ciò che intende misurare.

        Occorre confrontare:
        \begin{itemize}
            \item \textbf{Definizione concettuale}: definizione dei concetti, che da informazioni su 
            quelle che una parola trasmette tramite il consenso esistente tra le persone.
            \item \textbf{Definizione operazionale}: descrizione di un evento fisico. 
        \end{itemize}

        Prendiamo come esempio l'Amore:
        \begin{itemize}
            \item \textbf{Definizione concettuale}: Forte affetto per una persona, dedizione 
            appassionata ed esclusiva, istintiva ed intuitiva fra le persone, volta ad assicurare 
            reciproca felicità, o la soddisfazione sessuale.
            \item \textbf{Definizione operazionale}: Segnare in una scala da 0 a 100 quanto ama una 
            persona, ecc\ldots 
        \end{itemize}

        \textsc{Definizione operativa e quella concettuale devono approssimarsi o coincidere.}

        \begin{itemize}
            \item \textbf{Validità di costrutto}: Si riferisce a quanto lo strumento di misurazione 
            rifletta il costrutto corrispondente. 
            Il \textbf{costrutto} corrisponde al concetto oggetto di studio, ogni tratto 
            caratteriale, per esempio, è un costrutto (Autostima, stili cognitivi, Intelligenza).

            \item \textbf{Validità predittiva}: Si riferisce a quanto la scala è in grado di 
            predire effettivamente alcuni criteri esterni che, a livello teorico dovrebbero 
            essere predetti dalla scala stessa. 
            Per \textbf{criterio} si intendono le dimensioni associate alla variabile oggetto 
            di studio e che da essa discendono. 

            \item \textbf{Validità convergente}: Si riferisce a quanto la scala è correlata 
            con altre misurazioni alternative dello stesso aspetto. Per \textbf{convergenza} 
            si intende che tutte le prove devono convergere sul costrutto che si sta valutando, 
            sebbene ogni singola prova di per se non rifletter esattamente tale costrutto. 
            
            Per esempio, una scala costruita per misurare la dominanza dovrebbe includere almeno 
            alcune misure di qualità come leadership (correlazione positiva) e timidezza (correlazione inversa).
            
            \item \textbf{Validità discriminante}: Si riferisce a quanto la scala effettivamente 
            NON misuri aspetti che non intende misurare. 
            E' sempre possibile attribuire l'effetto di una dimensione della personalità sul 
            comportamento a un'altra dimensione di personalità. 

            Se la ricerca mostra che la dimensione non è correlata con una variabile, allora tale 
            variabile non può essere chiamata in causa come spiegazione alternativa per qualsiasi 
            effetto sulla prima. 

            Intelligenza NON correlata a Dominanza e Autostima NON correlata ad Intelligenza.
        \end{itemize}

        \clearpage

        

   
\end{document}