\documentclass{subfiles}
\begin{document}
Nella teoria di Freud, si mette in evidenza come i primi anni di vita siano fondamentali per l'esistenza di un individuo,
dando al passato un enorme rilievo. Altre prospettive teoriche si concentrano al futuro, prendendo in considerazione obiettivi
dell'individuo, mete da raggiungere, e gli scopi a cui tendere, (posizioni teoriche analizzate da autori come Allport, Bondura, Mischel e Kelly).
Nella prospettiva comportamentale, autori come Skinner, accentuano l'importanza del presente nella determinazione della personalità.

\subsubsection{Motivazioni del comportamento umano}
\subfile{../Livello III/Sottosottosezione 1.4.1 - Motivazioni del comportamento umano.tex}

\subsubsection{Il comportamento umano: liberamente scelto o determinato}
\subfile{../Livello III/Sottosottosezione 1.4.2 - Il comportamento umano: liberamente scelto o determinato.tex}

\subsubsection{Unicità o comunanza tra gli individui}
\subfile{../Livello III/Sottosottosezione 1.4.3 - Unicita o comunanza tra gli individui.tex}

\subsubsection{Controllo del comportamentamento umano}
\subfile{../Livello III/Sottosottosezione 1.4.4 - Controllo del comportamentamento umano.tex}

\subsubsection{Positività o negatività della natura umana.}
\subfile{../Livello III/Sottosottosezione 1.4.5 - Positivita o negativita della natura umana.tex}

\subsubsection{La natura è unitaria o conflittuale?}
\subfile{../Livello III/Sottosottosezione 1.4.6 - La natura: unitaria o conflittuale?.tex}
\end{document}