\documentclass{subfiles}
\begin{document}
Come si introduceva all'inizio della sezione, genericamente quando ci si riferisce alla personalità,
si tende ad attribuire a questa il significato delle azioni dell'individuo.
In gergo, tali attribuizioni prendono il nome di \emph{teorie ingenue}.
A queste si contrappongono le \emph{teorie scientifiche}; queste risultano essere più valide poiché è possibile procedere a una verifica sperimentale delle stesse.

In generale, una volta elaborata la teoria, lo psicologo della personalità cerca di \textit{tradurre} tale teoria in \textit{"applicazioni pratiche"}
che rechino benefici ai soggetti.
\clearpage
\end{document}