\documentclass{subfiles}
\begin{document}

Freud sviluppa l'idea che la \textbf{personalità adulta} si sviluppi attraverso 
\textbf{stadi}, ciascuno associato a una \textbf{zona erogena} specifica. 
La \textbf{zona erogena} è il centro dell'energia sessuale (libido) in quella fase di sviluppo.

Il bambino vive queste fasi in modo conflittuale, e se questi conflitti non vengono 
risolti, avviene un processo chiamato \textbf{fissazione} che porta a un aumento dell'energia 
in quel particolare stadio e una diminuzione dell'energia disponibile per gli stadi successivi.

La fissazione può avvenire per due motivi: il soggetto rimane ancorato allo stadio attuale 
senza voler passare al successivo o il soggetto è frustrato nel soddisfacimento dei bisogni 
dello stadio attuale e non passa al successivo finché non li appaga.\\

\subsubsection{Fase Orale}
La fase orale si colloca dalla nascita fino ai 18 mesi circa. La \textbf{bocca} è la zona 
erogena principale che media tutte le relazioni del bambino con il mondo. Il bambino ottiene 
il nutrimento attraverso la bocca, trae piacere da questa azione.

Il bambino è estremamente \textbf{dipendente} dall'ambiente, il conflitto principale consiste 
nel superare questa dipendenza durante lo \textbf{svezzamento}.

La fase si suddivide in due sottostadi: 
\begin{enumerate}
    \item \textbf{Fase Orale Incorporativa (Dalla nascita ai 6 mesi)}: Il bambino in questa fase 
    è estremamente dipendente dall'ambiente. 
    Se l'ambiente è benevolo, si sviluppano dimensioni come l'ottimismo e la fiducia, d'altro 
    canto se l'ambiente meno benevolo allora si sviluppano pessimismo e sfiducia. 
    Invece l'eccessiva presenza dell'ambiente può causare \textbf{dipendenza eccessiva}.

    \item \textbf{Fase Orale Sadica (Dai 6 mesi ai 18 mesi)}: Questa fase coincide con la 
    dentizione, dove il piacere del bambino deriva dal \textbf{mordere e masticare}. 
    Il superamento adeguato di questa fase influenza il futuro comportamento verbale del 
    bambino, rendendolo meno \textbf{sadico o sarcastico}.
\end{enumerate}

Individui che mostrano tratti di personalità `orale' possono trovare conforto nel cibo o nelle 
azioni orali per affrontare lo stress. Possono manifestare comportamenti aggressivi o dipendenza 
emotiva dagli altri. Questi individui possono anche sviluppare problematiche come alcolismo o 
obesità, perché cercano soddisfazione attraverso il cibo.
Nonostante siano inclini agli scambi sociali, possono soffrire di isolamento e mostrare una 
tendenza a rivelare apertamente se stessi.\\
    
\subsubsection{Fase Anale}
Durante lo stadio anale dello sviluppo, che va da circa i 18 mesi ai 3 anni, la 
\textbf{zona erogena} principale è l'\textbf{ano}, e il piacere è associato al processo della 
defecazione. 
L'evento cruciale in questa fase è il \textbf{controllo degli sfinteri}, dove si instaura il 
conflitto tra le spinte interne e i vincoli esterni imposti dall'educazione.

La \textbf{cruciale educazione al controllo degli sfinteri} può essere affrontata attraverso 
due approcci principali: 

\begin{enumerate}

    \item \textbf{Rinforzo del Controllo}: In questo approccio, si sollecita il bambino a 
    controllare le sue funzioni corporee nei tempi e nei luoghi appropriati, offrendo un 
    \textbf{rinforzo positivo} quando ci riesce. 
    Questo promuove l'idea di produzione appropriata nei tempi e nei modi adeguati, contribuendo 
    allo sviluppo di produttività e creatività nell'età adulta.

    \item \textbf{Punizione e Derisione}: Al contrario, un approccio punitivo e derisorio per 
    ogni fallimento nel controllo degli sfinteri può portare a conseguenze negative. 
    Se il bambino adotta un modello attivo di ribellione, potrebbe sviluppare 
    \textbf{tratti ribelli, disordinati e ostili} da adulto. Oppure, se trattiene le 
    funzioni, potrebbe manifestare \textbf{rigidità e ossessività}.

\end{enumerate}

La fase anale è caratterizzata dalla \textbf{triade anale}, che comprende a 
\textbf{tirchieria} (desiderio di trattenere), l'\textbf{ostinazione} 
(lotta di volontà contro il controllo degli sfinteri) e la \textbf{rigidità} 
(reazione contro disordine della defecazione), tratti che possono influenzare la personalità 
dell'individuo nell'età adulta.\\

\subsubsection{Fase fallica}
Durante la fase fallica dello sviluppo, che si colloca circa tra i 3 e i 5 anni di età, la 
\textbf{zona erogena} principale diventano le \textbf{parti genitali}, e il piacere è legato 
all'\textbf{autostimulazione}. In questa fase, la libido si posta verso il genitore dell'altro 
sesso, dando origine al \textbf{Complesso di Edipo (per i maschi) / Elettra (per le femmine)}.

\begin{itemize}

    \item Nel \textbf{Complesso di Edipo}, il bambino sviluppa un \textbf{desiderio di possesso} 
    e al contempo prova \textbf{gelosia e odio} nei confronti del genitore del sesso opposto, 
    considerato un rivale. Il timore del padre come rappresentante dell'autorità maschile porta 
    alla \textbf{paura della castrazione}. Per superare questo complesso, il bambino si 
    identifica con il padre, assimilandone le caratteristiche e i valori, permettendo così 
    di creare il Super-Io.

   \item Nel \textbf{Complesso di Elettra}, la bambina, dopo aver abbandonato l'amore per 
   la madre, sviluppa un \textbf{desiderio per il padre}, associato all'\textbf{invidia del pene}. 
   La consapevolezza dell'assenza del pene porta alla \textbf{rabbia verso la madre}, considerata 
   colpevole della castrazione, e al desiderio del padre. Anche la bambina si identifica con la 
   madre per superare questo complesso.

\end{itemize}

Il superamento del Complesso di Edipo/Elettra è cruciale per lo sviluppo di una personalità matura. Il modo in cui il bambino affronta i sentimenti di amore, odio, gelosia, colpa e paura determina lo sviluppo della personalità e della sessualità nell'età adulta. 

La \textbf{fissazione} durante la fase fallica produce una personalità che riflette il 
complesso edipico. I soggetti possono diventare grandi seduttori o individui di successo o in 
alternativa potrebbero avere poche o nulle relazioni sentimentali a causa di un senso di 
colpa persistente. \\

\subsubsection{Periodo di latenza}
Al termine della fase fallica, il bambino entra in un periodo di relativa calma, chiamato 
\textbf{periodo di latenza}. Questo periodo va dai 6 anni circa fino all'inizio 
dell'adolescenza, è un periodo in cui le pulsioni sessuali e aggressive sono meno attive, 
a causa dello \textbf{sviluppo dell'Io e del Super-Io}.
I bambini ora rivolgono l'attenzione verso altre attività, spesso intellettuali o sociali.\\

\subsubsection{Fase Genitale}
Durante l'\textbf{adolescenza} (12-18 anni) e nell'\textbf{età adulta}, che rappresenta la 
\textbf{fase genitale} dell'evoluzione psicosessuale secondo la teoria di Freud, le 
\textbf{pulsioni sessuali} ritrovano nuova linfa. Questo periodo segna l'esito della 
risoluzione dei conflitti delle fasi precedenti dello sviluppo psicosessuale.

Nell'età adulta, l'individuo tende a perseguire il \textbf{piacere} non solo per sé stesso 
ma anche per l'altro, attraverso manifestazioni come l'amore e l'affetto. 
Questo indica un'evoluzione e una maturazione della sessualità, dove l'interesse non è più 
concentrato esclusivamente sul soddisfacimento dei propri desideri, ma include anche la 
ricerca di relazioni significative e soddisfacenti con gli altri.\\


\end{document}