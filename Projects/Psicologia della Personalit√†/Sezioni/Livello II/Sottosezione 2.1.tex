\documentclass{subfiles}
\begin{document}

Esistono quattro differenti tipologie di dati ce vengono impiegati nell'indagine scientifica sulla 
personalità e che vengono identificate dall'acronimo \textbf{LOTS}.

\subsubsection{LIFE-RECORD-DATA (dati L)}
Sono le informazioni che possono essere direttamente ricavate dalla storia della vita di una 
persona (stato civile, occupazione, residenza, infrazioni, etc.).
Alcuni di tali dati non sono sempre rintracciabili e quindi occorrono altre fonti.

\subsubsection{OBSERVER-DATA (dati O)}
Sono le informazioni fornite da individui che entrano in contatto con la persona in interesse.
Esistono diversi tipi di osservazioni che si differenziano in base alle caratteristiche 
dell'osservatore e dal setting in cui viene effettuata l'osservazione.

Un'osservazione può essere condotta da: 
\begin{itemize}
    \item \textbf{osservatori clinici}: personale specializzato che conduce l'osservazione e che
     non ha alcun legame col soggetto. 
    \item \textbf{osservatori non-clinici}: osservatore con cui la persona oggetto di 
    osservazione ha una relazione già esistente, per esempio, un amico, insegnante, familiare, 
    che vengono \textit{`addestrati'} all'osservazione. 
\end{itemize}

Il setting di osservazione può essere:
\begin{itemize}
    \item \textbf{setting artificiale}: L'osservazione avviene in un luogo appositamente 
    predisposto, come un laboratorio. 
    \item  \textbf{setting naturalistico}: L'osservazione avviene in contesti di vita quotidiana 
    della persona oggetto di osservazione.
\end{itemize}

L'osservazione può essere condotta da più osservatori contemporaneamente sullo stesso soggetto.
La presenza di osservatori multipli stabilisce un \textbf{grado di coerenza e accordo inter-giudici}.
Un'indice più alto implica un'osservazione valida e attendibile.
Un'altra modalità d'osservazione è l'\textbf{introspezione}, ma presenta il limite di falsi ricordi. 

\subsubsection{TEST-DATA (dati T)}
Sono dati che si ottengono attraverso procedure di tipo sperimentale, in cui i partecipanti 
vengono messi davanti a una medesima situazione. 
Le differenze nelle risposte date dai partecipanti vengono spiegate alla luce di differenze 
nelle caratteristiche di personalità.
Nel caso in cui si voglia indagare l'influenza dell'impulsività sulla tendenza a effettuare 
scelte economiche vantaggiose, come per esempio l'Iowa Gambling Task, identificano due gruppi 
e confrontano i loro comportamenti.

\subsubsection{SELF-REPORT-DATA (dati S)}
Dati che derivano dall'auto descrizione o autovalutazione, dove le persone stesse assumono il 
ruolo di osservatori.
Vengono proposti questionari strutturati, dove viene richiesto di rispondere a un insieme 
specifico di domande in un formato vero-falso, o attraverso una scala \textbf{Likert} con 5-7 punti che vanno da "fortemente d'accordo" a "fortemente in disaccordo".
Alcuni questionari si focalizzano su un singolo aspetto della personalità, invece altri, 
chiamati \textbf{inventari}, valutano differenti dimensioni della personalità.

Limiti dell'uso dei dati self-report risiedono negli \textbf{stili di risposta} che creano alcune distorsioni nella misurazione della personalità.
\begin{itemize}
    \item \textbf{Acquiescenza}: Tendenza a rispondere sempre SI o sempre `abbastanza', 
    si risolve attraverso l'uso di item (domande) negativi.
    \item \textbf{Desiderabilità sociale}: Tendenza a presentare un'immagine positiva di sé, 
    si risolve attraverso la formulazione degli Item o scale `LIE'. 
\end{itemize}

\subsubsection{Coerenza dati LOTS}

Maggiori \textbf{discrepanze} tra dati S e dati T: Le auto-descrizioni \textbf{non concordano} 
con le misure sperimentali o con le misure implicite degli atteggiamenti.

Maggiori \textbf{concordanze} tra dati S e dati O: Le auto-descrizioni e le valutazioni da 
parte degli altri tendenzialmente \textbf{concordano} anche se i giudizi di valore su di sé 
possono essere più benevoli di quelli dati dagli altri. 

\vspace{2cm}

\end{document}