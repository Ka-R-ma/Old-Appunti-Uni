\documentclass{subfiles}
\begin{document}


\paragraph*{Contesto Filosofico e Scientifico}
Filosofi del passato, come Pascal, Brentano e Leibniz, avevano già dimostrato un interesse per 
l'\textbf{autocoscienza} e il rapporto complesso tra ragione e dimensioni psichiche non 
razionali, che non erano accessibili direttamente alla coscienza.

Verso la fine dell'Ottocento, mentre le \textbf{scienze naturali}, come la \textbf{fisica} e la 
\textbf{biologia}, erano considerate il vertice del sapere scientifico, alcuni studiosi delle 
\textbf{scienze sociali} cominciarono a dubitare che la \textbf{fisiologia} e la 
\textbf{neurofisiologia} potessero spiegare adeguatamente la complessità della \textbf{mente umana}.\\ 

\paragraph*{Influenze su Freud}
Sigmund Freud, laureatosi in \textbf{Medicina} nel 1881, crebbe in un'epoca in cui dominavano 
due approcci culturali in Europa: il pensiero evoluzionistico \textbf{darwiniano} e 
l'approccio \textbf{razionalista, meccanicista e naturalistico}, che assegnavano un peso 
maggiore alla ricerca \textbf{empirica} e all'\textbf{osservazione sistematica} rispetto alle 
concezioni più umanistiche. 

La \textbf{fisica} era considerata la disciplina chiave per comprendere qualsiasi aspetto 
del mondo reale, come affermato da figure come \textbf{Wilhelm} e \textbf{Von Brucke}. 
Tuttavia, Freud cominciò a dubitare del predominio della fisiologia e venne influenzato 
da diverse idee filosofiche dell'epoca.\\

\paragraph*{Ideali di Freud}
Freud riconobbe l'importanza della \textbf{psicologia} rispetto alla \textbf{fisiologia}, 
affermando l'esistenza di \textbf{`idee inconsce'} o `piccole percezioni' che non raggiungono 
la coscienza e la necessità di misurare e quantificare scientificamente i fenomeni \textbf{psichici}.\\

\paragraph*{Influenza di Pierre Janet}
Uno degli influenti pensatori fu \textbf{Pierre Janet}, un pioniere nel collegare eventi 
passati alla formazione di traumi attuali. Egli introdusse concetti come il \textbf{subconscio} 
e la \textbf{dissociazione}, spiegando come la mente potesse difendersi da eventi traumatici 
allontanando effetti e ricordi associati. 
Questi potrebbero riemergere in forma di \textbf{flashback} o \textbf{sogni}.\\

\paragraph*{Controversia e Rilevanza Attuale}
Nel 1913, vi fu una controversia su chi fosse il fondatore della psicoanalisi, con Freud e Janet 
come contendenti principali. Oggi, sebbene Freud sia universalmente riconosciuto come il padre 
della psicoanalisi, le idee di Janet continuano ad essere rilevanti per la comprensione di 
disturbi come il \textbf{Disturbo Post-traumatico da Stress} e i 
\textbf{Disturbi Dissociativi della Personalità}.\\

\end{document}