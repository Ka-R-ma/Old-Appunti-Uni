\documentclass{subfiles}
\begin{document}

Gli individui sono \textbf{sistemi complessi} di energie, in cui l'energia usata nei processi 
psichici (pensare, ricordare, sognare) è generata e rilasciata secondo \textbf{processi biologici}. L'\textbf{Es} è la sede di tali processi biologici, noti come \textbf{Pulsioni}.

\paragraph*{Pulsioni}
Il modello \textbf{Tensio-Riduttivo o Idraulico} descrive le pulsioni come sempre attive, generando 
tensione fino a quando non viene espressa un'azione che consente lo scarico della tensione.

Esistono due classi di pulsioni, di vita e di morte. Le pulsioni di vita (o \textbf{Eros}) 
riguardano la sopravvivenza, la riproduzione e il piacere, mentre le pulsioni di morte 
(o \textbf{Thanatos}) rappresentano un'innata ispirazione all'annullamento di sè da parte 
dell'organismo.\\

\paragraph*{Catarsi}
Il conflitto tra pulsione di vita e pulsione di morte genera l'\textbf{aggressività}, che si 
manifesta come azioni aggressive e distruttive verso gli altri. 
Se la tensione di una pulsione non viene scaricata, la pressione rimane e inoltre cresce, fino a 
non poter essere più trattenuta.

Il termine \textbf{catarsi} è usato per riferirsi al rilascio della tensione emotiva in una 
determinata esperienza.\\

\subsubsection{Angoscia e meccanismi di difesa}
Freud non considera l'angoscia come una pulsione di per sé, ma come un segnale di allarme, di 
allerta per l'Io che qualcosa di negativo sta per accadere.
Freud distingue tre tipi di angoscia: 

\begin{itemize}
    \item \textbf{Angoscia reale}: Si manifesta come reazione a un pericolo o danno atteso 
    dall'esterno (essere morsi da un cane, fare un incidente).
    \item \textbf{Angoscia nevrotica}: E' la reazione al timore delle punizioni ce potrebbero 
    seguire all'espressione delle richieste dell'Es. 
    \item \textbf{Angoscia morale}: E' il tipo di angoscia che insorge quando la soddisfazione 
    di una pulsione viene proibita dalla propria coscienza morale, visto di solito come senso 
    di colpa o vergogna.
\end{itemize}

Il fronteggiamento dell'angoscia è essenziale per mantenere l'equilibrio psichico. 
L'Io svolge un ruolo chiave come mediatore tra Realtà, Es e Super-Io, evitando l'angoscia quando 
funziona correttamente. Si possono adottare diverse strategie: 

\begin{itemize}
    \item \textbf{Prevenzione dei Pericoli Esterni}: Evitare o affrontare in modo adeguato 
    situazioni pericolose limita l'angoscia reale.
    \item \textbf{Scarica dei Bisogni dell'Es}: Soddisfare i bisogni dell'Es in modo tempestivo 
    riduce l'angoscia nevrotica.
    \item \textbf{Rispetto dei Principi Morali}: Evitare azioni `contro coscienza' aiuta a 
    limitare l'angoscia morale.
\end{itemize}

Quando l'angoscia aumenta, l'Io può adottare due approcci: 

\begin{itemize}
    \item \textbf{Strategie di Azione}: Risolvere il problema con azioni concrete.
    \item \textbf{Meccanismi di Difesa}: Meccanismi psichici che mediano tra desideri, bisogni, 
    affetti e impulsi dell'individuo da un lato, e proibizioni interne e realtà esterna dall'altra. Processi inconsci che regolano l'omeostasi psichica e proteggono dall'angoscia. 
    I meccanismi di difesa operano inconsciamente e sono in azione in tutte le persone. 
    (Teorizzati da \textbf{Anna Freud} nel 1936).
\end{itemize}

\paragraph{Repressione e rimozione}
Il meccanismo di difesa basilare è la \textbf{repressione}. 
Nella repressione una certa quota di energia disponibile per l'Io viene usata per mantenere 
fuori dalla coscienza gli impulsi inaccettabili.

La repressione può essere fatta consapevolmente, come la persona che prova ad allontanare un 
contenuto dalla consapevolezza, oppure può essere un processo inconscio; in tal caso è detto 
\textbf{rimozione}.

La repressione può essere usata per impedire che diventino consapevoli gli impulsi 
dell'Es, ricordi di eventi dolorosi associati a impulsi espressi, sentimenti 
negativi (vergogna), paure e sensi di colpa. 

Si parla di \textbf{repressione parziale} quando si cerca di nascondere i parte i ricordi, ma 
possibilmente essi sono recuperabili. \\

\paragraph{Negazione}
La \textbf{negazione} è un meccanismo di difesa in cui si rifiuta di credere che un evento sia 
accaduto o esista, spesso quando si è sopraffatti da realtà minacciose. 
Alcuni esempi includono il rifiuto di accettare la morte di una persona cara o il negare di aver 
subito abusi o fallimenti. 

La \textbf{repressione} e la \textbf{negazione} sono entrambi meccanismi di difesa che 
contrastano l'angoscia e il dolore, ma nel lungo periodo possono assorbire l'energia 
pulsionale, limitando le risorse dell'Io per altre attività. Questo può portare a comportamenti 
rigidi e poco adattivi nel tempo, generando malessere all'individuo. \\

\paragraph{Proiezione}
La \textbf{proiezione} è un meccanismo di difesa in cui gli individui attribuiscono le proprie 
qualità inaccettabili o desideri repressi ad altre persone. Questo riduce l'angoscia permettendo 
loro di esprimere in modo distorto qualità negative che altrimenti non sarebbero accettate. 

La proiezione serve a due scopi principali: Consente di realizzare i desideri in una forma o 
nell'altra, rilasciando energia utile a reprimerli e permette al desiderio di emergere in modo 
che l'Io e il Super-Io non lo riconoscano come proprio, permettendo così di evitare la minaccia 
associata al desiderio represso.\\

\paragraph{Razionalizzazione e Intellettualizzazione}
La \textbf{razionalizzazione} è un meccanismo di difesa che riduce l'angoscia attraverso una 
spiegazione razionale del comportamento angosciante attivato o vissuto. 
Gli individui spesso utilizzano la razionalizzazione per giustificare le proprie azioni o 
sentimenti in modo da mantenere l'autostima e proteggere la propria psiche. 

Un esempio è quando una persona rifiutata in amore dice: `Dopo tutto, quella donna non mi meritava', 
spiegazione razionale che aiuta a proteggere l'autostima del soggetto.

La \textbf{intellettualizzazione} è un meccanismo di difesa caratterizzato dalla tendenza a 
pensare alle minacce in modo freddo, analitico e distaccato. Gli individui che utilizzano questo 
meccanismo separano i loro pensieri dalle emozioni, creando una distanza tra l'evento minaccioso 
e l'emozione che lo accompagna.

Un esempio è quando i familiari di malati oncologici sviluppano una conoscenza approfondita 
di malattie e tecniche di cura. In questo caso, invece di affrontare direttamente le emozioni 
legate alla malattia, essi cercano di comprendere razionalmente la situazione attraverso una 
conoscenza dettagliata.\\

\paragraph{Spostamento e sublimazione}
Lo \textbf{spostamento} è un meccanismo di difesa in cui un impulso minaccioso viene spostato 
su un altro oggetto o situazione percepiti come meno minacciosi. Ad esempio, la rabbia verso 
un familiare può essere spostata verso un personaggio in un videogioco, considerato un bersaglio 
meno minaccioso. 

La \textbf{sublimazione} è un meccanismo di difesa in cui un impulso inaccettabile viene 
trasformato in azioni socialmente accettabili o produttive. Un esempio è quando il desiderio 
di uccidere e aggredire gli altri viene sublimato nella pratica della caccia. La sublimazione 
viene vista come un'espressione di maturità del soggetto perché non agisce a posteriori 
sull'angoscia, ma ne previene l'impulso in anticipo.\\

\subsubsection{Psicologia della vita quotidiana}
Un modo in cui tali impulsi si rivelano sono gli errori che commettiamo. 
Freud crede che tali eventi, derivino da impulsi nell'inconscio che emergono in una forma 
distorta, come errori, vuoti di memoria, lapsus, slittamenti del linguaggio, incidenti, denominati 
complessivamente \textbf{atti mancati}.

Freud ritiene che l'inconscio riveli se stesso ance attraverso i \textbf{sogni}. 
I sogni hanno due tipi di contenuto: il \textbf{contenuto manifesto} sono le immagini sensoriali 
e il \textbf{contenuto latente}, che comprende i pensieri, sentimenti e desideri inconsci sottesi 
al contenuto manifesto.

Il contenuto latente ha tre fonti:
\begin{itemize}
    \item La prima è la \textbf{stimolazione sensoriale} che avviene durante il sonno.
    \item La seconda fonte sono i \textbf{pensieri}, le \textbf{idee} collegati alla veglia.
    \item La terza fonte sono gli \textbf{impulsi inconsci}, la cui espressione è bloccata mentre 
    si è svegli e che sono spesso collegati ai conflitti di base.
\end{itemize}

\clearpage

\end{document}