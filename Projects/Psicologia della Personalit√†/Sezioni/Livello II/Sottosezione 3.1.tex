\documentclass{subfiles}
\begin{document}

La psicoanalisi è una prospettiva che attribuisce particolare valore ai processi di cui gli 
individui non sono consapevoli.

\textbf{Sigmund Freud (1856-1939)}, il fondatore della psicoanalisi, è nato a Freiberg in Moravia.
È stato figlio di mercanti tessili ebrei che aspiravano ad integrarsi nella cultura austriaca. 
La famiglia parlava tedesco e aveva ambizioni di scalata sociale. 
Erano molto tradizionalisti nel rispetto della religione ebraica, e Freud è stato educato di conseguenza.

Dopo aver conseguito la laurea in Medicina nel 1881, all'eta di 50, Freud ha fondato la 
psicoanalisi, un movimento innovativo e rivoluzionario per l'epoca destinato a diffondersi in 
tutto il mondo. Ha trascorso un anno in Francia da Charcot, occupandosi di malattia mentale e 
sperimentando l'ipnosi, ma insoddisfatto da tale tecnica, ha sperimentato la tecnica delle 
associazioni libere di Breuer.

Tra il 1892 e il 1895, Freud scrisse e pubblicò `Studi sull'isteria, dove ha esposto questo metodo. Nel 1897 ha iniziato l'autoanalisi, proseguendola fino alla morte. Nel 1899 ha scritto "L'interpretazione dei sogni", elaborando la concezione dell'uomo come sistema dinamico. Nel 1909 ha tenuto diverse conferenze negli USA, dove si è recato con Jung, suo discepolo.

Nel 1923 ha lottato contro un tumore al palato e ha affrontato diversi drammi familiari legati 
alla morte di una figlia e di un nipotino. 
Dopo l'emanazione delle leggi razziali nel 1938, Freud è emigrato a Londra, dove è morto 
l'anno successivo all'età di 83 anni.


\end{document}