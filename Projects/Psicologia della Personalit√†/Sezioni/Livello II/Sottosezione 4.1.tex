\documentclass{subfiles}
\begin{document}

Alfred Adler (1870-1937), psicologo austriaco, si discosta dalla teoria freudiana concentandosi 
maggiormente sulle \textbf{forze sociali} e sui processi \textbf{consapevoli}, criticando la 
dimensione pulsionale e i processi inconsci, percepiti come troppo pervasivi nella teoria freudiana. \\

\paragraph*{Formazione e Vita Personale}
Nato da genitori ebrei ungheresi e cresciuto nei sobborghi di Vienna, Adler lottò contro il 
ratichismo, cercando di compensare il suo difetto fisico con l'attività sportiva.
Laureatosi in medicina, inizia a praticare come medico generico, a si interessa presto alla psicologia.
Nel 1902, incontra Freud e partecipa alle riunioni del mercoledì, ma successivamente si allontana da 
lui per fondare la `\textbf{Società per la Libera Psicoanalisi}'.\\

\paragraph*{Contributi alla Psicologia}
Durante la Prima Guerra Mondiale, Adler si occupa di nevrosi e si avvicina al marxismo, promuovendo 
riforme sociali basate su principi democratici e sull'attenzione alle esigenze individuali.

Emigra negli USA nel 1934, dove diventa professore di Psicologia presso la Columbia University e 
il Medical College di Long Island.

La sua morte avviene nel 1937 a causa di un crisi coronarica, mentre si trovava in Scozia per 
una serie di conferenze.\\

\subsubsection{Psicologia Individuale di Alfred Adler}
La Psicologia Individuale di Alfred Adler si focalizza sul concetto di \textbf{inferiorità}, che 
comprende sia aspetti fisici che psicologici. 
Secondo Adler, tutti sperimentano un senso di inferiorità sin dall'infanzia, spesso derivante da 
difetti fisici o circostanze sociali. Questo senso di inferiorità agisce come una spinta 
motivazionale, spingendo gli individui a compiere sforzi per superare tali sentimenti e a 
perseguire la \textbf{realizzazione personale}. \\

\paragraph*{Dinamica del Senso di Inferiorità}
Il senso di inferiorità induce gli individui a sviluppare \textbf{strategie di compensazione} per 
superare i loro difetti. Ad esempio, se qualcuno si sente inferiore per una mancanza di abilità 
verbale, potrebbe sviluppare un interesse per la scrittura per compensare questa debolezza. 
Questo processo di compensazione guida lo sviluppo della \textbf{personalità}.\\

\paragraph*{Volontà di Potenza e Sentimento di Comunità}
Adler sostiene che la principale motivazione umana sia la \textbf{volontà di potenza}, il 
desiderio di diventare una persona competente, efficace e rispettata. 
Questo desiderio si collega al \textbf{sentimento di comunità}, che riflette il desiderio di 
connessione e appartenenza sociale. 

Nella personalità funzionale, l'\textbf{aspirazione alla superiorità} si manifesta come 
capacità di interagire con gli altri in modo equilibrato e sano, ovvero in modi che tengano 
conto sia dei bisogni e del benessere degli altri sia del proprio desiderio di competere e 
`farsi valere'.\\

\subsubsection{Personalità Sana vs. Nevrotica}
Nella visione di Adler, una \textbf{personalità sana} si sviluppa intorno al 
\textbf{Sè Creativo}, una variabile soggettiva che è in grado di perseguire le proprie mete 
in armonia con gli altri, riconoscendo e rispettando i bisogni degli altri.

Al contrario, una \textbf{personalità nevrotica} si forma quando l'aspirazione alla superiorità 
è frustrata, portando a comportamenti aggressivi, ostili e disfunzionali.

Il \textbf{Sè creativo} rappresenta il nucleo della personalità e guida il comportamento e 
l'interpretazione della realtà. Lo \textbf{stile di vita} si sviluppa nelle prime fasi 
dell'infanzia e riflette i modi in cui gli individui affrontano le sfide e perseguono le loro mete.\\

\paragraph*{Approccio Terapeutico Adleriano}
Nella terapia adleriana, si studia la \textbf{Costellazione Familiare}, ovvero si esamina la 
storia personale del paziente e si cerca di comprendere come il suo stile di vita e le sue 
relazioni familiari abbiano influenzato il suo sviluppo. L'obiettivo è aiutare il paziente a 
identificare le sue mete e a sviluppare strategie efficaci per raggiungerle, promuovendo il 
\textbf{benessere individuale e sociale}.\\


\end{document}