\documentclass{subfiles}
\begin{document}
Siano \(E_{1}, E_{2}, \dots, E_{n}\) successione di eventi stocasticamente indipendenti. Posto \(T_{k}\) il numero di prove fino al k-esimo successo, segue
\[\begin{aligned}
        P(T_{k} = i) & = \binom{i - 1}{k - 1}p^{k - 1}q^{i - k}p \\
                     & = \binom{i - 1}{k - 1}p^{k}q^{i - k}      \\
    \end{aligned}\]
poiché si ha che nelle \(i - 1\) prove si hanno \(k - 1\) successi.

\begin{Example*}
    Viene lanciata una moneta fino ad ottenere testa per la seconda volta: si stabilisca
    \begin{multicols}{2}
        \begin{enumerate}
            \item \(P(T_{2} = 3)\)
            \item \(P(T_{2} = 4)\)
        \end{enumerate}
    \end{multicols}

    \begin{enumerate}
        \item Si consideri l'evento associato a \((T_{2} = 3)\), si ha
              \[
                  (T_{2} = 3) = TCT \lor CTT
              \]
              questo perché al terzo lancio necessariamente si ottiene testa, dunque
              \[
                  P(T_{2} = 3) = P(TCT) + P(CTT)
              \]

        \item Analogamente a prima, si consideri l'evento associato a \((T_{2} = 4)\), segue
              \[
                  (T_{2} = 4) = CCTT \lor CTCT \lor TCCT
              \]
              da cui pertanto
              \[
                  P(T_{2} = 4) P(CCTT) + P(CTCT) + P(TCCT)
              \]
    \end{enumerate}
\end{Example*}

\subsubsection{Distribuzione binomiale inversa}
Siano \(E_{1}, E_{2}, \dots, E_{n}\) successione di eventi stocasticamente indipendenti. Posto \(Y_{n}\) il numero di insuccessi all'i-esima prova, si ha
\begin{equation}
    P(Y = n) = P(X = n + 1) = pq^{n}
\end{equation}

\begin{Note*}
    \(Y\) non gode di assenza di memoria. Non vale cioè
    \[
        P(Y > n + m \given Y > n) = P(Y > m)
    \]
    Si dimostra infatti che \(P(Y > n + m \given Y > n) = q^{m} \text{e} P(Y > m) = q^{m + 1}\).
\end{Note*}
\end{document}