\documentclass{subfiles}
\begin{document}
Analogamente la probabilità finora trattata, quella condizionata soddisfa gli Assiomi \eqref{Axiom:1}, \eqref{Axiom:2}, \eqref{Axiom:3}. Segue quindi
\begin{itemize}
    \item \(0 \le P(E_{1} \given F) \le 1\)
          \begin{Proof*}
              la prima disuguaglianza è ovvia, la seconda discerne dal fatto che \(EF \subset F \implies P(EF) \le P(F)\).
          \end{Proof*}

    \item \(P(\Omega\given F) = 1\)
          \begin{Proof*}
              considerando l'Equazione \eqref{eq:2} segue
              \[
                  P(\Omega \given F) = \frac{P(SF)}{P(F)} = \frac{P(F)}{P(F)} = 1
              \]
          \end{Proof*}

    \item Dati \(E_{i}\) eventi, \(i = \Set{1, 2, 3, \dots}\), a due a due disgiunti, allora
          \[
              P \left( \bigvee\limits_{i = 1}^{\infty}{E_{i} \given F} \right) = \sum\limits_{i = 1}^{\infty}{P(E_{i} \given F)}
          \]
          \begin{Proof*}
              considerando nuovamente  l'Equazione \eqref{eq:2} segue
              \[\begin{aligned}
                      P \left( \bigvee\limits_{i = 1}^{\infty}{E_{i} \given F} \right) & = \frac{P \left(\left(\bigvee\limits_{i = 1}^{\infty}{E_{i}} \right) F \right)}{P(F)} \\
                                                                                       & = \frac{P \left(\bigvee\limits_{i = 1}^{\infty}{E_{i}F} \right)}{P(F)}                \\
                                                                                       & = \frac{\sum\limits_{i = 1}^{n}{P(E_{i}F)}}{P(F)}                                     \\
                                                                                       & = \sum\limits_{i = 1}^{n}{P(E_{i} \given F)}
                  \end{aligned}\]
          \end{Proof*}
\end{itemize}
\clearpage
\end{document}
