\documentclass{subfiles}
\begin{document}
Dati \(E \text{ed} H\), con le probabilità \(P(E) \text{e} P(E \given H)\) si può avere:
\[
    P(E \given H) > P(E), \quad P(E \given H) < P(E), \quad P(E \given H) = P(E)
\]
Nei primi due casi si dirà che \(E\) è rispettivamente \emph{correlato positivamente o negativamente} da \(H\).
Nel caso di \(P(E \given H) = P(E)\) invece, si dirà che \(E\) è \emph{stocasticamente} indipendente da \(H\).

\begin{Definition*}
    Dati \(E, H\) eventi, con \(H \neq \varnothing\), si dirà \(E\) stocasticamente indipendente da \(H\) se
    \begin{equation}
        P(E \given H) = P(E)
    \end{equation}
\end{Definition*}
\noindent
In generale, data una famiglia di eventi \(F\), gli eventi di \(F\) si diranno indipendenti stocasticamente se,
per ogni sottofamiglia \(\Set{E_{1}, E_{2}, \cdots, E_{n}} \text{di} F\), \(n \ge 2\) si ha
\[
    P(E_{1}E_{2}\cdots E_{n}) = P(E_{1})P(E_{2})\cdots P(E_{n})
\]
\end{document}