\documentclass{subfiles}
\begin{document}
Sia \(X\) un numero aleatorio. Questi dicasi avere distribuzione normale di parametri (\(\mu, \sigma\)) se, considerata la densità di probabilità si ha che
\[
    f(x) = \frac{1}{\sqrt{2 \pi}} e^{- \frac{1}{2} \left(\frac{x - \mu}{\sigma}\right)^{2}}, \quad \forall x \in \R
\]

\begin{Remark*}
    Si osserva facilmente che la \emph{distribuzione normale standard}, è una distribuzione normale di valori \(\mu = 0, \sigma = 1\).
\end{Remark*}

\subsubsection{Trasformazioni di numeri aleatori con distribuzione standard}
Sia \(X \sim N_{\mu, \sigma}\), sia \(Y = aX + b\) una sua trasformazione lineare.
\noindent Ci si chiede \(Y \sim N_{\mu_{1}, \sigma_{1}}\) ?

\noindent Sia considerata la distribuzione di probabilità di \(Y\), dalla sua definizione segue
\[
    F(y) = P(Y \le y) = P(aX + b \le y)
\]
\noindent si presentano i seguenti casi
\begin{enumerate}
    \item \(a < 0\) da cui: \(F(y) = P(X \ge \tfrac{y - b}{a}) = 1 - \Phi_{\mu, \sigma}(\tfrac{y - b}{a})\).
          Segue da ciò
          \[
              f(y) = \frac{1}{\sqrt{2\pi}\sigma_{1}} e^{- \frac{1}{2} \left( \frac{y - \mu_{1}}{\sigma_{1}} \right)^{2}}
          \]
          con \(\mu_{1} = a\mu + b, \sigma_{1} = -a\sigma\).

    \item \(a > 0\) da cui: \(F(y) = P(X \le \tfrac{y - b}{a}) = \Phi_{\mu, \sigma}(\tfrac{y - b}{a})\).
          Segue da ciò
          \[
              f(y) = \frac{1}{\sqrt{2\pi}\sigma_{1}a} e^{- \frac{1}{2} \left( \frac{y - \mu_{1}}{\sigma_{1}} \right)^{2}}
          \]
          con \(\mu_{1} = a\mu + b, \sigma_{1} = a\sigma\).
\end{enumerate}
\clearpage
\end{document}