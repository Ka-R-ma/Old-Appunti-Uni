\documentclass{subfiles}
\begin{document}
\begin{Definition*}
    si definiscono permutazioni \(P_{n}\) di un insieme di \(n\) elementi, il numero di disposizioni per \(r = n\). Cioè
    \begin{equation}
        P_{n} = D_{n, n} = n(n - 1)(n - 2) \cdots 1 = n!
    \end{equation}
\end{Definition*}

\begin{Example*}
    quanti sono i possibili anagrammi della parola ``COSA'', anche senza senso?
    \\
    Applicando l'Equazione \eqref{eq:6}, segue
    \[
        P_{n} = D_{n, n} = 4 \cdot 3 \cdot 2 \cdot 1  = 4! = 24
    \]
\end{Example*}
\noindent
\begin{Remark*}
    se la parola da anagrammare avesse delle lettere ripetute, l'Equazione \eqref{eq:6} non sarebbe corretta.
    In casi simili si parla di permutazioni con ripetizioni. In generale se \(k_{1}, k_{2}, \dots, k_{m}\) sono elementi ripetuti, le permutazioni distinte sono
    \[
        P_{n}^{k_{1}, k_{2}, \dots, k_{m}} = \frac{P_{n}}{k_{1}! k_{2}! \dots k_{m}!} = \frac{n!}{k_{1}! k_{2}! \dots k_{m}!}
    \]
\end{Remark*}
\end{document}