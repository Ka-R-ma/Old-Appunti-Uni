\documentclass{subfiles}
\begin{document}
Sia \(X = 0, 1, 2, \dots\) numero aleatorio, questa si dice avere distribuzione di Poisson, di parametro \(\lambda\) se
\[
    P(X = i) = e^{-\lambda} \frac{\lambda}{i!} \qquad i = 0, 1, 2, \dots \ , \lambda > 0
\]
La distribuzione di Poisson ha diverse applicazioni, tra queste l'approssimazione di un numero aleatorio con distribuzione binomiale a parametri \(n, p\).
\begin{Proof*}
    Sia \(X\) numero aleatorio a parametri \(n, p\), sia \(\lambda = np\), segue
    \[\begin{aligned}
            P(X = i) & = \frac{n!}{(n - 1)! i!} p^{i}(1 - p)^{n - 1}                                                                                                                  \\
                     & = \frac{n!}{(n - 1)! i!} \left(\frac{\lambda}{n}\right)^{i}\left(1 - \frac{\lambda}{n}\right)^{n - 1}                                                          \\
                     & = \frac{n(n - 1)(n - 2) \cdots (n - i + 1)}{n^{i}} \frac{\lambda^{i}}{i!}\frac{\left(1 - \frac{\lambda}{n}\right)^{n}}{\left(1 - \frac{\lambda}{n}\right)^{i}}
        \end{aligned}\]
\end{Proof*}
Per \(n\) molto piccoli e \(p\) sufficientemente piccoli tali che \(\lambda\) sia costante
\[\begin{aligned}
        \left(1 - \frac{\lambda}{n}\right)^{n}           & \approx e^{-\lambda} \\
        \frac{n(n - 1)(n - 2) \cdots (n - i + 1)}{n^{i}} & \approx 1            \\
        \left(1 - \frac{\lambda}{n}\right)^{i}           & \approx 1            \\
    \end{aligned}\]
segue
\[
    P(X = i) \approx e^{-\lambda} \frac{\lambda^{i}}{i!}
\]
Inoltre \(\forall i \in \N\), si ha che
\[
    P(X = i + 1) = \frac{\lambda}{i + 1} P(X = i)
\]
\clearpage
\end{document}