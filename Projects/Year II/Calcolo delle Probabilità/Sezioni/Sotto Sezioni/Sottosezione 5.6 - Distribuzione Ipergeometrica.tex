\documentclass{subfiles}
\begin{document}
Si effettuano \(n\) estrazioni senza ripetizioni da un urna contenente \(N\) palline, di cui \(pN\) bianche e \(qN\) nere, \(qN = N - pN\).
Siano
\[\begin{gathered}
        E_{i} = \text{"esce bianca all'i-esima estrazione"} \\
        X = \sum\limits_{i = 1}^{n}{\abs{E_{i}}} = \text{"numero aleatorio di palline bianche estratte nelle \emph{n} estrazioni"}
    \end{gathered}\]
Si consideri \(P(E_{i}), i \in \Set{1, \dots, n}\), segue
\[\begin{gathered}
        P(E_{1}) = \frac{pN}{N} = p     \\
        \begin{aligned}
            P(E_{2}) = P(E_{1} \cap \Omega) & = P(E_{2} \cap (E_{1} \lor E_{1}^{C}))              \\
                                            & = P(E_{2} \given E_{1}) + P(E_{2} \given E_{1}^{C}) \\
                                            & = \frac{pN - 1}{N - 1}p + \frac{pN}{N - 1}q = p     \\
        \end{aligned}
    \end{gathered}\]
Analogamente se si applica il calcolo combinatorio, segue
\[
    P(E_{2}) = \frac{(N - 1)pN}{N(N - 1)} = p
\]
In generale
\[
    P(E_{i}) = \frac{(N - 1)(N - 2) \cdots (N - i + 1)pN}{N(N - 1)(N - 2) \cdots (N - i + 1)} = p
\]
Ma \(E_{1}, E_{2}\) sono stocasticamente indipendenti?\\
Si osserva che
\[
    P(E_{2} \given E_{1}) = \frac{pN - 1}{N} \neq P(E_{2}) = \frac{pN}{N} = p
\]
Sia \(0 \le h \le n\), ci si chiede quanto valga \(P(X = h)\).\\
Si ha che \(\forall h \in \Set{0, \dots, n}, P(X = h) = P(E_{1}E_{2} \cdots E_{h} E_{h + 1}^{c} \cdots E_{n}^{C})\), vale cioé
\[
    P(X = h) = \frac{D_{pN, n} D_{qN, n - h}}{D_{N, n}} = \frac{\binom{N - m}{pN - h}}{\binom{N}{pN}}
\]
il cui risultato unicamente dipendente da \(n, h\).
\clearpage
Considerando il generico costituente
\[
    E_{i_{1}}E_{i_{2}} \cdots E_{i_{h}}E_{i_{h + 1}} \cdots E_{i_{n}}
\]
favorevole ad \((X = h)\), segue
\[
    (X = h) = \bigvee\limits_{\Set{i_{1}, \dots, i_{n}} \subseteq \Set{1, \dots, n}}{E_{i_{1}}E_{i_{2}} \cdots E_{i_{h}}E_{i_{h + 1}} \cdots E_{i_{n}}}
\]
ossia l'unione degli \(\tbinom{n}{h}\) costituenti a favore di \((X = h)\).
In definitiva
\begin{equation}
    P(X = h) = \binom{n}{h} \frac{\binom{N - n}{pN - h}}{\binom{N}{pN}}
\end{equation}

\begin{Definition*}
    si definisce la distribuzione relativa ad X \emph{ipergeometrica}. In simboli
    \[
        X \sim H(N, n, pN)
    \]
\end{Definition*}

\begin{Example*}
    un urna contiene dodici palline, di cui cinque bianche e le restanti nere. Si effettuano tre estrazioni, se
    \[\begin{gathered}
            E_{i} = \text{"l'i-esima pallina estratta è bianca"} \\
            X = \sum\limits_{i = 1}^{3}{\abs{E_{i}}}
        \end{gathered}\]
    quanto vale \(P(X = 2)\)?
    \\ \\
    Si osserva che
    \[
        P(X = 2) = P(E_{1}E_{2}\overline{E_{3}} \lor E_{1}\overline{E_{2}}E_{3} \lor E_{1}^{C}E_{2}E_{3})
    \]
    da cui applicando l'Equazione \eqref{eq:9}, segue
    \[
        P(X = 2) = \frac{\binom{3}{2}\binom{9}{3}}{\binom{12}{5}} \approx 0,318
    \]
    verificando anche con l'Equazione \eqref{eq:7}, si ha
    \[
        P(X = 2) = \binom{3}{2}\left(\frac{5}{12}\right)^{3}\left(\frac{7}{12}\right) = 0,303 \approx 0,318
    \]
\end{Example*}
\end{document}