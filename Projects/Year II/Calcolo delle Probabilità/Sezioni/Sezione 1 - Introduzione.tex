\documentclass{subfiles}
\begin{document}
Si consideri il seguente caso: lanciando tre dati, è maggiore la probabilità che la somma dei valori ottenuti sia 9 o che sia 10?\\
Un metodo semplice per stabilire quanto richiesto, consiste nell'elencare tutti i possibili modi in cui è possibile ottenere 9 e 10.

% Import Figure 1
\subfile{Figure/Figura 1.tex}
\noindent
Sia \(n\) il numero totale di casi possibili, nel caso in esame 216, la probabilità \(P\) di un evento\footnotemark[1] \(E\) risulta essere definita come segue.
\begin{equation}
    P(E) = \frac{\sum \#}{n}
\end{equation}
ove \(\#\) indica il numero di casi favorevoli all'evento in questione.
\\ \\
Applicando l'Equazione \eqref{eq:1} al caso in esame, segue
\[\begin{aligned}
        P(SOMMA = 9)  & = \frac{25}{216} \approx 0.12 \\
        P(SOMMA = 10) & = \frac{27}{216} = 0.125      \\
    \end{aligned}\]
da cui risulta ovvio \(P(SOMMA = 10) > P(SOMMA = 9)\).

\footnotetext[1]{La definizione di evento sarà data in seguito.}
\end{document}
\clearpage