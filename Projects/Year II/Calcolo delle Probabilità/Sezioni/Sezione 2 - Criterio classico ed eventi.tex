\documentclass{subfiles}
\begin{document}
Uno dei criteri alla base del calcolo probabilistico è il \emph{criterio classico della probabilità},
il quale stabilisce che dato \(E\) un evento, la probabilità \(P\) che questi si verifichi è pari al rapporto di casi favorevoli su casi possibili, cioè
\[
    P(E) = \frac{\# \text{casi a favore ad E}}{\# \text{casi possibili}} = \frac{r_{E}}{m}
\]
ove \(r_{E}\) sono i casi a favore di \(E \text{,} m\) tutti i casi possibili.

\begin{Definition*}
    una proposizione o un'affermazione di cui è possibile stabilire la veridicità è detto \emph{evento}.
\end{Definition*}
\noindent
Di un evento è possibile stabilire l'indicatore, che segnala se l'evento è o non è verificato.
\[
    \abs{E} = \begin{cases}
        1, \text{se E è vero}   \\
        0, \text{se E è falso}, \\
    \end{cases}
\]
inoltre se a priori è noto il valore di \(\abs{E}\), questi si dirà \emph{certo} se \(\abs{E} = 1\) e lo si indicherà con \(\Omega\),
\emph{impossibile} se \(\abs{E} = 0\) e lo si indicherà con \(\varnothing\).

\subsection{Proprietà degli eventi}
\subfile{Sotto Sezioni/Sottosezione 2.1 - Proprieta degli eventi.tex}

\subsection{Assiomi e proprietà derivate}
\subfile{Sotto Sezioni/Sottosezione 2.2 - Assiomi e proprieta derivate.tex}
\end{document}