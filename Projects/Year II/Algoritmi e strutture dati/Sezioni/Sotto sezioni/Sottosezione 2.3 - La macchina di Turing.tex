\documentclass{subfiles}
\begin{document}
Prima di discutere la macchina di Turing, è necessario parlare di \emph{relazione polinomiale}
\begin{Definition*}
    Due funzioni \(f_{1}(n) \text{e} f_{2}(n)\) sono in relazione polinomiale se esistono \(p_{1}(x) \text{e} p_{2}(x)\) tali da soddisfare la seguente relazione.
    \[
        f_{1}(n) \le p_{1}(f_{2}(n)) \land f_{2} \le p_{2}(f_{1}(n)) \qquad , \forall n
    \]
    \begin{Example*}
        Siano \(f_{1}(n) = 2n^{2} \text{e} f_{2}(n) = n^{5}\): si osserva che queste sono in relazione polinomiale.
        Infatti se \(p_{1}(x) = 2x \text{e} p_{2}(x) = x^{3}\) segue: \(2n^{2} \le 2n^{5} \text{e} n^{5} \le (2n^{2})^{3}\)
    \end{Example*}
\end{Definition*}

Si considera ora un nuovo modello di calcolo, la \emph{macchina di Turing}.

\begin{Definition*}
    Una macchina di Turing (MT) multi-nastro si compone di \emph{k} nastri, ciascuno dei quali è diviso in celle, ciascuna delle quali contiene un simbolo di nastro.
    Le operazioni sono dettate da un programma primitivo:il \emph{controllo finito}.
\end{Definition*}

\noindent In una computazione, in accordo col controllo finito e il carattere puntato dalla testina di ciascun nastro, una MT può effettuare almeno un delle seguenti operazioni.
\begin{enumerate}
    \item Cambiare lo stato del controllo finito.
    \item Sovrascrivere uno o più caratteri nelle celle indicate dalle testine.
    \item Per ciascun nastro, in maniera indipendente, spostare la testina verso destra, sinistra o lasciarla li.
\end{enumerate}

Formalmente una generica MT viene identificata tramite la settupla\footnote[4]{Per il significato di ciascun componente si veda "Appunti di Informatica Teorica".}
\[
    (Q, T, I, \delta, b, q_{0}, q_{f})
\]
L'attività di una MT può formalmente essere descritta tramite istantanee (ID). Quest'ultime sono k-ple \(\alpha_{1}, \ldots, \alpha_{k}\),
con \(\alpha_{i}\) una stringa del tipo \(xqy\) tale che, \(xy\) sia la stringa dell'i-esimo nastro esclusi i black iniziali e finali, con \(q\) lo stato della MT.
\end{document}