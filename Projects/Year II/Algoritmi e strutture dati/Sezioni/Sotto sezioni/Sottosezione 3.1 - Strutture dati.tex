\documentclass{subfiles}
\begin{document}
In questa sezione saranno trattate brevemente strutture quali \emph{grafi} e \emph{alberi}.
L'analisi di strutture quali \emph{insiemi}, \emph{dizionari} e varianti degli alberi saranno analizzate nelle successive sezioni.

\subsubsection{Grafi}
\begin{Definition*}
    Un grafo \(G = (V, E)\) si compone di un insieme finito e non vuoto di vertici \(V\), e uno di archi \(E\).
    Si dirà che \(G\) è \emph{orientato} (o diretto) se

    \begin{equation}
        \forall (v, w) \in E, v \to w\footnote[5]{Con \(v \to w\) si intende che (v, w) è una coppia orientata.}
    \end{equation}

    \noindent Si dirà che \(G\) non è orientato se non vale \emph{Eq. \eqref{eq:1}}.
\end{Definition*}

Dato \(G\) un grafo orientato, si dirà che due vertici \(v \text{e} w\) sono \emph{adiacenti} se \((v, w) \in E\).
Se \(G\) è invece un grafo non orientato, i dirà che due vertici \(v \text{e} w\) sono \emph{adiacenti} se \((v, w) \in E \land (w, v) \in E\):
si assume infatti che \((v, w) = (w, v)\).

\begin{Definition*}
    Dato \(G\) un grafo, orientato o meno, dicasi \emph{cammino} una sequenza di archi del tipo \((v_{1}, v_{2}), (v_{2}, v_{3}), \ldots, (v_{n- 1}, v_{n})\).
\end{Definition*}

\begin{Definition*}
    Un cammino semplice\footnote[6]{Un cammino si dice semplice se ogni arco viene percorso una sola volta} di lunghezza almeno 1, che inizia e finisce in uno stesso vertice, è detto \emph{ciclo}.
\end{Definition*}

\begin{Remark*}
    In un grafo non diretto, un ciclo ha lunghezza almeno 3.
\end{Remark*}

\noindent Dato un grafo \(G\) questi può essere rappresentato in vari modi, tra queste vi è la rappresentazione come \emph{matrice di adiacenza}.
Questa è una matrice \(\norm{V} \cp \norm{V}\) di 0 e 1 ove \(a_{ij} == 1 \iff (v_{i}, v_{j}) \in E\). Considerandone i vantaggi e gli svantaggi,
tale rappresentazione è ottimale poiché permette di verificare la presenza di un arco in tempo costante, d'altro canto pero lo spazio richiesto è \(\order{\norm{V}^{2}}\).
\\
Ulteriore rappresentazione è effettuata per \emph{liste di adiacenza}: per ciascun vertice di \(V\) è memorizzata una lista di vertici ad esso adiacente.
\clearpage

\subsubsection{Alberi}
\begin{Definition*}
    Dato \(G\) un grafo orientato, aciclico, questi è detto albero se soddisfa le seguenti proprietà.
    \begin{enumerate}
        \item Esiste un vertice \(v\), la \emph{radice}, tale che in esso non entrino archi.
        \item Ogni vertice, meno la radice, ha un unico arco d'entrata.
        \item Esiste un cammino, che dimostrasi unico, dalla radice verso ogni vertice.
    \end{enumerate}
\end{Definition*}

\begin{Note*}
    Se \(G\) è un grafo diretto, composto da alberi questi questo è detto \emph{foresta}.
\end{Note*}

\begin{Definition*}
    Sia \(F = (V, E)\) una foresta. Se \((v, w) \in E\) dicasi \(v\) \emph{antenato} e \(w\) \emph{discendente}.
    Un vertice senza discendenti è detto \emph{foglia}. Dicasi inoltre l'insieme di un vertice e dei suoi discendenti \emph{sotto-albero} di \(F\).
\end{Definition*}

\noindent La \emph{profondità di un vertice} in un albero, è la lunghezza del cammino che vi è dalla radice al vertice.
Dicasi \emph{altezza di un vertice} la lunghezza del cammino più lungo dal vertice ad una sua foglia; se il vertice è la radice si parlerà di \emph{altezza dell'albero}.
In fine il \emph{livello di un vertice} è l'altezza dell'albero a cui si sottrae la profondità del vertice.

\begin{Definition*}
    Un albero binario è detto \emph{completo} se, per un qualche \(k\), tutti i vertici con profondità minore di \(k\) hanno sia figlio destro che sinistro,
    e ogni vertice ddìi profondità \(k\) è una foglia.
\end{Definition*}

\begin{Remark*}
    Un albero binario di altezza \(k\) ha esattamente \(2^{k + 1} - 1\) foglie.
\end{Remark*}

\noindent Molti algoritmi fanno uso degli alberi, sfruttando in genere la visita dello stesso. Tali visite sono riportate di seguito.
\begin{itemize}
    \item \textbf{pre-ordine:} si procede col visitare la radice successivamente, procedendo ricorsivamente, il sotto-albero sinistro e destro.
    \item \textbf{post-ordine:} si procede a visitare il sotto-albero sinistro, il desto e in ultimo la radice.
    \item \textbf{in-ordine:} si procede con il visitare il sotto-albero sinistro, la radice, e in fine il sotto-albero destro.
\end{itemize}
\end{document}