\documentclass{subfiles}
\begin{document}
\begin{Note*}
    In questa sezione saranno discusse solamente strutture elementari quali code, dizionari, pile, grafi e alberi, per le strutture avanzate si rimanda alle %TODO: uncomment \emph{Sezioni \ref{Sec:5} e \ref*{Sec:6}}.
\end{Note*}

\subsubsection{Code}
Una coda è una sequenza \(S\) di n elementi.
Operazioni consentite sulle code sono le seguenti.
\begin{itemize}
    \item ISEMPTY(): verifica se \(S\) è vuota o meno.
    \item ENQUEUE(elem e): aggiunge e come ultimo elemento.
    \item DEQUEUE(): toglie da \(S\) il primo elemento e lo restituisce.
    \item FIRST(): restituisce il primo elemento senza toglierlo
\end{itemize}

\end{document}