\documentclass{subfiles}
\begin{document}
In questa sezione si analizzeranno alcuni algoritmi di ordinamento la cui complessità è \(\order{n^{2}}\).

\subsubsection{Selection sort}
Ricevuto l'input, l'algoritmo procede con l'estrarre all'i-esimo passo il minore degli n - i elementi rimasti.
Si osserva dunque che la complessità dell'algoritmo è dato dalla seguente sommatoria.

\[
    \sum\limits_{i = 1}^{n - 1}{i} = \frac{n(n - 1)}{2} \implies \order{n^{2}}
\]

\noindent Di seguito si riporta la struttura algoritmica del selection sort.
\subfile{../Figure/Figura 4.1 - Selection sort.tex}

\subsubsection{Insertion sort}
Ricevuto l'input, all'i-esimo passo, l'i-esimo elemento e ordinato rispetto gli i - 1, elementi precedentemente ordinati.
Si osserva pertanto che la complessità dell'algoritmo è dato dalla seguente sommatoria.

\[
    \sum\limits_{i = 1}^{n - 1}{i} = \frac{n(n - 1)}{2} \implies \order{n^{2}}
\]

\noindent Di seguito si riporta la struttura algoritmica del selection sort.
\subfile{../Figure/Figura 4.2 - Insertion sort.tex}

\subsubsection{Bubble sort}
Ricevuto l'input, all'i-esimo passo, si confronta un coppia adiacente di elementi, se non sono ordinati si procede a scambiarli.
La complessità dell'algoritmo è dato dalla seguente sommatoria.

\[
    \sum\limits_{i = 1}^{n - 1}{n - 1} = (n -1)^{2} \implies \order{n^{2}}
\]

\noindent Di seguito si riporta la struttura algoritmica del selection sort.
\subfile{../Figure/Figura 4.3 - Bubble sort.tex}
\end{document}