\documentclass{subfiles}
\begin{document}
\begin{Definition*}
    Sia \(G = (V, E)\) un grafo diretto.
    Siano \(V_{i}, 1 \le i \le k\) partizioni di \(V\), tali che presi due vertici \(v, w\),
    i due sono in relazione se e solo se esiste un cammino dall'uno all'altro e viceversa.
    Siano \(E_{i}, 1 \le i \le k\) le partizioni di \(E\) che collegano vertici di \(V_{i}\).
    \\
    Il sottografo \(G_{i} = (V_{i}, E_{i})\) è detta \emph{componente fortemente connessa} di \(G\).
    Un grafo è detto fortemente connesso se e solo se esiste un'unica componente fortemente connessa al suo interno.
\end{Definition*}

\noindent Si analizza ora la relazione tra componenti fortemente connesse e DFS.

\begin{Lemma}
    Sia \(G_{i} = (V_{i}, E_{i})\) una componente fortemente connessa di un certo grafo orientato \(G\).
    Sia \(S = (V, T)\) Spanning-Forest per \(G\). Allora i vertici di \(G_{i}\) e gli archi di \(E_{i} \cap T\) formano un albero.

    \begin{Proof*}
        Siano \(v, w \text{vertici in} V_{i}\). Senza perdita di generalità, si assuma \(v < w\).
        Poiché \(v, w\) sono nella stessa componente fortemente connessa, esiste un cammino \(p\) tra i due.
        Sia \(x\) il vertice di \(p\) minore. Giunti ad un discendente di \(x\) tramite \(p\), accade che non si può uscire dal sotto-albero di discendenti di \(x\).
        Dunque \(w\) è discendente di \(x\). Poiché ciascun vertice è numerato in pre-ordine, ogni vertice compreso tra \(x, w\) è discendente di \(x\).
        Segue che poiché \(x \le v < w, v\) è discendente di \(x\).
        \\ \\
        Sia \(r\) antenato comune dei vertici in \(G_{i}\), col DFNUMBER minore.
        Se \(v\) è in \(G_{i}\), ogni vertice nel cammino tra \(r, v\) è in \(G_{i}\).
    \end{Proof*}
\end{Lemma}

\begin{Lemma}
    Per ogni \(1 \le i \le j, G_{i}\) si compone di tutti i vertici discendenti da \(r_{i}\), che non sono in \(G_{1}, \ldots, G_{i - 1}\).
    \begin{Proof*}
        La radice \(r_{j} \text{, per } j > i\) non può discendere da \(r_{i}\). Infatti la chiamata a \(SEARCH(r_{i})\) termina prima di quella a \(SEARCH(r_{j})\).
    \end{Proof*}
\end{Lemma}


\noindent Al fine di semplificare la ricerca delle radici, sia definita la seguente funzione.
\[
    \begin{aligned}
        LOWLINK[v] = MIN(\Set{v} \cup \{w \such & \text{esista un cross edge, o un back edge tra} v \text{e} w. \\
                                                & \text{Inoltre la radice contente} w \text{è antenato di} v\}) \\
    \end{aligned}
\]
\clearpage

\begin{Lemma}
    Sia \(G\) un grafo orientato. Un vertice \(v\) è radice di una componente fortemente connessa se e solo se \(LOWLINK[v] = v\).
    \begin{Proof*}
        Sia \(LOWLINK[v] = v\). Se \(v\) non è la radice di una componente fortemente connessa contente \(v\) stesso, allora un antenato proprio \(r \text{di} v\) lo è.
        Esiste quindi un cammino \(p\) tra \(v, r\). Si considera il primo arco in \(p\) da un discendente di \(v\) ad un certo vertice \(w\), che non discenda da \(v\).
        Segue che \((v, w)\) è un back edge o un cross edge. comune \(w < v\). In quanto, \(r \text{è antenato di} v\),
        segue che esiste un cammino da \(v \text{a} w\).
        Si conclude che \(w,r\) sono nella stessa componente fortemente connessa; se però ciò fosse vero \(LOWLINK[v] \le w < v\): una contraddizione.

        Sia viceversa \(v\) radice di una componente fortemente connessa di un grafo \(G\).
        Per definizione stessa \(LOWLINK[v] \le v\). Si supponga \(LOWLINK[v] < v\).
        Segue che esistono \(w, r\) tali che
        \begin{enumerate}
            \item \(w\) sia raggiunto da un discendente di \(v\) tramite back edge o cross edge;
            \item \(r\) è radice della componente fortemente connessa contente \(w\);
            \item \(r \text{è antenato di} v\);
            \item \(w < v\).
        \end{enumerate}

        \noindent Per il secondo punto \(r \le w\). Per il quarto \(r < v\), implicando, per il punto tre, che \(r\) sia antenato proprio di \(v\).
        Segue però che, poiché in \(G\) esiste un cammino da \(v \text{a} r\) passante per \(w\), quest'ultimi sono nella stessa componente fortemente connessa;
        concludendo pertanto che \(v\) non sia radice: una contraddizione.
    \end{Proof*}
\end{Lemma}

\noindent La ricerca delle componenti fortemente connesse è riportata di seguito.
\subfile{../Figure/Figura 6.3 - Algoritmo per la ricerca delle CFC.tex}
\end{document}