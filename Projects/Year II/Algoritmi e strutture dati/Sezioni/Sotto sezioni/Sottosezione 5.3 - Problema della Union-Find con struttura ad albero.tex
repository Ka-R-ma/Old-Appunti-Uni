\documentclass{subfiles}
\begin{document}
Il problema della union-find può essere semplicemente descritto come segue: dati degli insiemi, come gestire efficientemente operazioni di unione e ricerca?

\begin{Note*}
    Esistono soluzioni che sfruttano gli array, ma per quanto efficienti risultano lente se comparate a quelle di seguito proposte.
\end{Note*}

\noindent Ogni insieme \(A\) può essere rappresentato come un albero non orientato, i cui vertici sono elementi di \(A\).
\\ \\
Considerando l'istruzione UNION, se \(T_{A} \text{e} T_{B}\) rappresentato, rispettivamente, gli insiemi \(A \text{e} B\), segue che questa richiede tempo costante.
Si osserva infatti che basta aggiungere un puntatore tra i due alberi.
Circa l'istruzione FIND, questa è effettuata localizzando il vertice ricercato e percorrendo il cammino dallo stesso,
verso la radice. Risulta ovvio che questo richiede al più \(\order{n}\)\footnote[7]{Questo perché potrebbe capitare che l'albero risulti sbilanciato.} operazioni.
\\ \\
Si osserva che il costo dell'istruzione FIND può essere ridotto, se l'albero è bilanciato.
Come ottenere tale bilanciamento? Sfruttando il seguente lemma.

\begin{Lemma}
    Se nell'eseguire un'istruzione di UNION, si fa in modo che l'albero di altezza minore punti a quello di altezza maggiore,
    non esisterà albero nella foresta\footnote[8]{Con foresta in questo caso ci si riferisce ad altri possibili insiemi su cui si deve eseguire una UNION} di altezza \(h\),
    a meno che questi non abbia almeno \(2^{h}\) vertici.

    \begin{Proof*}
        La dimostrazione procede per induzione su \(h\).
        \begin{Base*}
            Sia \(h = 0\), l'ipotesi è ovviamente vera, poiché ogni albero ha almeno un nodo.
        \end{Base*}
        \begin{Induction*}
            Si supponga che quanto detto sia vero per ogni valore minore di \(h \ge 1\).
            Sia \(T\) un albero di altezza \(h\) col numero minore di foglie. Allora \(T\) è ottenuto come UNION di \(T_{1} \text{e} T_{2}\),
            con \(\norm{T_{1}} = h - 1 \text{e} T_{1}\) ha al più lo stesso numero di vertici di \(T_{2}\).
            Per ipotesi induttiva \(T_{1} \text{e} T_{2}\) hanno almeno \(2^{h - 1}\) vertici, per cui \(T\) ne ha almeno \(2^{h}\).
        \end{Induction*}
    \end{Proof*}
\end{Lemma}

\noindent Una modifica all'algoritmo prevede la \emph{path compression}. Si è visto che il costo maggiore dell'intero algoritmo è dato dall'istruzione FIND,
pertanto ciò che è di interesse è ridurre tale costo. Si è detto che un'istruzione FIND(i) comporta ol dover percorrere un cammino del tipo, \(i, v_{1}, \ldots, v_{k}, r\).
Se si rende ciascuno di questi vertici figlio di \(r\), si dimostra che le FIND successive risultano più efficienti.
\end{document}