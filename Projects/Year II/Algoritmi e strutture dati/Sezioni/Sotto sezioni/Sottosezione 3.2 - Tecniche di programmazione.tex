\documentclass{subfiles}
\begin{document}
In questa sezione si discuteranno alcune tecniche di programmazione, quali \emph{divide-and-conquer} e \emph{programmazione dinamica}.

\subsubsection{Divide-and-conquer}
La tecnica divide-and-conquer si fonda sull'idea di suddividere un problema in problemi di taglia inferiore e, risolto ciascun sotto-problema,
unire le sotto-soluzioni per risolvere in problema di partenza.
\\ \\
Si supponga di dover ricercare il massimo ed il minimo in un array di elementi S. Un idea semplice è quella di effettuare una procedura del tipo a seguire.
\subfile{../Figure/Figura 4 - Ricerca elementare del massimo.tex}

\noindent Con un algoritmo tipo quello sopra, si impiegano n - 1 confronti per ricercare il massimo, ed n - 2 per la ricerca del minimo.
\\ \\
Applicando ora una procedura divide-and-conquer, si dimostra che è possibile ricercare sia il massimo che il minimo in meno di \(\order{n^{2}}\) confronti.
Sia considerato l'algoritmo a seguito riportato, e sia \(T(n)\) il numero di confronti effettuati.
\subfile{../Figure/Figura 5 - Algoritmo MAXMIN.tex}

\noindent Analizzando il numero di confronti si osserva
\[
    T(n) = \begin{cases}
        1, \qquad           & n = 2 \\
        2T(n/2) + 2, \qquad & n > 2 \\
    \end{cases}
\]
\noindent Si dimostra che \(T(n) = \tfrac{3}{2} n - 2\) soddisfa la relazione ricorsiva.
\clearpage

\noindent Si consideri il prodotto di due interi a n bit. Con il metodo classico, si impiegherebbero \(\order{n^{2}}\) operazioni;
se si applica invece il metodo a seguire, si limita tale numero a \(\order{n^{\log_{2}{3}}}\).
Siano \(x \text{e} y\) i due interi a n bit. Si proceda con il dividere i due, in altrettante parti:
considerando ciascuna parte come numero a n/2 bit, il prodotto può essere espresso come
\[
    \begin{aligned}
        xy & = (2^{n/2}a + b)(2^{n/2}c + d)     \\
           & =  2^{n}ac + (ad + bc)2^{n/2} + bd
    \end{aligned}
\]
\noindent Un'implementazione algoritmica risulta essere la seguente
\subfile{../Figure/Figura 6 - Prodotto di interi a n bit.tex}

\noindent il cui costo computazionale risulta essere
\[
    T(n) = \begin{cases}
        c, \qquad            & n = 1 \\
        3T(n/2) + cn, \qquad & n > 1 \\
    \end{cases}
\]
\subsubsection{Master Theorem}
Come visto, la tecnica di divide-and-conquer genera delle complessità di tempo espresse da equazioni ricorsive, le quali non risultano semplici da analizzare.
Per tale ragione si necessità di uno strumento teorico che ne facilita il calcolo, tale strumento è il \emph{Master Theorem}.

\begin{Theorem*}[Master]
    Data una funzione ricorsiva del tipo
    \[
        T(n) = aT(\frac{n}{b}) + f(n)
    \]
    con \(a \ge 1, b > 1 \text{ed} f\) asintoticamente positiva, allora valgono i seguenti casi:
    \begin{itemize}
        \item \textbf{Caso 1.} Se \(f(n) = \order{n^{\log_{b}{a - \varepsilon}}}, \varepsilon > 0\),  allora \(T(n) = \Theta(n^{\log_{b}{a}})\).
        \item \textbf{Caso 2.} Se \(f(n) = \Theta(n^{\log_{b}{a}})\),  allora \(T(n) = \Theta(n^{\log_{b}{a}} \log n)\).
        \item \textbf{Caso 3.} Se \(f(n) = \Omega(n^{\log_{b}{a + \varepsilon}}), \varepsilon > 0\),  allora \(T(n) = \Theta(f(n))\).
    \end{itemize}
\end{Theorem*}
\clearpage

\subsubsection{Programmazione dinamica}
La programmazione dinamica prevede la risoluzione di problemi, tramite la suddivisione e soluzione ricorsiva di sotto-problemi dello stesso.
\\ \\
Affinché un problema sia risolvibile con la programmazione dinamica è necessario che questo soddisfi certe condizioni:
\begin{enumerate}
    \item deve presentare una sotto-struttura ottimale, con la quale si intende la possibilità di trovare una soluzione ottimale come combinazione delle soluzioni ai suoi sotto-problemi;
    \item i sotto-problemi devono essere sovrapponibili.
\end{enumerate}

\noindent In generale la computazione procede da sotto-problemi di taglia inferiore a sotto-problemi di taglia maggiore.
Vantaggio di ciò è il fatto che ogni qual volta in problema di taglia minore è risolto, questi è memorizzato cosi da non dover essere eventualmente ricalcolarlo qualora servisse.

\begin{Example*}
    Si consideri dunque il seguente prodotto, poiché l'ordine di moltiplicazione influenza il numero di operazioni totali, ci si chiede quale ordine convenga scegliere.
    \[
        M = M_{1} \cp M_{2} \cp M_{3} \cp M_{4}
    \]
    \noindent Rispettivamente di dimensioni \([10 \cp 20], [20 \cp 50], [50 \cp 1], [1 \cp 100]\).
    \\
    \'E ovvio che valutare tutti i possibili prodotti sia sconveniente, specie al crescere di n.
    Fortunatamente la programmazione dinamica fornisce un'algoritmo con complessità \(\order{n^{3}}\), la cui implementazione algoritmica è riportata al termine della pagina.
    \\
    Sia \(m_{ij}\) il costo minimo per calcolare \(M_{i} \cp \ldots \cp M_{j}, 1 \le i \le j \le n\). Segue
    \[
        m_{ij} = \begin{cases}
            0, \qquad                                                                        & \text{se} i = j \\
            \min\limits_{i \le k \le j}{(m_{ik} + m_{k+1, j} + r_{i - 1}r_{k}r_{j})}, \qquad & \text{se} j > i
        \end{cases}
    \]
    \noindent ove \(m_{ik}\) è il costo minimo per calcolare \(M' = M_{i} \cp \ldots \cp M_{k}\), \(m_{k + 1, j}\) è il costo minimo per valutare \(M" = M_{k + 1} \cp \ldots \cp M_{j}\),
    infine \(r_{i - 1}r_{k}r_{j}\) è il costo per valutare \(M' \cp M"\).
    \subfile{../Figure/Figura 7 - Moltiplicazione tra matrici.tex}

\end{Example*}

\subsubsection{Tecniche Greedy}
L'utilizzo di tecniche Greedy è opportuno per quei problemi d'ottimo, in cui scegliere ad ogni passo l'opzione migliore, comporta una soluzione ottimale nel complesso.
\\ \\
Problema legato alle tecniche Greedy; non è tanto la ricerca dell'algoritmo, quanto più dimostrare che l'algoritmo scelto produca effettivamente la soluzione ottima.
In genere, per provare che l'algoritmo scelto sia quello ottimo, si procede con un ragionamento per assurdo, sulla base di quanto segue.
\\ \\
Siano \(S_{g} \text{e} S_{O}\), rispettivamente, la successione di scelte Greedy e quella ottima.
Sia per assurdo \(S_{g} \neq S_{O}\). Da tale assunzione deve esistere almeno una scelta per cui le due soluzioni differiscano.
Si considera la prima di tali scelte e si procede con definire una nuova successione di scelte \(S_{O}'\), ottenuta da \(S_{O}\) sostituendo la scelta non Greedy con quella Greedy.
Si verificano i seguenti casi:
\begin{enumerate}
    \item risulta che \(S_{O}'\) sia meno ottimizzata rispetto \(S_{O}\). Da ciò segue,
          poiché le scelte Greedy per definizione sono da effettuare passo dopo passo, segue che effettivamente \(S_{O} \neq S_{g}\).
    \item risulta che \(S_{O}'\) sia ugualmente o meglio ottimizzata di \(S_{O}\). Da cui applicare la scelta Greedy è conveniente.
\end{enumerate}

Se iterando per ciascuna delle scelte che differiscono tra \(S_{g} \text{e} S_{O}\), non si rientra nel primo caso, \(S_{g}\) è ottima.
\end{document}