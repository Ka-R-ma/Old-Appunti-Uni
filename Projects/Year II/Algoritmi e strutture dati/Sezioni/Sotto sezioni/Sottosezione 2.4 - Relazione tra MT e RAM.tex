\documentclass{article}
\begin{document}
Principale applicazione delle MT è quello di determinare un \emph{lower bound} al tempo e allo spazio necessari alla risoluzione di un problema.
\\ \\
Considerando la relazione tra MT e RAM, risulta ovvio che una RAM possa simulare una MT a k nastri, semplicemente mantenendo un cella del nastro della MT in un registro.
Si supponga una MT con complessità di tempo \(T(n) \ge n\); ne segue che una RAM legga gli input in \(\order{T(n)}\) se a costo uniforme,
in \(\order{T(n) \ln T(n)}\) se con costo logaritmico. In ambo i casi, il tempo di una RAM è limitato superiormente dal tempo della MT.
Un risultato inverso, cioè la possibilità che una MT simuli un RAM, vale solo se la RAM è con costo logaritmico.
In tal caso vale il teorema a seguire.

\begin{Theorem}
    Sia \(L\) un linguaggio accettato da una RAM con costo logaritmico, in tempo \(T(n)\).
    Se la RAM non fa uso di istruzioni \lstinline[language = RAM]{MULT} e/o \lstinline[language = RAM]{DIV}, allora la complessità di tempo è al più \(\order{T^{2}(n)}\).
\end{Theorem}

\begin{Proof*}
    Sia rappresentato ciascun registro della RAM, non contenente zeri, come in \emph{Figura \ref{Fig:2.3}}.
    \subfile{../Figure/Figura 2.3 - MT simulante RAM con costo logaritmico.tex}

    \noindent Il nastro è dunque una sequenza di coppie \((i_{j}, c_{j})\), scritte in binario, separate da un delimitatore.
    Il contenuto dell'accumulatore è memorizzato su di un secondo nastro, e un terzo nastro è usato come supporto.
    In aggiunta a questi vi sono due nastri aggiuntivi: uno per gli input e uno per gli output della RAM.
    Segue che un passo della RAM è rappresentato da un insieme finito di passi della MT.
    \\ \\
    Si procede ora col dimostrare che una RAM con costo computazionale \(k\) richiede al più \(\order{k^{2}}\) passi di una MT.
    Il costo per memorizzare \(c_{j} \text{in} i_{j}\) è \(l(c_{j}) + l(i_{j})\), da cui si conclude che la lunghezza di caratteri non blank è \(\order{k}\).
    \\
    La simulazioni di istruzioni diverse da \lstinline[language = RAM]{STORE} sono \(\order{k}\), poiché il costo principale è dato dalla ricerca nel nastro.
    Analogamente il costo di uno \lstinline[language = RAM]{STORE} è dato dal tempo di ricerca nel nastro, in aggiunta al tempo necessario per copiarlo entrambe \(\order{k}\).
    Da cio si deduce che, a meno di \lstinline[language = RAM]{MULT} e/o \lstinline[language = RAM]{DIV}, un'istruzione RAM può essere simulata in \(\order{k}\) passi della MT.
    Infine poiché sotto il criterio logaritmico ciascuna istruzione RAM costa almeno un'unita di tempo, il costo totale speso dalla MT è \(\order{k^{2}}\)
\end{Proof*}
\end{document}