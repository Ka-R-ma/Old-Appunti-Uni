\documentclass{article}
\begin{document}
La \emph{tesi di Church} stabilisce che tutti i modelli di calcolo sono equivalenti, ossia che risolvono i medesimi problemi, differendo solo per le prestazioni.
\\ \\
Considerato ciò, segue che RAM ed MT devono essere equivalenti.
Sia considerato l'esecuzione della funzione \(f(n) = 2^{(2^{n})}\).
Si osserva che se calcolata con una RAM con costo uniforme, applicando doubling, questa richiede \(\order{n}\) unita di tempo.
Considerata una MT, d'altra parte si ha che, tralasciato il calcolo effettivo di \(f(n)\), sono necessari almeno \(2^{n}\) celle per la sola rappresentazione.
Inoltre poiché nel caso delle MT lo spazio è un lower bound del tempo, segue che sono necessarie \(\Omega(2^{n})\) unità di tempo.
\\ \\
Da quanto detto segue che in apparenza non vi sia relazione tra i due modelli.
Ma come detto quello analizzato è il modello RAM a costo uniforme, che per certi versi è inverosimile nella realtà.
\\
Sia quindi considerato il modello RAM con costo logaritmico, poiché per accedere in memoria questi impiega tempo proporzionale all'operando,
sempre con doubling si ha che tale modello richiede \(\order{n2^{n}}\) unità di tempo.
Segue che MT e RAM con costo logaritmico siano in relazione polinomiale tra loro, dimostrandone l'equivalenza tra i due.
\end{document}