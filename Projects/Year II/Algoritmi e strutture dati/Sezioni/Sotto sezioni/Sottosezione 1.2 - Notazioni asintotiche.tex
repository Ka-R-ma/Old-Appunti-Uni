\documentclass{subfiles}
\begin{document}
Le diverse complessità di un'algoritmo possono essere studiate secondo tre aspetti: \emph{caso ottimo, caso pessimo, caso medio}.

\subsubsection{Caso ottimo: notazione Omega}
La notazione Omega \(\Omega\) definisce un limite inferiore ad un funzione \(f(n)\).
In generale, data una certa funzione \(g(n)\) si definisce \(\Omega(g(n))\) come segue.

\[
    \Omega(g(n)) = \set{f(n)}{\exists c \in \R, n_{0} \in \N, c,n > 0 \given 0 \le c \cdot g(n) \le f(n), \forall n \ge n_{0}}
\]

\subsubsection{Caso pessimo: notazione O-grande}
La notazione O-grande definisce un limite superiore ad una funzione \(f(n)\).
In generale, data una certa funzione \(g(n)\) si definisce \(\order{g(n)}\) come segue.

\[
    \order{g(n)} = \set{f(n)}{\exists c \in \R, n_{0} \in \N , c, n_{0} > 0 \given f(n) \le c \cdot g(n), \forall n \ge n_{0}}
\]

\subsubsection{Caso medio: notazione Theta}
La notazione Theta \(\Theta\) definisce dei limiti ad una funzione \(f(n)\).
In generale, data una certa funzione \(g(n)\) si definisce \(\Theta(g(n))\) come segue.

\[
    \Theta(g(n)) = \set{f(n)}{\exists c_{1}, c_{2} \in \R, n_{0} \in \N \given c_{1} \cdot g(n) \le f(n) \le c_{2} \cdot g(n), \forall n \ge n_{0}}
\]

\end{document}