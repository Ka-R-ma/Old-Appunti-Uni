\documentclass{subfiles}
\begin{document}
Operazioni elementari comuni ad un dizionari e a una coda con priorità sono: INSERT, DELETE; operazioni specifiche risultano essere rispettivamente:
MEMBER per i dizionari e MIN per le code con priorità.
\\ \\
Considerando l'istruzione INSERT (in \emph{Figura \ref{Fig:5.1}} se ne riporta l'implementazione algoritmica): sia \(a\) un elemento da inserire.
Per prima cosa si ricerca la posizione per una foglia \(l\) che conterrà \(a\).
Assunto che l'albero abbia più di un elemento, la ricerca prosegue fin quando si trova un vertice \(f\) con due o tre figli.
Se \(f\) ha solo due figli, si supponga \(l_{1} \text{e} l_{2}\), si fa in modo che \(l\) diventi figlio di \(f\).
\\
Definendo \(E[x]\) il valore contenuto nel vertice \(x\), \(L[x]\) il massimo valore contenuto nel sotto-albero con radice il figlio sinistro di \(x\),
ed \(M[x]\) il massimo valore contenuto nel sotto-albero con radice il figlio centrale di \(x\); si presentano tre casi.
\begin{enumerate}
    \item \(a < E[l_{1}]\), allora si rende \(l\) figlio sinistro di \(f\). Inoltre, \(L[f] = a \text{e} M[f] = l_{1}\).
    \item \(E[l_{1}] < a < E[l_{2}]\), si rende \(l\) figlio centrale di \(f\). Inoltre \(M[f] = a\).
    \item \(E[l_{2}] < a\), si rende \(l\) il terzo figlio di \(f\).
\end{enumerate}

\noindent Supponendo che \(f\) abbia tre figli, si supponga \(l_{1}, l_{2}, l_{3}\), segue che dover rendere \(l\) figlio di \(f\), comporta che questi abbia quattro figli,
violando così la proprietà di albero 2-3. Sia \(g\) un nuovo vertice, al fine di mantenere la proprietà di albero 2-3, si rendono i due figli più a destra di \(f\) figli di \(g\).
In fine, si rendono \(f \text{e} g\) figli dello stesso padre, eventualmente ripetendo ricorsivamente la procedura.
\\
Caso particolare è quello della radice. Se la radice ha quattro figli, si crea una nuova radice con due figli,
ciascuno dei quali a loro volta avrà due dei figli della precedente radice.
\subfile{../Figure/Figura 5.1 - Procedura AddSon.tex}

\begin{Theorem}
    La procedura di \emph{Figura \ref{Fig:5.1}}, inserisce un nuovo elemento in \(\order{\log n}\). In più mantiene l'ordine originale e conserva la struttura di albero 2-3.
\end{Theorem}
\clearpage

\noindent Analizzando ora l'operazione di DELETE, sia \(x\) il vertice da eliminare, si osservano le seguenti possibilità.
\begin{enumerate}
    \item Il vertice \(x\) è la radice: si procede eliminandola\footnote[9]{Caso banale: si ha un'unico vertice.}.
    \item Il vertice \(x\) è figlio di un vertice con tre figli: lo si elimina.
    \item Il vertice \(x\) è figlio di un vertice  \(f\) con due figli, si hanno i seguenti casi.
          \begin{enumerate}
              \item \(f\)è la radice: si eliminano sia \(x\) che la radice; l'ultimo figlio è reso la nuova radice.
              \item Il padre non è la radice. Si supponga che \(f\) abbia una fratello \(g\) con due figli, si rende il vertice da non eliminare figlio di \(g\).
                    Procedendo con l'eliminare \(f \text{ed} x\). Se \(g\) ha tre figli, si rende il figlio più a destra figlio sinistro di \(f\); si procede con eliminare \(x\).
          \end{enumerate}
\end{enumerate}

\noindent Considerando in fine l'operazione di MIN: si osserva che per costruzione, il minimo risiede nella foglia più a sinistra, richiedendo dunque tempo \(\order{\log n}\).
\\ \\
In conclusione, gli alberi due tre permettono di implementare dizionari e code con priorità in \(\order{n \log n}\).
\end{document}