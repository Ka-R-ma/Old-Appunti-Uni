\documentclass{subfiles}
\begin{document}
In questa sezione si discuteranno gli algoritmi di ordinamento, che a meno di casi particolari, risultano essere pià efficenti.
Si tratta di algoritmi la cui complessità è \(\order{n \log n}\).

\subsubsection{Merge sort}
Ricevuto l'input, il merge sort procede applicando una strategia divide-and-conquer.
Si fa in modo che un l'ordinamento di n elementi, è ridotto ricorsivamente, all'ordinamento di due sotto-problemi di taglia n/2.
Giunti a problemi di taglia unitaria, questi vengono a risolti ordinando coppie distinte di elementi, ottenendo problemi di taglia 2, e cosi via.

Si osserva dunque che il costo dell'algoritmo è dato dalla seguente espressione ricorsiva.

\[
    C(n) = \begin{cases}
        1, \qquad                    & \text{se} n = 1 \\
        2 C(n/2) + \order{n}, \qquad & \text{se} n > 1
    \end{cases}\]

\noindent Applicando il \emph{Teorema Master}, si osserva che si rientra nel secondo caso, dunque il costo complessivo è \(\order{n \log n}\).
Di seguito si riporta l'implementazione algoritmica del Merge sort.

\subfile{../Figure/Figura 11 - Merge sort.tex}

\end{document}