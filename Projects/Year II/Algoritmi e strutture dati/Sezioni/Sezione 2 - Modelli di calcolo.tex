\documentclass{subfiles}
\begin{document}
Prima di poter parlare di algoritmi, è necessario stabile il \emph{modello di calcolo} con cui si intende risolvere un problema;
si pensi ad un modello di calcolo come un modello che descrive come l'output sia ottenuto in funzione dell'input.
\\ \\
Nella presente sezione si descriveranno due dei modelli più diffusi quali \emph{modello RAM} e \emph{macchina di Turing}.

\subsection{Modello RAM}
\subfile{Sotto sezioni/Sottosezione 2.1 - Modello RAM.tex}
\clearpage

\subsection{Complessità di un programma RAM}
\subfile{Sotto sezioni/Sottosezione 2.2 - Complessita di un programma RAM.tex}
\clearpage

\subsection{La macchina di Turing}
\subfile{Sotto sezioni/Sottosezione 2.3 - La macchina di Turing.tex}
\clearpage

\subsection{Relazione tra MT e RAM}
\subfile{Sotto sezioni/Sottosezione 2.4 - Relazione tra MT e RAM.tex}
\clearpage
\end{document}