\documentclass{subfiles}
\begin{document}
La teoria dell'NP-completezza nasce a partire dagli anni `70, diventando subito fondamento dell'informatica, sia con risvolti teorici che pratici.
\\ \\
Si è in precedenza visto come con la programmazione dinamica sia possibile realizzare algoritmi polinomiali, a partire da algoritmi esponenziali.
Nonostante ciò, nel corso degli anni si dimostrata l'esistenza di problemi che sono (almeno al momento) risolvibili in tempi esponenziali.
\\ \\
Lo svilupparsi dell'intelligenza artificiale, porto gli studiosi ad associare l'abilità di una macchina alla capacità della stessa di dimostrare teoremi:
è la nascite delle tecniche \emph{theorem proving}. Si osservi che quest'ultime cercano una soluzione esaustiva, pertanto sono esponenziali.
\\ \\
Molti studiosi, tra cui \emph{Eugene Lewler}, ma ben più importante \emph{Steven Cook}, cercarono di comprendere i limiti di queste tecniche.
In particolare Cook, giunse alla conclusione che qual'ora si fossero riuscite a trovare delle tecniche ottimali in logica proposizionale,
si sarebbero trovate delle tecniche per una classe molto più vasta di problemi.
\\ \\
L'effettiva nascita della teoria è da attribuire a \emph{Richard Karp}, questi riusci a trovare un nesso tra gli studi Cook e osservazioni di logica combinatoria,
gia note al tempo.
\\ \\
Si può riassumere in breve l'NP-completezza affermando che: esistono linguaggi tali che
\begin{enumerate}
    \item Il problema del riconoscimento prende tempo esponenziale su una TM classica.
    \item Non è noto un lower bound di tipo esponenziale.
    \item Contributo di Cook e Levin: Se uno di questi problemi è risolvibile in tempo polinomiale, lo sono anche gli altri.
    \item Questa classe contiene un nucleo “STORICO” (identi cato da Karp) che poi è stato ampliato considerevolmente.
\end{enumerate}
\clearpage

\subsection{MT non deterministica}
\subfile{Sotto sezioni/Sottosezione 7.1 - MT non deterministica.tex}

\subsection{Classi di problemi}
\subfile{Sotto sezioni/Sottosezione 7.2 - Classi di problemi.tex}

\subsection{Problema di soddisfacibilità di espressioni booleane}
\subfile{Sotto sezioni/Sottosezione 7.3 - SAT.tex}
\clearpage

\subsection{Clique problem}
\subfile{Sotto sezioni/Sottosezione 7.4 - Clique problem.tex}
\clearpage

\subsection{Vertex cover}
\subfile{Sotto sezioni/Sottosezione 7.5 - Vertex cover.tex}
\clearpage

\subsection{Isomorfismo tra sotto-grafi}
\subfile{Sotto sezioni/Sottosezione 7.6 - Isomorfismo tra sottografi.tex}

\subsection{P-space e EXP-space}
\subfile{Sotto sezioni/Sottosezione 7.7 - PSpace e EXPSpace.tex}
\end{document}