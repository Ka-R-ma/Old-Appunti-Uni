\documentclass{subfiles}
\begin{document}
Il problema dell'ordinamento (o sorting), può essere formulato come segue: dato un insieme di elementi \(a_{1}, a_{2}, \ldots, a_{n}\), sui quali è posta una relazione d'ordine totale,
si è interessati ad una permutazione \(\pi\), tali che \(a_{\pi(i)} \le a_{\pi(i + 1)}, \forall i < n\).

\subsection{Proprietà degli algoritmi di ordinamento}
\subfile{Sotto sezioni/Sottosezione 4.1 - Proprieta degli algoritmi di ordinamento.tex}

\subsection{Algoritmi quadratici}
\subfile{Sotto sezioni/Sottosezione 4.2 - Algoritmi quadratici.tex}
\clearpage

\subsection{Ordinamenti ottimi}
\subfile{Sotto sezioni/Sottosezione 4.3 - Ordinamenti ottimi.tex}

\subsection{Ordinamenti lineari}
\subfile{Sotto sezioni/Sottosezione 4.4 - Ordinamenti lineari.tex}
\end{document}