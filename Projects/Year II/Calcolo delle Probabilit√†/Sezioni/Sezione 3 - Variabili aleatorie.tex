\documentclass{subfiles}
\begin{document}
Spesso quando si effettua lo studio di un fenomeno aleatorio, si è molto più interessati a una qualche funzione degli esiti che agli stessi.\\
Si prenda ad esempio il lancio di una moneta: ci si domanda quante volte esca testa.
\\ \\
Quantità come quella dell'esempio si definiscono \emph{variabili aleatorie}.
Conseguentemente, poiché il valore assunto da tali variabili è dipendente dall'esito del fenomeno, è possibile attribuire a queste una probabilità.

\begin{Example*}
    Si supponga di lanciare tre monete. Sia \(\gamma\) il numero di volte in cui esce testa.
    Da ciò \(\gamma\) assume possibilmente i valori 0, 1, 2, 3, le cui rispettive probabilità sono
    \[\begin{aligned}
             & P \{\gamma = 0\} = P \{(C, C, C)\} = \tfrac{1}{8}                       \\
             & P \{\gamma = 1\} = P \{(C, C, T), (C, T, C), (T, C, C)\} = \tfrac{3}{8} \\
             & P \{\gamma = 2\} = P \{(C, T, T), (T, C, T), (T, T, C)\} = \tfrac{3}{8} \\
             & P \{\gamma = 3\} = P \{(T, T, T)\} = \tfrac{1}{8}                       \\
        \end{aligned}\]
    Ma la probabilità di un singolo evento è pari a 1, da cui
    \[
        P\left( \bigvee\limits_{i = 0}^{3}{\{\gamma = i\}} \right) = \sum\limits_{i = 0}^{3}{P\{\gamma = i\}} = 1
    \]
\end{Example*}
\end{document}