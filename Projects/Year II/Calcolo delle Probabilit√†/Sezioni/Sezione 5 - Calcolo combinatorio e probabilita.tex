\documentclass{subfiles}
\begin{document}
Prima di applicare alla probabilità il calcolo combinatorio, è bene fare un richiamo a concetti come permutazioni, disposizioni, combinazioni.
\\ \\
Si consideri il seguente esempio
\begin{Example*}
    siano \emph{A, B, C} località, siano \(p_{1}, p_{2}, p_{3}\) percorsi da \emph{A} a \emph{B}  e siano \(s_{1}, s_{2}\) percorsi da \emph{B} a \emph{C}.
    Ci si chiede: quanti sono i percorsi distinti da \(A \text{a} C\)?
    \\ \\
    Si osserva che per ognuno dei percorsi da \emph{A} a \emph{B}, si hanno due possibili scelte da \emph{B} a \emph{C}.
    Si hanno dunque le seguenti possibilità
    \[
        (p_{1}, s_{1}) \quad (p_{2}, s_{1}) \quad (p_{3}, s_{1}) \quad (p_{1}, s_{2}) \quad (p_{2}, s_{2}) \quad (p_{3}, s_{2})
    \]
    Quanto appena applicato è noto come \emph{principio di moltiplicazione}, il quale stabilisce che:
    dati \(n\) modi per effettuare una scelta, per ciascuno di essi si hanno \(m\) modi per farne un'altra, esistono allora \(n \cdot m\) diverse scelte.
\end{Example*}

\subsection{Disposizioni}
\subfile{Sotto Sezioni/Sottosezione 5.1 - Disposizioni.tex}

\subsection{Permutazioni}
\subfile{Sotto Sezioni/Sottosezione 5.2 - Permutazioni.tex}

\subsection{Combinazioni}
\subfile{Sotto Sezioni/Sottosezione 5.3 - Combinazioni.tex}

\subsection{Distribuzione binomiale}
\subfile{Sotto Sezioni/Sottosezione 5.4 - Distribuzione binomiale.tex}

\subsection{Distribuzione geometrica}
\subfile{Sotto Sezioni/Sottosezione 5.5 - Distribuzione geometrica.tex}

\subsection{Distribuzione ipergeometrica}
\subfile{Sotto Sezioni/Sottosezione 5.6 - Distribuzione Ipergeometrica.tex}

\subsection{Distribuzione di Poisson}
\subfile{Sotto Sezioni/Sottosezione 5.7 - Distribuzione di Poisson.tex}

\subsection{Distribuzione di Pascal e binomiale inversa}
\subfile{Sotto Sezioni/Sottosezione 5.8 - Distribuxione di Pascal e binomiale inversa.tex}

\subsection{Valore atteso e varianza di numeri aleatori discreti}
\subfile{Sotto Sezioni/Sottosezione 5.9 - Valore atteso e varianza di numeri aleatori discreti.tex}


\end{document}