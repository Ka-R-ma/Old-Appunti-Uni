\documentclass{subfiles}
\begin{document}
Si supponga di lanciare due dadi, si consideri che il primo dado dia 3, se si indica con \(E_{2}\) tale evento e con \(E_{1}\) l'evento \emph{somma dei dadi uguale a otto},
qual è la probabilità di \(E_{1}\)?
\\ \\
Considerando che tutti gli esiti siano equiprobabili, poiché dato \(E_{2}\), i possibili esiti
\[
    (3, 1) \quad (3, 2) \quad (3, 3) \quad (3, 4) \quad (3, 5) \quad (3, 6)
\]
segue che ciascuno di essi ha probabilità \(\tfrac{1}{6}\).
\\ \\
In generale \(\forall E_{1}, E_{2}\) tali che \(E_{1}\) è condizionato da \(E_{2}\)
\begin{equation}
    P(E_{1} \given E_{2}) = \frac{P(E_{1}E_{2})}{P(E_{2})} \qquad \text{tale che} P(E_{2}) > 0
\end{equation}
Generalizzando ulteriormente: dati \(E_{1} \cdots E_{n}\) eventi, si ha
\[
    P(E_{1}\cdots E_{n}) = P(E_{1})P(E_{1} \given E_{2})\cdots P(E_{n} \given E_{1}\cdots E_{n-1})
\]
la cui dimostrazione applicando l'Equazione risulta essere
\[
    P(E_{1})\frac{P(E_{1}E_{2}) \cdots P(E_{1} \cdots E_{n})}{P(E_{1}) \cdots P(E_{1} \cdots E_{n - 1})} = P(E_{1} \cdots E_{n})
\]

\subsection{Formula di Bayes}
\subfile{Sotto Sezioni/Sottosezione 4.1 - Formula di Bayes.tex}

\subsection{Probabilità condizionata e assiomi}
\subfile{Sotto Sezioni/Sottosezione 4.2 - Probabilita condizionata e assiomi.tex}

\subsection{Esercizi esemplificativi}
\subfile{Sotto Sezioni/Sottosezione 4.3 - Esercizi esemplificativi.tex}

\subsection{Indipendenza stocastica}
\subfile{Sotto Sezioni/Sottosezione 4.4 - Indipendenza stocastica.tex}
\end{document}