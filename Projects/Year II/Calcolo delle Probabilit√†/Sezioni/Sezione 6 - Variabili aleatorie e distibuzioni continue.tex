\documentclass{subfiles}
\begin{document}
\begin{Definition*}
    Sia \(X\) un numero aleatorio con densità \(f(x)\), si definisce valore atteso di \(X\) come
    \begin{equation}
        \mathbb{E}(X) = \int{-\infty}{+\infty}{x f(x)}{x}
    \end{equation}
\end{Definition*}
\begin{Example*}
    Sia \(X\) un numero aleatorio con densità \(f(x)\) descritta come a seguito
    \[\begin{cases}
            2x,  \quad 0 \le x \le 1 \\
            0, \quad \text{altrimenti}
        \end{cases}\]
    Applicando l'Equazione \eqref{eq:11},  poiché per \(x > 1 \land x < 0, f(x) = 0\), segue
    \[
        \mathbb{E}(X) = \int{0}{1}{2x^{2}}{x} = \frac{2}{3}
    \]

    \noindent
    La varianza di un numero aleatorio continuo è definita analogamente ai numeri aleatori discreti: cioè
    \[
        \Var(X) = \mathbb{E}(X^{2}) - \mathbb{E}(X)^{2}
    \]
\end{Example*}

\subsection{Distribuzione uniforme}
\subfile{Sotto Sezioni/Sottosezione 6.1 - Distribuzione uniforme.tex}

\subsection{Distribuzione esponenziale}
\subfile{Sotto Sezioni/Sottosezione 6.2 - Distribuzione esponenziale.tex}

\subsection{Funzione e distribuzioni Gamma}
\subfile{Sotto Sezioni/Sottosezione 6.3 - Funzione e distribuzione Gamma.tex}
\end{document}