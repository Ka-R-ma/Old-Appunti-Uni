\documentclass{subfiles}
\begin{document}
Due concetti fondamentali del calcolo probabilistico sono quelli di \emph{valore atteso} e \emph{varianza}.
\begin{Definition*}
    Dato \(X\) un numero aleatorio che per ogni valore assunto ha una certa probabilità \(p_{i}\), si definisce valore atteso \(\mathbb{E}(X)\) di X, la media pesata dei valori assunti da X,
    ciascuno con peso \(p_{i}\). Cioè
    \[
        \mathbb{E}(X) = \sum\limits_{i = 0}^{+ \infty}{x_{i}}{p_{i}}
    \]
\end{Definition*}

\begin{Definition*}
    Dato \(X\) un numero aleatorio di media \(\mu\), si definisce varianza
    \[\Var(X) = \mathbb{E}[(X - \mu)^{2}]\]
\end{Definition*}
Più spesso la varianza è definita come
\[
    \Var(X) = \mathbb{E}(X^{2}) - \mathbb{E}(X)^{2}
\]
\end{document}