\documentclass{subfiles}
\begin{document}
Il calcolo delle probabilità si fonda su alcune nozioni che, da ora in avanti, si daranno per assodate.
Tali nozioni sono tre assiomi che di seguito saranno trattati.

\begin{Axiom}\label{Axiom:1}
    Dato un evento \emph{E}, la probabilità che questo si verifichi è sempre compresa tra 0 e 1. Cioè
    \[
        0 \le P(E) \le 1
    \]
\end{Axiom}

\begin{Axiom}\label{Axiom:2}
    Considerato uno spazio campionato \(\Omega\), la probabilità che questi si verifichi è pari a 1. Cioè
    \[
        P(\Omega) = 1
    \]
\end{Axiom}

\begin{Axiom}\label{Axiom:3}
    Dati un numero \emph{n} di eventi, a due a due incompatibili, la probabilità che almeno uno tra gli n eventi si verifichi è pari alla somma delle rispettive probabilità. Cioè
    \[
        P \left( \bigvee\limits_{i = 1}^{\infty}{E_{i}} \right) = \sum\limits_{i = 1}^{\infty}{P(E_{i})}
    \]
\end{Axiom}
\noindent
Si analizzano ora alcune proprietà derivate dagli assiomi precedentemente introdotti, spesso utili alla risoluzioni di problemi probabilistici.
\begin{Proposition}
    Dati due eventi disgiunti \(E_{1} \text{,} E_{2}\), tali che \(E_{1} \implies E_{2}\), la probabilità di \(E_{1}\) è minore o uguale a quella di \(E_{2}\). Cioè
    \[
        P(E_{1}) \le P(E_{2})
    \]

    \begin{Proof*}
        se \(E_{1} \implies E_{2}\), allora \(E_{2} = E_{1} \lor E_{1}^{C}E_{2}\).
        Ma \(E_{1} \text{e} E_{1}^{C}E_{2}\) sono disgiunti, da cui per l'Assioma (\ref{Axiom:3}) segue
        \[
            P(E_{2}) = P(E_{1}) + P(E_{1}^{C}E_{2})
        \]
        ma ciò implica, poiché \(P(E_{1}^{C}E_{2}) \ge 0\), che
        \[
            P(E_{1}) \le P(E_{2})
        \]
    \end{Proof*}
\end{Proposition}

\begin{Proposition}
    Dati due eventi disgiunti \(E_{1} \text{,} E_{2}\), tali che \(P(E_{1} \lor E_{2}) = 1\), segue per gli Assiomi (\ref{Axiom:2}), (\ref{Axiom:3}) che
    \[
        P(E_{1} \lor E_{2}) = P(E_{1}) + P(E_{2}) = 1
    \]
    ma da ciò
    \[
        P(E_{1}) = 1 - P(E_{2})
    \]
\end{Proposition}

\begin{Proposition}\label{Proposition:1}
    Dati due eventi, la probabilità della loro unione è pari alla somma delle probabilità dei singoli eventi meno quella della loro intersezione. Cioè
    \[
        P(E_{1} \lor E_{2}) = P(E_{1}) + P(E_{2}) - P(E_{1}E_{2})
    \]

    \begin{Proof*}
        si nota che \(E_{1} \lor E_{2} = E_{1} \lor E_{1}^{C}E_{2}\), da cui per l'Assioma (\ref{Axiom:3}) segue
        \[
            P(E_{1} \lor E_{2}) = P(E_{1}) + P(E_{1}^{C}E_{2})
        \]
        ma a sua volta \(E_{2} = E_{1}E_{2} \lor E_{1}^{C}E_{2}\), da cui sempre per l'Assioma (\ref{Axiom:3}) segue
        \[
            P(E_{2}) = (E_{1}E_{2}) + P(E_{1}^{C}E_{2})
        \]
        da cui sostituendo si ottiene quanto si voleva dimostrare.
    \end{Proof*}
\end{Proposition}
\begin{Note*}
    La proposizione (\ref{Proposition:1}) è generalizzabile ad \emph{n} eventi applicando il principio dei cassetti.
\end{Note*}
\end{document}