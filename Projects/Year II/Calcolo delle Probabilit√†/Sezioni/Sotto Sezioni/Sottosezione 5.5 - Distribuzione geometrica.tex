\documentclass{subfiles}
\begin{document}
\begin{Definition*}
    un certo numero aleatorio \(X \in \N\) si dice avere \emph{distribuzione geometrica} di parametro \(p \in (0, 1)\), in simboli \(X \sim G(p)\), se
    \[
        P(X = n) = (1 - p)^{n - 1} p, n \in \N
    \]
\end{Definition*}
\noindent
Sia \(E_{1},E_{2},\dots, E_{n}\) una successione di eventi, tali che \(P(E_{i}) = p \in (0, 1), q = 1 - p\).
\[
    X = \text{"Numero aleatorio di prove fino al primo successo"}
\]
ossia
\[
    X = \inf{n \such n \abs{E_{n}} = 1}
\]
Si ha dunque
\[\begin{aligned}
        (X = 1) & = E_{1} \implies P(X = 1) = P(E_{1}) = p                                                                                     \\
        (X = 2) & = E_{1}^{C}E_{2} \implies P(X = 2) = P(E_{1}^{C}E_{2}) = qp                                                                  \\
        (X = n) & = E_{1}^{C}E_{2}^{C}\cdots E_{n-1}^{C} E_{n} \implies P(X = n) = P(E_{1}^{C}E_{2}^{C}\cdots E_{n-1}^{C} E_{n}) = q^{n - 1} p \\
    \end{aligned}\]
In generale
\[
    P(X = n) = q^{n - 1}p = (1 - p)^{n - 1}p, n \in \N
\]
\begin{Example*}
    sia \(X > m\), ci si chiede \(P(X > n + m \given X > m)\).
    \\
    Si osserva che
    \[
        (X > n + m) \impliedby (X > m)
    \]
    segue quindi
    \[
        (X > n + m) \cap (X > m) = (X > n + m)
    \]
    pertanto
    \begin{equation}
        \begin{aligned}
            P(X > n + m \given X > m) & = \frac{P(X > n + m)}{P(X > m)}              \\
                                      & = \frac{q^{n + m}}{q^{n}} = q^{n} = P(X > n)
        \end{aligned}
    \end{equation}
\end{Example*}
\noindent
Quanto appena affermato in Equazione \eqref{eq:8}, è noto come \emph{proprietà di assenza di memoria}, per la quale in generale
\[
    P(X > n + m \given X > m) = P(X > n), \forall n, m \in \N
\]
\clearpage

\begin{Example*}
    Una moneta viene lanciata \(X\) volte.
    Sia
    \[
        X = \text{"Numero aleatorio di lanci utili ad ottenere testa per la prima volta"}
    \]
    e gli eventi
    \[\begin{gathered}
            A = (X \ pari) \\
            B = (X \ dispari) \\
        \end{gathered}\]
    Si stabilisca se
    \[\begin{gathered}
            P(A) = P(B) \\
            P(A) < P(B) \\
            P(A) > P(B) \\
        \end{gathered}\]
    Supponendo inoltre di non aver ottenuto testa nei primi 1000 lanci, quanto vale la probabilità \(\alpha\) che non esca testa nei successivi 10?
    Supposto invece di aver ottenuto testa per \(X \ pari\), quanto vale la probabilità condizionata \(\beta\) di ottenere testa al secondo lancio.

    \begin{enumerate}
        \item Si parta col considerare \(P(A)\), segue
              \[\begin{aligned}
                      P(A) & = P(X \in 2n, n \in \N)                                                                                                                                                                                         \\
                           & = P\left(\bigvee\limits_{n = 1}^{+ \infty} (X = 2n)\right)                                                                                                                                                      \\
                           & = \sum\limits_{n = 1}^{+ \infty}{P(X = 2n)}  = \sum\limits_{n = 1}^{+ \infty}{q^{2n - 1}p} = \sum\limits_{n = 1}^{+ \infty}{\left(\frac{1}{4}\right)^{n}} = \frac{1}{4} \frac{1}{1 - \frac{1}{4}} = \frac{1}{3}
                  \end{aligned}\]
              Si consideri ora \(P(B)\), segue
              \[\begin{aligned}
                      P(B) & = P(X \in 2n - 1, n\in \N)                                                                                                                          \\
                           & = P\left(\bigvee\limits_{n = 1}^{+ \infty} (X = 2n - 1)\right)                                                                                      \\
                           & = \sum\limits_{n = 1}^{+ \infty}{P(X = 2n - 1)} = \sum\limits_{n = 1}^{+ \infty}{q^{2n - 2}p} = \frac{1}{2} \frac{1}{1 - \frac{1}{4}} = \frac{2}{3}
                  \end{aligned}\]
              Segue ovviamente
              \[
                  P(A) < P(B)
              \]

        \item Si osserva che \(\alpha = P(X > 1010 \given X > 1000)\), ma poiché della forma \(P(X > m + n  \given X > m)\), segue che
              \[\begin{aligned}
                      \alpha = P(X > 1010 \given X > 1000) & = \frac{P(X > 1010)}{P(X > 1000)}                                                                         \\
                                                           & = \frac{\left(\frac{1}{2}\right)^{1010}}{\left(\frac{1}{2}\right)^{1000}} = \left(\frac{1}{2}\right)^{10}
                  \end{aligned}\]

        \item Si osserva che \(\beta = P(X = 2 \given X \ pari)\), segue
              \[\begin{aligned}
                      \beta = P(X = 2 \given X \ pari) & = \frac{P((X = 2) \cap (X \ pari))}{P(X \ pari)} \\
                                                       & = \frac{P(X = 2)}{P(X \ pari)}
                  \end{aligned}\]
              ma se si considera \(E_{i}\) l'evento "esce testa all'i-esimo lancio", si ha \((X = 2) = E_{1}^{C}E_{2}\), si ha
              \[
                  P(X = 2) = P(E_{1}^{C}E_{2}) = P(E_{1}^{C})P(E_{2})
              \]
              inoltre poiché \(\forall i \in \Set{1, 2, \dots, n}, P(E_{i}) = \tfrac{1}{2}\), segue
              \[
                  \beta = \frac{P(X = 2)}{P(X \ pari)} = \frac{P(E_{1}^{C})P(E_{2})}{P(X \ pari)} = \frac{\frac{1}{2} \frac{1}{2}}{\frac{1}{3}} = \frac{3}{4}
              \]
    \end{enumerate}
\end{Example*}
\clearpage
\end{document}