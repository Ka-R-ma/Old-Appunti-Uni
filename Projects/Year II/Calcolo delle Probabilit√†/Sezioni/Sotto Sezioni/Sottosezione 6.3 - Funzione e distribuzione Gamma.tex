\documentclass{subfiles}
\begin{document}
\subsubsection*{Funzione Gamma}
La funzione Gamma è definita come
\[
    \Gamma(\alpha) = \int{0}{\infty}{e^{-y}y^{\alpha - 1}}{y}
\]
Se opportunamente integrata si ha, per \(\alpha \in \Z\), che
\[
    \Gamma(\alpha) = (\alpha - 1)!
\]

\subsubsection*{Distribuzione Gamma}
Sia \(X\) un numero aleatorio. Questa si dice avere distribuzione Gamma, di parametri \((\alpha, \lambda)\), se
\[f(x) = \begin{cases}
        \frac{\lambda e^{-\lambda x} (\lambda x)^{\alpha - 1}}{\Gamma(\alpha)}, x \ge 0 \\
        0, \text{altrimenti}
    \end{cases}\]
Inoltre se \(\alpha \in \N\), la distribuzione Gamma è utilizzata per quantizzare il tempo di attesa prima che si verifichino \(n\) eventi.

Sia \(T_{n}\) il tempo necessario al verificarsi dell'n-esimo evento. Segue
\[
    P(T_{n} \le x) \iff P(N(x) \ge n)
\]
ove \(N(x)\) è il numero di eventi verificatesi in \([0, x]\).\\

Da cui
\[\begin{aligned}
        P(T_{n} \le x) & = P(N(x) \ge x)                                                                   \\
                       & = \sum\limits_{i = n}^{\infty}{P(N(x) = i)}                                       \\
                       & = \sum\limits_{i = n}^{\infty}{\frac{e^{-\lambda x} (\lambda x)^{\lambda i}}{i!}} \\
    \end{aligned}\]
da cui derivando per \emph{x}, si dimostra che
\[
    P(T_{n} \le x) = \frac{\lambda e^{-\lambda x} (\lambda x)^{\alpha - 1}}{\Gamma(\alpha)}
\]
\clearpage
\begin{Example*}
    Sia \(T_{n}\) il tempo di attesa utile all'arrivo dell'n-simo cliente, con distribuzione Gamma \(\lambda = 2\).
    Si calcoli
    \begin{enumerate}
        \item \(T_{1} > 10\);
        \item \(T_{5} \le 30\);
        \item \(T_{3} > 10\).
    \end{enumerate}

    \begin{enumerate}
        \item Si osserva che
              \[\begin{aligned}
                      P(T_{1} < 10) & = \int{0}{\infty}{\frac{2x e^{-2x}}{0!}}{x} \\
                                    & = \int{0}{\infty}{2x e^{-2x}}{x} = e^{-20}
                  \end{aligned}\]

        \item Ricordando che
              \[
                  \int{\frac{2x e^{-2x}}{n - 1}!}{x} = 1
              \]
              segue
              \[
                  P(T_{5} \le 30) = 1 - P(T_{5} > 30) = P(N_{30} \ge 5)
              \]
              da cui
              \[
                  P(T_{5} \le 30) = \sum\limits_{i = 0}^{4} \frac{(2 \cdot 30)^{i}}{i!}e^{-60}
              \]

        \item Applicando un ragionamento analogo al punto precedente
              \[\begin{aligned}
                      P(T_{3} > 10) & = P(N_{10} < 3)                                                             \\
                                    & = \sum\limits_{i = 0}^{2}{\frac{(2 \cdot 30)^{i}}{i!} e^{-20}} = 221e^{-20}
                  \end{aligned}\]
    \end{enumerate}
\end{Example*}
\end{document}