\documentclass{subfiles}
\begin{document}
Gli eventi sono soggetti ad alcune proprietà, quali
\begin{itemize}
    \item negazione;
    \item implicazione;
    \item uguaglianza;
    \item unione;
    \item intersezione.
\end{itemize}

\subsubsection{Negazione}
Dato un evento E, l'evento negato
\[E^{C} \begin{cases}
        \text{vero, se E è falso} \\
        \text{falso, se E è vero} \\
    \end{cases}\]

\subsubsection{Implicazione}
Dati due eventi \(E_{1} \text{,} E_{2}\), si dirà che \(E_{1}\) implica \(E_{2}\) se
\[\begin{aligned}
        \abs{E_{1}} & = 1 \implies \abs{E_{2}} = 1    \\
        \abs{E_{1}} & = 0 \implies \abs{E_{2}} = 0, 1 \\
    \end{aligned}\]

\subsubsection{Uguaglianza}
Dati due eventi \(E_{1} \text{,} E_{2}\), si dice che \(E_{1} = E_{2}\) se ogni esito di \(E_{1}\) è verificato in \(E_{2}\), e viceversa.

\subsubsection{Unione}
Dati due eventi \(E_{1} \text{,} E_{2}\), si definisce l'evento \(E_{3} = E_{1} \lor E_{2}\) che soddisfa gli esiti di \(E_{1} \text{o} E_{2}\), unione.
\\ \\
In particolare
\begin{itemize}
    \item \(E \lor E^{C} = \Omega\) \\
    \item \(E \lor \varnothing = E\) \\
    \item \(E \lor E = E\) \\
    \item \(E \lor \Omega = \Omega\)
\end{itemize}

\subsubsection{Intersezione}
Dati due eventi \(E_{1} \text{,} E_{2}\), si definisce l'evento \(E_{3} = E_{1} \land E_{2}\) che soddisfa gli esiti presenti sia in \(E_{1} \text{che in } E_{2}\), intersezione.

\begin{Note*}
    Se \(E_{1} \land E_{2} = \varnothing\), i due eventi si dicono \emph{incompatibili} (o disgiunti).
\end{Note*}

\clearpage
\end{document}