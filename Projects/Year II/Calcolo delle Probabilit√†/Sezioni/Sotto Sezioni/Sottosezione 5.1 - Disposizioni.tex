\documentclass{subfiles}
\begin{document}
\begin{Definition*}
    dato \(S = \Set{a_{1}, a_{2}, \dots, a_{n}}\) insieme di \emph{n} elementi, si definiscono disposizioni \(D_{n, k}\), per un certo \(k\) intero positivo,
    i sottoinsiemi di \emph{k} elementi distinti che si possono formare da \emph{S}.
\end{Definition*}

Si consideri il seguente esempio in due casi
\begin{enumerate}
    \item si effettuano restituzioni;
    \item non si effettuano restituzioni.
\end{enumerate}

\begin{Example*}
    da un urna contenente \(n\) palline, se ne estraggono \(r \leq n\). Quante sono le possibili disposizioni?
    \begin{enumerate}
        \item Ogni pallina estratta è reinserita, segue
              \begin{equation}
                  D_{n, r} = n^{r}
              \end{equation}

        \item Una volta estratta la pallina, questa non viene reinserita, segue
              \begin{equation}
                  D_{n, r} = n(n - 1)(n- 2)\cdots(n-r + 1)
              \end{equation}
    \end{enumerate}
\end{Example*}
\end{document}