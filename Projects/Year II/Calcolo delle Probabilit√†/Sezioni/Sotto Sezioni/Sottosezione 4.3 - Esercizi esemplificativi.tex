\documentclass{subfiles}
\begin{document}
\begin{Exercise*}
    siano date due urne \(U \text{e} V\). Si sceglie casualmente tra le due, e da questa si effettuano estrazioni con ripetizione.
    \[
        U = \Set{b, b, n} \qquad V = \Set{n, n, b}
    \]
    Considerando pertanto
    \[\begin{aligned}
            H     & = \text{"l'urna scelta è U"}       \\
            E_{i} & = \text{"l'i-esima pallina è `b'"} \\
        \end{aligned}\]
    segue
    \[
        P(H) = P(H^{C}) = \frac{1}{2}
    \]
    inoltre
    \[\begin{aligned}
            P(E_{i}) = P(E_{i} \cup \Omega) & = P(E_{i} H) + P(E_{i} H^{C})                           \\
                                            & = P(E_{i} \given H)P(H) + P(E_{i} \given H^{C})P(H^{C})
        \end{aligned}\]

    Si calcolino
    \begin{itemize}
        \item \(P(H \given E_{1})\): cioè, sapendo che la prima pallina estratta è bianca, qual è la probabilità che l'urna scelta sia U.
        \item \(P(H \given E_{1}E_{2}^{C}E_{3}^{C})\): cioè, sapendo che le palline estratte sono nell'ordine \(b, n, n\), qual è la probabilità che l'urna scelta sia U.
    \end{itemize}

    \begin{Solution*}
        per la risoluzione degli esercizi si applicherà Bayes.
        \begin{enumerate}
            \item \[
                      P(H \given E_{1}) = \frac{P(E_{1}H)}{P(E_{1})} = \frac{P(E_{1} \given H)P(H)}{P(E_{1} \given H) + P(E_{1} \given H^{C})} = \frac{2}{3}
                  \]

            \item \[
                      P(H \given E_{1} E_{2}^{C}E_{3}^{C}) = \frac{P(E_{1}E_{2}^{C}E_{3}^{C}H)}{P(E_{1}E_{2}^{C}E_{3}^{C})} = \frac{P(E_{1}E_{2}^{C}E_{3}^{C}H)}{P(E_{1}E_{2}^{C}E_{3}^{C}H) + P(E_{1}\overline{E_{2}E_{3}H})}
                  \]
                  ma
                  \[\begin{gathered}
                          P(E_{1}E_{2}^{C}E_{3}^{C}H) = P(\overline{E_{3}} \given \overline{E_{2}}E_{1}H)P(\overline{E_{2}} \given E_{1}H)P(E_{1} \given H)P(H) = \frac{1}{27} \\
                          P(E_{1}E_{2}^{C}E_{3}^{C}H^{C}) = P(\overline{E_{3}} \given \overline{E_{2}}E_{1}H^{C})P(\overline{E_{2}} \given E_{1}H^{C})P(E_{1} \given H^{C})P(H^{C}) = \frac{2}{27} \\
                      \end{gathered}\]
                  da cui
                  \[
                      P(H \given E_{1} E_{2}^{C}E_{3}^{C}) = \frac{1}{3}
                  \]
        \end{enumerate}
    \end{Solution*}
\end{Exercise*}

\begin{Exercise*}
    siano date tre urne \(X, Y, Z\). Si sceglie una delle tre senza conoscerne il contenuto.
    \[
        X = \Set{Premio} \quad Y = \Set{Capra} \quad Z = \Set{Capra}
    \]
    Supponendo si scelga inconsciamente l'urna X, ci si chiede se convenga cambiare la propria scelta se, una volta rivelato il contenuto dell'urna Y questa è una capra.

    \begin{Solution*}
        si ha che
        \[
            \forall i \in \Set{1, 2, 3} \quad P(A_{i}) = \frac{1}{3}
        \]
        inoltre
        \[
            P(A_{1} \given A^{C}) = \frac{P(A_{1}A^{C})}{P(A^{C})} = \frac{P(A_{1})}{P(A_{1}) + P(A_{3})} = \frac{1}{2}
        \]
        cioè la scelta apparentemente sembra indifferente. Ma è davvero così?
        \\ \\
        Se si definisce \(I = \text{"la scatola due contiene una capra"}\) si hanno due scenari
        \[\begin{gathered}
                I \implies A_{2} \text{? SI} \\
                I \impliedby A_{2} \text{? NO}
            \end{gathered}\]
        Pertanto è più corretto calcolare la probabilità condizionata ad \(I\), segue
        \[
            P(A_{i} \given I) = \frac{P(I \given A_{1})P(A_{1})}{\sum\limits_{j = 1}^{3}{P(I \given A_{j})P(A_{j})}}
        \]
        ma
        \[\begin{gathered}
                P(I \given A_{1}) = p \\
                P(I \given A_{2}) = 0 \\
                P(I \given A_{3}) = 1 \\
            \end{gathered}\]
        segue dunque
        \[
            P(A_{1} \given I) = \frac{p\frac{1}{3}}{p\frac{1}{3} + \frac{1}{3}} = \frac{p}{p + 1} \leq \frac{1}{p + 1} = P(A_{3} \given I)
        \]
        con
        \[\begin{gathered}
                P(A_{1} \given I) = P(A_{3} \given I) \text{se} p = 1 \\
                P(A_{1} \given I) < P(A_{3} \given I) \text{se} p < 1\\
            \end{gathered}\]
    \end{Solution*}
\end{Exercise*}
\end{document}
\clearpage