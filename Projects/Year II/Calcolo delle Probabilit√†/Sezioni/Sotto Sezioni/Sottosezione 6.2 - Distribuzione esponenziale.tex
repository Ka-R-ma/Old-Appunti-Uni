\documentclass{subfiles}
\begin{document}
Sia \(X\) un numero aleatorio. Questi si dice avere distribuzione esponenziale, di parametro \(\lambda > 0\), se la densità \(f(x)\) di \(X\) è definita come segue.
\[f(x) = \begin{cases}
        \lambda e^{-\lambda x}, \text{se} x \ge 0 \\
        0, \text{altrimenti}
    \end{cases}\]

Considerando invece la distribuzione \(F(x)\), si ha che
\[\begin{aligned}
        F(x) & = P(X \le x)                                                              \\
             & = \int{0}{x}{\lambda e^{\lambda x}}{t} = 1 - e^{\lambda x}, \quad x \ge 0
    \end{aligned}\]

\begin{Note*}
    Da opportuni calcoli si può dimostrare che \(\lambda = \tfrac{1}{\mathbb{E}(X)}\).
    Si dimostra inoltre che se \(X \sim Exp(\lambda)\), questi gode di assenza di memoria, e vale il viceversa.
\end{Note*}

\subsubsection{Funzione di sopravvivenza}
Sia \(X\)un numero aleatorio continuo; si definisce \(S(X)\) la probabilità che \(X > x, \forall x \in \R\). Cioè
\[
    S(X) = P(X > x), \qquad \forall x \in \R
\]
Da ciò segue
\[
    S(X) = \int{x}{\infty}{f(t)}{t}
\]
Inoltre se \(X \sim Exp(\lambda)\), segue
\[S(X) = \begin{cases}
        1, \text{se} x \le 0  \\
        e^{-\lambda x}, x > 0 \\
    \end{cases}\]
cioè
\[
    S(X) = 1 - F(X)
\]
\clearpage
\end{document}