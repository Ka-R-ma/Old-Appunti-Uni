\documentclass[12pt]{article}

\usepackage{../../../../Admin/Utils/CustomMacros}

\begin{document}


\newgeometry{textwidth = 355pt, marginparwidth = 126pt}
\subfile{../Sezioni/TitlePage.tex}

\tableofcontents
\restoregeometry

\section{Introduzione e sintassi base}
Java è un linguaggio semplice, orientato ad oggetti, distribuito, interpretato, robusto, sicuro, indipendente dall'architettura,
portabile, performante, dinamico e multi-threaded.
\noindent
Concentrandoci sull'orientamento ad oggetti, esso è un approccio basato su classi contenenti le informazioni nei dati utilizzabili
tramite i metodi e servono a creare oggetti. Inoltre ciò permette il riutilizzo nel codice.

I programmi in Java sono divisi in classi descritte in uno o più file di testo con estensione ``.java''.

\begin{Note*} \label{Note:1}
    Un singolo file può avere più classi ma al più una classe può essere marcata come \lstinline[language = java]{public}.
\end{Note*}
\begin{Note*}\label{Note:2}
    Il punto di accesso di un programma Java è il metodo \textbf{main} contenuto nella classe principale.
\end{Note*}


\subsection{Commenti}
\subfile{../Sezioni/Sotto Sezioni/Sottosezione 1.1 - Commenti.tex}

\subsection{Tipi primitivi}
\subfile{../Sezioni/Sotto Sezioni/Sottosezione 1.2 - TipiPrimitivi.tex}

\subsection{Operatori}
\subfile{../Sezioni/Sotto Sezioni/Sottosezione 1.3 - Operatori.tex}

\end{document}