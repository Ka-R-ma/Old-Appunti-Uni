\documentclass{subfiles}
\begin{document}
    I file .java devono essere denominati con la classe principale definita al loro interno. Le classi sono modelli per creare 
    e manipolare oggetti. Ogni oggetto ha:
    \begin{itemize}
        \item un'identità (quindi codice identificativo e memoria);
        \item uno stato (definito da \textbf{attributi});
        \item un comportamento (definito da operazioni dette \textbf{metodi} che ne cambiano lo stato).
    \end{itemize}
    Ogni oggetto è istanza di una classe. Per descrivere gli stati possibili di un oggetto, una classe utilizza variabili 
    dette  attributi mentre per i comportamenti si definiscono i metodi.\\
    La sintassi di una classe è:
    %todo:
    %<modificatore-visibilità> <NomeClasse>{
    %   <dichiarazione_attributi>
    %   <dichiarazione_metodi>
    %}
    \\La sintassi di un metodo, invece, è molto simile a quella di una funzione nel linguaggio C:

    \begin{center}
        \begin{lstlisting}[language = java]

            <modificatoreVisibilita> <tipoRitorno> <nomeMetodo>(<parametri>){
                <corpo>
            }

        \end{lstlisting}
    \end{center}
    
    I principali modificatori di visibilità sono \lstinline[language = java]{public} e \lstinline[language = java]{private} ed è
    assegnabile sia ai metodi che agli attributi. Inoltre si possono definire degli attributi costanti mediante la parole chiave
    \lstinline[language = java]{final}.
    \\Le classi definiscono al loro interno un \textbf{costruttore}ovvero un metodo che definisce lo stato iniziale degli oggetti.
    
    Il costruttore di default, ovvero senza parametri, è definito dal compilatore se e solo se non ci sono altri costruttori.

    \begin{Example*}
        \begin{center}
           \begin{lstlisting}[language = java]

               public class Serbatoio{
                    private int livello;
                    private final int   MAX_VALUE = 1000;
                
                    public Serbatoio(){
                        livello = 10;    
                    }
                    
                    public void rifornisci(int j){
                        livello += j;   
                    }
                    
                    public void getLivello(){
                        return livello;   
                    }
               }
           \end{lstlisting}
        \end{center}
   \end{Example*}
    
   \begin{Note*}
    Utilizzare il modificatore di visibilità private per metodi, attributi o classi non permette di accedere ad essi 
    dall'esterno della classe.
   \end{Note*}

   

   \subsection{Variabili e valori di default}
   \subfile{Sotto Sezioni/Sottosezione 2.1 - Var.tex}

   \subsection{Metodo Main}
   \subfile{Sotto Sezioni/Sottosezione 2.2 - Main.tex}

   \subsection{Compilazione}
   \subfile{Sotto Sezioni/Sottosezione 2.3 - Compilazione.tex}

   \subsection{Garbage Collector}
   \subfile{Sotto Sezioni/Sottosezione 2.4 - GarbageCollector.tex}


\end{document}