\documentclass{subfiles}
\begin{document}
    Come per gli altri linguaggi di programmazione, Java definisce dei tipi di dati:
    \begin{itemize}
        \item \lstinline[language = Java]{byte}: intero 8 bit con segno;
        \item \lstinline[language = Java]{short}: intero 16 bit con segno;
        \item \lstinline[language = Java]{int}: intero 32 bit con segno;
        \item \lstinline[language = Java]{long}:intero 64 bit con segno;
        \item \lstinline[language = Java]{float}: floating-point 32 bit con segno;
        \item \lstinline[language = Java]{double}: floating-point 64 bit con segno;
        \item \lstinline[language = Java]{boolean}: due elementi (true, false) 8 bit;
        \item \lstinline[language = Java]{char}: carattere 16  it (Unicode).
    \end{itemize}

    \subsubsection{Dichiarazione}
    Le dichiarazioni avvengono esattamente come nel linguaggio C, in esse quindi si può fare anche un'assegnazione diretta
    e/o anche dichiarazioni multiple.

    \begin{Note*}
        Una dichiarazione non può iniziare con una cifra né si può utilizzare la parola riservata \lstinline[language = java]{class}.
    \end{Note*}

    \subsubsection{Letterali}
    Le dichiarazioni dirette risultano simili al linguaggio C o in alcuni casi più specifiche. Vediamo i casi:
    \begin{itemize}
        \item \textbf{caratteri}: in Unicode o dichiarazione tra apici;
        \item \textbf{caratteri di controllo}:   
        \begin{itemize}
            \item \textbackslash n: avanazamento di riga;
            \item \textbackslash t: tabulazione caratteri.
        \end{itemize} 
        \item \textbf{stringhe}: racchiuse tra virgolette;
        \item \textbf{interi}:
        \begin{itemize}
            \item int: suffisso assente;
            \item long: suffisso \textbf{l} o \textbf{L}.
        \end{itemize}
        Inoltre si può definire la base di un intero:
        \begin{itemize}
            \item base decimale: inizia per 0 o un numero che inizia per 1...9 \(\implies\) 0 \ 129L;
            \item base ottale: inizia con 0 \(\implies\) 0777;
            \item base esadecimale: inizia con 0x \(\implies\) 0x127;
        \end{itemize}
        \item \textbf{reali}: devono necessariamente avere una cifra nell parte intera o frazionaria  e almeno un elemento tra punto decimale, 
        esponente o suffisso. Si distinguono in: 
        \begin{itemize}
            \item \textbf{float}: suffisso \textbf{f} o \textbf{F};
            \item \textbf{double}: suffisso assente o \textbf{d} o \textbf{D};
        \end{itemize}
    
    \end{itemize}
    \clearpage
\end{document}