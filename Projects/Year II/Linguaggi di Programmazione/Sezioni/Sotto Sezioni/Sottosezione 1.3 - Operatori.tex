\documentclass{subfile}
\begin{document}
    Come nel linguaggio C esistono operatori utilizzabili per effettuare confronti o operazioni tra dati. Si distinguono in:
    1\begin{itemize}
        \item \textbf{operatori di relazione}: utilizzabili per tipi numerici e booleani sono: 
        \begin{itemize}
            \item \lstinline[language = java]{==};
            \item  \lstinline[language = java]{!=}.
        \end{itemize}
        \item \textbf{operatore di complemento logico}: utilizzabile per complementare il risultato di un'espressione booleana: 
        \lstinline[language = java]{!};
        \item \textbf{operatori logici}: vengono utilizzati per le espressioni booleane, si dividono in:
        \begin{itemize}
            \item \textbf{AND}: indicato con \lstinline[language = java]{&} o se in una condizione con \lstinline[language = java]{&&};
            \item \textbf{OR}: indicato con \lstinline[language = java]{|} o se in una condizione con \lstinline[language = java]{||};
            \item \textbf{OR esclusivo}: indicato con \lstinline[language = java]{^}.
        \end{itemize}
        \item \textbf{confronto numerico}: per confrontare due tipi numerici si usano:
        \begin{itemize}
            \item \lstinline[language = java]{>};
            \item \lstinline[language = java]{>=};
            \item \lstinline[language = java]{<};
            \item \lstinline[language = java]{<=}.
        \end{itemize}
        Si può anche usare \lstinline[language = java]{==};
        \item \textbf{incremento e decremento postfisso e prefisso}: come nel linguaggio C si può incrementare o decrementare un valore di 1
        senza dovere esprimere l'istruzione completa. Questa opereazione può essere espressa prima(prefissa) o dopo(postfissa) la variabile
        in base alle necessità. Queste operazioni vengono effettuate tramite ++ e/o --;
        \item \textbf{complemento bit a bit}: utilizzabile per ottenere il complemento bit a bit di un tipo numerico:
        \lstinline[language = java]{~};
        \item \textbf{operatore condizionale}: anche detto operatore ternario, funziona come nel linguaggio C, se la condizione è 
        soddisfatta viene effettuata la prima istruzione altrimenti la seconda: 
        condizione ? istruzione1 : istruzione2;
        \item \textbf{concatenazione}: concatena una stringa con un altra stringa o tipo numerico(in questo caso esso viene trasformato 
        in stringa). La concatenazione viene effettuate tramite l'operatore \lstinline[language = java]{+}.
        \end{itemize}
\end{document}