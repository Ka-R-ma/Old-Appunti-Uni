\documentclass{subfile}
\begin{document}
    Per la compilazione e l'esecuzione dei file java si utilizzano rispettivamente:
    \begin{itemize}
        \item \lstinline[language = java]{javac <nomeFile>.java};
        \item \lstinline[language = java]{java <nomeFile>}.
    \end{itemize}
    Il primo compila il file .java ed in caso di successo di genera l'eseguibile \lstinline[language = java]{<nomeFile>.class} 
    mentre il secondo esegue appunto il file .class.

    \begin{Note*}
         Se più classi in un file .java contengono un metodo main verrà eseguito quello con lo stesso nome del file.
    \end{Note*}
    Diversi file java nella stessa cartella possono accedere a tutti i loro attributi e metodi a meno che non siano marcati
    private.

\end{document}