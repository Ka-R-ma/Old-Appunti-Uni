\documentclass{subfile}
\begin{document}
Per costruire gli oggetti e cambiarne lo stato si usa il metodo main dichiarato con:

\lstinline[language = java]{public static void main(String[] args)}.\\\\
Nel metodo main funge da client per tutte le classi quindi in esso si scrive il programma eseguibile che genererà gli
eventuali oggetti. Solitamente, il metodo main si mette in un file a parte denominato \textbf{Main.java}.\\\\
Nel metodo main per richiamare metodi o attributi di una classe si utilizza la notazione punto:
\lstinline[language = java]{<nomeOggetto>.<nomeMetodo>(<parametri>)}.\\
\begin{Example*}
    \item []
    \begin{center}
        \begin{lstlisting}[style = general, language = java]
            public static void main (String[] args){
                Serbatoio = new Serbatoio();
                s.rifornisci(12);
                System.out.println("Il livello e`:" + s.getLivello());
            }
        \end{lstlisting}
    \end{center}
\end{Example*}
\clearpage
\end{document}