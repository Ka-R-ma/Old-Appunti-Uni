Un sistema operativo (SO) può essere pensato come un insieme di programmi, cui scopo è quello di gestire l'hardware e di interfacciare l'utente e la macchina.
\\ \\
Se si pensa ad un generico calcolatore, si può pensare a questi come l'insieme di \emph{SO, hardware, applicativi, utenti}.
Da tale decomposizione, il sistema operativo ha anche il compito di gestire il coordinamento dei vari applicativi.
\begin{Definition*}
    Un sistema operativo è l'unico programma sempre in esecuzione su una macchina con lo scopo di garantire il corretto funzionamento della stessa.
\end{Definition*}

\subsection{SO e applicativi}
\subfile{Sotto Sezioni/Sottosezione 1.1 - SO e applicativi.tex}