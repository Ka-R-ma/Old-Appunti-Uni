Si pensi alla GUI come una versione semplificata della CLI\footnotemark.
In quanto versione semplificata risulta ovvio che la GUI sia meno potente della CLI, seppur svolgendone i medesimi compiti.

\noindent Si può in tal senso pensare alla GUI come un programma che ne esegue altri.
Analizzando la generica esecuzione di un programma da GUI, questa fa si che vengano eseguite, nell'ordine le \emph{syscall} fork ed exec.

\subsubsection{SO e syscall}
Considerando una generica syscall, questa si compone di due parti quali un'identificatore e un insieme di parametri.
\begin{itemize}
    \item \textbf{Identificatore:} codice univoco che permette di accedere in maniera diretta al programma.
    \item \textbf{Parametri:} insieme di informazioni utili all'esecuzione del programma.
          Si osservi che qual'ora il numero di parametri superi quello dei registri, i primi verranno memorizzati in un blocco di memoria,
          successivamente si passerà a parametro il puntatore a tale locazione.
\end{itemize}