La struttura con cui un sistema operativo può essere realizzata non è univoca, esistono difatti tre tipologie principali di SO.
\begin{itemize}
    \item \textbf{SO a struttura semplice:} anche detti monolitici, sono sistemi \emph{tightly-coupled} in cui non vi è una netta suddivisione del sistema.
          Con questa tipologia, le applicazioni accedono direttamente alle routine, scrivendo direttamente su disco e/o video.

    \item \textbf{SO con struttura a strati:} sono sistemi \emph{loosely-coupled} in cui il sistema è diviso in strati,
          ciascuno dei quali può utilizzare funzioni proprie o dello strato successivo.

    \item \textbf{SO con struttura a micro-kernel:} sono sistemi simili a quelli a strati, ne differiscono per i compiti assegnati al kernel:
          si fa si che quest'ultimo gestisca solamente i compite strettamente necessari alla sopravvivenza del sistema stesso.
\end{itemize}