\documentclass{subfiles}
\begin{document}
\begin{Definition*}
    Dato \(\Sigma\) un certo alfabeto, si definisce grammatica \(G\) la seguente quadrupla.
    \[
        G = (\Sigma, V, S, P)
    \]
    ove
    \begin{itemize}
        \item \(\Sigma\) è l'alfabeto di simboli terminali;
        \item \(V\) è l'alfabeto dei simbolo non terminali;
        \item \(S\) è un simbolo non terminale detto assioma;
        \item \(P\) è l'insieme delle regole di produzione.
    \end{itemize}
\end{Definition*}
\begin{Example*}
    Sia \(\Sigma = \Set{a, b}\), sia \(V = \Set{S}\). Stabilire una grammatica che generi il linguaggio \(a^{n}b^{n}\).
    \begin{Solution*}
        Poiché \(V = \Set{S}\), sia \(S\) l'assioma, segue che le regole di produzione sono le seguenti.
        \[\begin{aligned}
                \textcircled{1} & \quad S \to ab  \\
                \textcircled{2} & \quad S \to aSb \\
            \end{aligned}\]

        \noindent Considerando ad esempio la parola \(a^{3}b^{3}\), questa è ottenuta applicando due volte \textcircled{2} di produzione, seguite da \textcircled{1}.
    \end{Solution*}
\end{Example*}

\subsection{Alberi sintattici}
\subfile{Sotto Sezioni/Sottosezione 6.1 - Alberi sintattici.tex}

\subsection{Proprietà delle CFG}
\subfile{Sotto Sezioni/Sottosezione 6.2 - Proprieta delle CFG.tex}

\subsection{Grammatiche unilaterali}
\subfile{Sotto Sezioni/Sottosezione 6.3 - Grammatiche unilaterali.tex}

\subsection{Forma normale di Chomsky}
\subfile{Sotto Sezioni/Sottosezione 6.4 - Forma normale di Chomsky.tex}

\subsection{Pumping Lemma per le CFG}
\subfile{Sotto Sezioni/Sottosezione 6.5 - Pumping Lemma per i CF.tex}

\subsection{Gerarchia di Chomsky}
\subfile{Sotto Sezioni/Sottosezione 6.6 - Gerarchia di Chomsky.tex}
\end{document}