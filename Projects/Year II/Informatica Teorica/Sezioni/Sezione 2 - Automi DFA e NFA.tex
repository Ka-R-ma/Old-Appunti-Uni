\documentclass{subfiles}
\begin{document}
Come anticipato in \emph{Sezione \eqref{sec:1}}: un automa è un sistema automatico, rappresentato da un grafo.

\noindent Si tenga presente che esistono due classi di automi
\begin{itemize}
    \item deterministici o DFA;
    \item non deterministici o NFA.
\end{itemize}

\subsection{I DFA}
\subfile{Sotto Sezioni/Sottosezione 2.1 - I DFA.tex}

\subsection{Gli NFA}
\subfile{Sotto Sezioni/Sottosezione 2.2 - Gli NFA.tex}

\subsection{DFA e NFA: linguaggi e proprietà dei linguaggi}
\subfile{Sotto Sezioni/Sottosezione 2.3 - DFA e NFA, linguaggi e proprita dei linguaggi.tex}

\subsection{Equivalenza tra NFA e DFA}
\subfile{Sotto Sezioni/Sottosezione 2.4 - Equivalenza tra DFA e NFA.tex}

\subsection{Esercizi su DFA e NFA}
\subfile{Sotto Sezioni/Sottosezione 2.5 - Esercizi su DFA e NFA.tex}
\end{document}