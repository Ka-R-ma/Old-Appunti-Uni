\documentclass{subfiles}
\begin{document}
Concetti  centrali della teoria degli automi sono gli alfabeti, le stringhe e i linguaggi.
\begin{itemize}
    \item \emph{Gli alfabeti:}
          si definisce alfabeto \(\Sigma\) un insieme finito di caratteri.

    \item \emph{Le Stringhe:}
          dato \(\Sigma\) un alfabeto, si definisce stringa \(\omega\) una sequenza di simboli scelti dall'alfabeto.

          Caso particolare è la stringa vuota \(\varepsilon\): una stringa composta da zero simboli.


          Data \(\omega\) una stringa, di questa è possibile stabilirne la lunghezza: ossia il numero di caratteri di cui si compone.

          Infine, considerate \(\omega_{1} = a_{1} \cdots a_{k} \text{e} \omega_{2} = b_{1} \cdots b_{j}\) due stringhe,
          si definisce \(\omega_{1} \circ \omega_{2} = \omega_{1}\omega_{2}=  a_{1} \cdots a_{k} b_{1} \cdots b_{j}\) concatenazione di \(\omega_{1} \text{e} \omega_{2}\).

    \item \emph{I Linguaggi:} dato \(\Sigma\) un alfabeto, si definisce linguaggio \(L \text{su} \Sigma\) un sottoinsieme delle stringe ottenibili con l'alfabeto.
\end{itemize}
\clearpage
\end{document}