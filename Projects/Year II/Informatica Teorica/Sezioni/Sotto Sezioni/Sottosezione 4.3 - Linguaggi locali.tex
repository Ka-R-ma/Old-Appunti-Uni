\documentclass{subfiles}
\begin{document}
\begin{Definition*}
    Sia \(L\) un linguaggio. Questi dicasi locale se esprimibile tramite la seguente quadrupla.
    \[
        (Ini(L), Fin(L), Dig(L), Null(L))
    \]
    ove
    \begin{itemize}
        \item \(Ini(L)\) stabilisce l'insieme di caratteri con cui \(\omega \in L\) può iniziare.
        \item \(Fin(L)\) stabilisce l'insieme di caratteri con cui \(\omega \in L\) può terminare.
        \item \(Dig(L)\) stabilisce l'insieme di tutte le possibili coppie di caratteri che \(\omega \in L\) può contenere.
        \item \(Null(L)\) stabilisce se l'insieme contiene o meno la parola vuota.
    \end{itemize}
\end{Definition*}

\subsubsection{Calcolo ricorsivo di Ini, Fin, Dig, Null}
Sia \(L\) un linguaggio locale. La costruzione della quadrupla, che è ricorsiva, è descritta a seguire.

\begin{itemize}
    \item \textbf{Ini:}
          considerando la parte ricorsiva
          \begin{itemize}
              \item \(Ini(e + f) = Ini(e) \cup Ini(f)\);
              \item \(Ini(ef) = Ini(e) \cup Null(e) Ini(f)\);
              \item \(Ini(e^{*}) = Ini(e)\).
          \end{itemize}

    \item \textbf{Fin:}
          considerando la parte ricorsiva
          \begin{itemize}
              \item \(Fin(e + f) = Fin(e) \cup Fin(f)\);
              \item \(Fin(ef) = Fin(f) \cup Fin(e) Null(f)\);
              \item \(Fin(e^{*}) = Fin(e)\).
          \end{itemize}

    \item \textbf{Null:}
          considerando la parte ricorsiva
          \begin{itemize}
              \item \(Null(e + f) = Null(e) \cup Null(f)\);
              \item \(Null(ef) = Null(e) \cap Null(f)\);
              \item \(Null(e^{*}) = \varepsilon\).
          \end{itemize}

    \item \textbf{Dig:}
          considerando la parte ricorsiva
          \begin{itemize}
              \item \(Dig(e + f) = Dig(e) \cup Dig(f)\);
              \item \(Dig(ef) = Dig(e) \cup Dig(f) \cup Fin(e)Ini(f)\);
              \item \(Dig(e^{*}) = Dig(e) \cup Fin(e)Ini(e)\).
          \end{itemize}
\end{itemize}

\begin{Note*}
    Nel descrivere il calcolo di \(Ini, Fin, Dig, Null\), è stata trascurata la base.

    \noindent Infatti \(*(\varepsilon) = \varepsilon, *(\emptyset) = \emptyset\), ove \(* \text{sostituisce} Ini, Fin, Dig, Null\), escluso \(Dig(\varepsilon) = \emptyset\).

    \noindent Inoltre \(Ini(a) = Fin(a) = a \text{se} a\) è un carattere, mentre \(Dig(a) = Null(a) = \emptyset\).
\end{Note*}

\subsubsection{Automi locali}
\begin{Definition*}
    Sia \(L\) un linguaggio locale. Si definisce \emph{automa locale} un DFA che riconosce \(L\).

    \noindent La costruzione dell'automa locale è realizzata secondo quanto segue.
    \begin{itemize}
        \item \(Q = \Set{q_{0}} \cup \Sigma\);
        \item Se \(Null(L) = \varepsilon\) allora \(q_{0}\) è accettante;
        \item \(\forall a_{i} \in \Sigma\), ogni arco etichettato \(a_{i}\), entra nello stato \(q_{a_{i}}\);
        \item Da \(q_{0}\) escono le transizioni definite da \(Ini\);
        \item Le altre transizioni sono definite da \(Dig\);
        \item Gli stati finali sono indicati da \(Fin\).
    \end{itemize}
\end{Definition*}
\end{document}