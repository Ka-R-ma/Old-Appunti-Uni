\documentclass{subfiles}
\begin{document}
Capita spesso di perdere molto tempo nello stabilire se un linguaggio \(L\) è regolare o meno.
Per semplificare tale processo è possibile utilizzare uno strumento molto potente: il \emph{pumping lemma}.

\begin{Lemma}
    Sia \(L\) un linguaggio. Questi non è regolare se
    \[
        \forall n \such \exists \omega \in L, \abs{\omega} \le n, \exists x,y,z \such \omega = xyz
    \]
    per cui almeno una delle seguenti proprietà non è soddisfatta.
    \begin{multicols}{3}
        \begin{itemize}
            \item \(y \neq \varepsilon\)
            \item \(\abs{xy} \le n\)
            \item \(\forall k \ge 0, xy^{k}z \in L\)
        \end{itemize}
    \end{multicols}
\end{Lemma}

\begin{Exercise*}
    Sia \(L = \set{\omega \in \Sigma = \Set{a, b}}{\omega = a^{n}b^{n}}\). Stabilire se \(L\) è regolare.
    \begin{Solution*}
        Procedendo applicando il pumping lemma
        \[
            \forall n, \qquad a^{n}b^{n} \in L
        \]
        cioè
        \[
            \underbrace{a \cdots a}_{n \ volte} \underbrace{b \cdots b}_{n \ volte} \in L
        \]

        \noindent Considerando ora le partizioni valide
        \begin{enumerate}
            \item \(x = \underbrace{a \cdots a}_{i \ volte},  y = \underbrace{a \cdots a}_{j \ volte}, z = \underbrace{b\cdots b}_{n \ volte} \text{con} i + j = n\).
            \item \(x = \underbrace{a \cdots a}_{i \ volte}, y = \underbrace{a \cdots a}_{j \ volte}\underbrace{b \cdots b}_{l \ volte}, z = \underbrace{b \cdots b}_{h \ volte} \text{con} i + j = n, l + h = n\).
        \end{enumerate}
        si ha che la prima viola la terza proprietà per \(k = 0\), la seconda partizione, e in generale quelle che richiedono almeno una \(b\), violano la seconda proprietà.
    \end{Solution*}
\end{Exercise*}
\end{document}