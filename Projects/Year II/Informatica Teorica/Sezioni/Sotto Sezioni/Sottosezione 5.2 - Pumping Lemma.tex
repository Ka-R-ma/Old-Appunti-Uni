\documentclass{subfiles}
\begin{document}
Capita spesso di perdere molto tempo nello stabilire se un linguaggio \(L\) è regolare o meno.
Per semplificare tale processo è possibile utilizzare uno strumento molto potente: il \emph{pumping lemma}.

\begin{Lemma}
    Sia \(L\) un linguaggio. Questi non è regolare se
    \[
        \forall n \such \exists \omega \in L, \abs{\omega} \ge n, \exists x,y,z \such \omega = xyz
    \]
    per cui almeno una delle seguenti proprietà non è soddisfatta.
    \begin{multicols}{3}
        \begin{itemize}
            \item \(y \neq \varepsilon\)
            \item \(\abs{xy} \le n\)
            \item \(\forall k \ge 0, xy^{k}z \in L\)
        \end{itemize}
    \end{multicols}
\end{Lemma}

\begin{Exercise*}
    Sia \(L = \set{\omega}{\omega = a^{n}b^{n}, n \ge 0}\). Stabilire se \(L\) è regolare.
    \begin{Solution*}
        Sia supposto \(L\) non regolare, segue
        \[
            \forall n \quad a^{n}b^{n} = \underbrace{a \cdots a}_{n \ volte} \underbrace{b \cdots b}_{\ n \ volte} \in L
        \]

        \noindent Procedendo col considerare alcune partizioni
        \begin{enumerate}
            \item Sia \(x = a^{i}, y = a^{j}, z = b^{n}\) tale che \(i + j = n\):
                  si osserva che, posto \(k = 0\), segue \(a^{i}b^{n} \notin L\), poiché \(i < n\).

            \item Sia \(x = a^{i}, y = a^{j}b^{h}, z = b^{l}\) tale che \(i + j = n \text{,} h + l = n\):
                  si osserva però che \(\abs{xy} = \abs{a^{i + j}b^{h}} > n\).
        \end{enumerate}

        \begin{Note*}
            Le partizioni non riportate sono state trascurate, in quanto ovvio non soddisfacenti almeno una delle proprietà.
        \end{Note*}
    \end{Solution*}
\end{Exercise*}
\begin{Remark*}
    Sia il pumping lemma per i linguaggi REC, sia quello per i linguaggi CF in \emph{Sezione \eqref{sec:6.5}}, garantiscono esclusivamente la non appartenenza ad una data famiglia di linguaggi.
    Cioè, se un linguaggio soddisfa il pumping lemma, non è certo che questi sia regolare (o CF).
\end{Remark*}
\clearpage
\end{document}