\documentclass{subfiles}
\begin{document}
\begin{Definition*}
    Sia \(e\) una RegEx. Questa si dice lineare se nessun carattere in \(e\) è ripetuto.
\end{Definition*}

\noindent Sia \(e\) una RegEx non lineare. Questa può essere linearizzata semplicemente ridefinendo le occorrenze multiple,
così che l'espressione \(e'\) ottenuta dalla linearizzazione sia definito su un nuovo alfabeto \(\Sigma^{'}\).

\begin{Algorithm*}
    Sia \(e\) espressione regolare. La costruzione di un'automa locale che riconosce \(e\) è realizzata come segue.
    \begin{enumerate}
        \item Se \(e\) è regolare si al punto 2, altrimenti si procede alla linearizzazione.
        \item Si definisce da quadrupla \((Ini(L), Fin(L), Dig(L), Null(L))\), procedendo ricorsivamente al calcolo degli insiemi.
        \item Si costruisce l'automa locale seguendo le transizioni della quadrupla.
        \item Se la quadrupla è definita dopo aver linearizzato \(e\), allora si procede rimuovendo la ridefinizione dei caratteri.
        \item Si procede alla subset construction.
    \end{enumerate}
\end{Algorithm*}

\begin{Example*}
    Sia \(e = (ab)^{*} + c^{*}\). Si costruisce un automa che riconosca \(e\).

    \begin{Solution*}
        Procedendo applicando l'algoritmo si ha quanto segue.
        \begin{enumerate}
            \item Si osserva che \(e\) è lineare, si passa dunque alla costruzione della quadrupla, da cui
                  \begin{multicols}{2}
                      \begin{itemize}
                          \item \(Ini(e) = \Set{a, c}\)
                          \item \(Fin(e) = \Set{b, c}\)
                          \item \(Dig(e) = \Set{ab, ba, cc}\)
                          \item \(Null(e) = \varepsilon\)
                      \end{itemize}
                  \end{multicols}

            \item Procedendo alla costruzione dell'automa, segue
                  \subfile{../Figure/Figura Es 4.4.1.tex}

                  \begin{Remark*}
                      Poiché \(e\) è lineare si ha che non è necessario procedere alla subset construction.
                      Infatti l'automa di cui sopra è gia un DFA.
                  \end{Remark*}
        \end{enumerate}
    \end{Solution*}
\end{Example*}
\clearpage

\begin{Example*}
    Sia \(e = (ab + a)^{*} ba^{*}\). Si costruisca un automa che riconosca \(e\).
    \begin{Solution*}
        Procedendo applicando l'algoritmo si ha quanto segue.
        \begin{enumerate}
            \item Si osserva che \(e\) non è lineare, si procede alla sua linearizzazione. Segue che
                  \[
                      e = (ab + a)^{*}ba^{*}  \quad \text{diventa} \quad e' = (ab + c)^{*}df^{*}
                  \]
                  \vspace{-12.5pt}
            \item Considerando la quadrupla, segue
                  \begin{multicols}{2}
                      \begin{itemize}
                          \item \(Ini(e') = \Set{a, c, d}\)
                          \item \(Fin(e') = \Set{d, f}\)
                          \item \(Dig(e') = \Set{ab, bc, ca, ba, \ldots}\)
                          \item \(Null(e') = \emptyset\)
                      \end{itemize}
                  \end{multicols}

            \item Passando all'automa, segue
                  \subfile{../Figure/Figura Es 4.4.2.tex}

            \item Procedendo rimuovendo la ridefinizione dei caratteri, si ha
                  \subfile{../Figure/Figura Es 4.4.3.tex}

            \item Concludendo con la subset construction, segue
                  \subfile{../Figure/Figura Es 4.4.4.tex}

        \end{enumerate}
    \end{Solution*}
\end{Example*}
\end{document}