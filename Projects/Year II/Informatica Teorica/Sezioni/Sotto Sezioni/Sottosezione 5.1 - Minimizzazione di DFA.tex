\documentclass{subfiles}
\begin{document}
Sia \(A\) un DFA. A seguito di quanto detto sopra, ne consegue che è possibile realizzare un DFA \(B \text{equivalente ad} A\), ma con un numero minimo di stati.
Analogo ragionamento è estensibile agli NFA, sebbene per questi non è sempre vero.

\subsubsection{Relazione di indistinguibilità}
Sia \(A\) un DFA, siano \(p \text{e} q\) suoi stati. Si ha che
\[
    p \given q \iff (\delta^{*}(q, \omega) \land \delta^{*}(q, \omega)) \in F, \quad \forall \omega \in \Sigma^{*}
\]
oppure
\[
    p \given q \iff (\delta^{*}(q, \omega) \land \delta^{*}(q, \omega)) \notin F, \quad \forall \omega \in \Sigma^{*}
\]

\noindent Cioè \(p \text{e} q\) sono indistinguibili se per ogni parola del linguaggio universale si ha che, calcolando la funzione di transizione estesa per i due stati,
entrambi conducono ad uno stato accettante/rifiutante per \(\omega\).

\subsubsection{Algoritmo riempi-tabella}
Uno strumento utile alla minimizzazione è l'algoritmo riempi-tabella, con il quale è possibile stabilire ricorsivamente gli stati equivalenti.
\begin{Base*}
    Se \(p\) è accettante e \(q\) non lo è, allora la coppia \((p, q)\) è distinguibile.
\end{Base*}
\begin{Induction*}
    Se \(p, q\) sono stati tali che, per un simbolo di input \(\alpha\), si ha che
    \[
        \delta(p, \alpha) \land \delta(q, \alpha)
    \]
    conducono a stati noti come distinguibili, allora \((p, q)\) sono distinguibili.
\end{Induction*}

\begin{Theorem}
    Se due stati non sono distinti dall'algoritmo riempi-tabella, alloro sono equivalenti.
\end{Theorem}
\end{document}