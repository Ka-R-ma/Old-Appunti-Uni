\documentclass{subfiles}
\begin{document}
\begin{Definition*}
    Sia \(Q\) un problema. Dicasi \(Q\) problema di \emph{classe P} se e solo se \(Q\) è risolvibile da una MT polinomialmente deterministica.
\end{Definition*}

\begin{Definition*}
    Sia \(L\) un linguaggio. Si dirà \(L\) appartenente alla \emph{classe P} se:
    esiste un polinomio \(f(n)\) tale che \(L\) sia deciso da una MT polinomialmente deterministica in tempo \(f(n)\).
\end{Definition*}

\begin{Definition*}
    Sia \(L\) un linguaggio. Si dirà \(L\) appartenente alla \emph{classe NP} se:
    esiste una MT polinomialmente non deterministica che lo riconosce in tempo polinomiale \(f(n)\).
\end{Definition*}

\begin{Remark*}
    Poiché ogni MT deterministica è anche non deterministica, si ha
    \[
        P \subseteq NP
    \]

    \noindent Da cio ci si potrebbe chiedere \(P = NP\)? Ad oggi tale domanda rimane ancora senza una risposta certa.
\end{Remark*}
\clearpage

\subsubsection{Problemi NP-completi}
\begin{Definition*}
    Sia \(P\) un problema. Dicasi \(P\) essere NP-completo se
    \begin{itemize}
        \item \(P \in NP\);
        \item \(\forall P^{'} \in NP, \exists f \such \omega \in P^{'} \iff f(\omega) \in P\).
    \end{itemize}
\end{Definition*}

\begin{Theorem}\label{thm:9.1}
    Sia \(P_{1} \in NP-completi\), sia \(P_{2} \in NP\). Se \(P_{1}\) è polinomialmente riducibile a \(P_{2}\), allora \(P_{2} \in NP-completi\).
\end{Theorem}

\begin{Theorem}\label{thm:9.2}
    Se \(Q \in NP-completi \land Q \in P \implies P = NP\).
\end{Theorem}

\begin{Note*}
    Dal \emph{Teorema \eqref{thm:9.2}} segue che, qualora ci si riuscisse, dimostrando che un problema \(Q_{1}\) appartenente agli NP-completi è riducibile ad un problema \(Q_{2}\) in P,
    si dimostrebbe dal \emph{Teorema \eqref{thm:9.1}} che ogni problema NP-completo è riducibile a \(Q_{2}\), dimostrando in tal modo \(P = NP\).
\end{Note*}
\end{document}