\documentclass{subfiles}
\begin{document}
Sia \(G\) una CFG, questa può essere trasformata in PDA applicando il seguente algoritmo.
\begin{Algorithm*}
    \begin{enumerate}
        \item []
        \item L'alfabeto del PDA sarà l'alfabeto della CFG, cioè \(\Sigma_{P} = \Sigma_{G}\).
        \item L'alfabeto di pila è dato dall'unione dell'alfabeto della CFG, dei simboli non terminali e del simbolo di pila vuota, cioè \(\Gamma = \Sigma_{G} \cup V_{G} \cup \Set{Z_{0}}\).
        \item Il PDA ha due soli stai: \(q_{0}\) iniziale e \(q_{1}\) finale. Se realizzato con pila vuota, in un certo senso, \(q_{0} = q_{1}\).
        \item Allo stato \(q_{1}\) si arriva solo per transizioni da \(q_{0}\), con mosse del tipo
              \[
                  \delta(q_{0}, \varepsilon, Z_{0}) \to (q_{1}, Z_{0})
              \]
        \item Per ogni carattere \(x \in \Sigma_{P}\) si definisce una transizione del tipo
              \[
                  \delta(q_{0}, x, x) \to (q_{0}, \varepsilon)
              \]
        \item Le altre transizioni sono definite a partire dalle regole di produzione.
              \begin{itemize}
                  \item Se la regola è del tipo \(A \to BA_{1} \cdots A_{n}, n \ge 0\), si aggiunge una transizione del tipo \(\delta(q_{0}, \varepsilon, A) \to (q_{0}, A_{n} \cdots A_{1}B)\).
                  \item Se la regola è del tipo \(A \to bA_{1} \cdots A_{n}, n \ge 0\), si aggiunge una transizione del tipo \(\delta(q_{0}, \varepsilon, A) \to (q_{0}, A_{n} \cdots A_{1})\).
                  \item Se la regola è del tipo \(S \to \varepsilon \), si aggiunge una transizione del tipo \(\delta(q_{0}, \varepsilon, S) \to (q_{0}, \varepsilon)\).
              \end{itemize}
    \end{enumerate}
\end{Algorithm*}
\clearpage

\begin{Exercise*}
    Sia \(L = \set{\omega}{\omega = a^{n}b^{n + 2}, n \ge 0}\). Costruire il PDA che riconosce \(L\).
    \begin{Solution*}
        Si procede stabilendo per prima cosa la CFG che genera \(L\). Posto \(S\) l'assioma, si ha
        \[\begin{aligned}
                 & S \to Abb \given bb \\
                 & A \to ab \given aAb \\
            \end{aligned}\]

        \noindent Normalizzando secondo Chomsky, segue
        \[\begin{aligned}
                 & S \to DB \given BB \\
                 & D \to CB           \\
                 & C \to AD \given AB \\
                 & B \to b            \\
                 & A \to a            \\
            \end{aligned}\]

        \noindent Considerando adesso le transizioni del PDA, dall'algoritmo precedentemente introdotto segue
        \[\begin{aligned}
                 & \delta(q_{0}, \varepsilon, Z_{0}) \to (q_{0}, S) \\
                 & \delta(q_{0}, a, a) \to (q_{0}, \varepsilon)     \\
                 & \delta(q_{0}, b, b) \to (q_{0}, \varepsilon)     \\
                 & \delta(q_{0}, \varepsilon, S) \to (q_{0}, DB)    \\
                 & \delta(q_{0}, \varepsilon, S) \to (q_{0}, BB)    \\
                 & \delta(q_{0}, \varepsilon, D) \to (q_{0}, CB)    \\
                 & \delta(q_{0}, \varepsilon, C) \to (q_{0}, AD)    \\
                 & \delta(q_{0}, \varepsilon, C) \to (q_{0}, AB)    \\
                 & \delta(q_{0}, \varepsilon, B) \to (q_{0}, b)     \\
                 & \delta(q_{0}, \varepsilon, A ) \to (q_{0}, a)    \\
            \end{aligned}\]

        \noindent da cui in conclusione si ha il seguente automa, accettante per pila vuota.
        \subfile{../Figure/Figura Es 7.1.1.tex}
    \end{Solution*}
\end{Exercise*}

\clearpage
\end{document}





