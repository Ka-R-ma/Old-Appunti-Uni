\documentclass{subfiles}
\begin{document}
A differenza di un automa, le MT hanno la possibilità di non terminare mai la loro esecuzione.
Da ciò nasce il \emph{problema di arresto di una MT}.

\noindent Parlando ora dei linguaggi validi per una MT, si hanno
\begin{itemize}
    \item \emph{linguaggi riconosciuti:} linguaggi che per ogni parola appartenente o meno al linguaggio, portano ad un arresto;
    \item \emph{linguaggi accettati:} linguaggi che portano ad un arresto solamente per parole interne al linguaggio.
\end{itemize}

\subsubsection{Tesi di Turing-Church}
La tesi di Turing-Church, che sebbene sia solo una tesi non è mai stata confutata, stabilisce: \emph{una funzione è calcolabile se e solo se una MT la calcola.}

\subsubsection{Codifica binaria di una MT}
Sia \(M\) una macchina di Turing, siano \(q_{1}, \ldots, q_{r}\) suoi stati, sia \(q_{1}\) stato iniziale e \(q_{2}\) finale.
Siano \(X_{1}, \ldots, X_{s}\) i simboli di nastro, siano \(0 = X_{1}, 1 = X_{2}, B = X_{3}\) siano infine \(L = D_{1}, R = D_{2}\).
Considerando la funzione di transizione
\[
    \delta(q_{i}, X_{j}) = (q_{k}, X_{l}, D_{m})
\]
sia \(0^{i}10^{j}10^{k}10^{l}10^{m}\) la sua codifica.

\noindent Poiché le transizioni sono in numero finito, posta \(C_{t}\) la t-esima transizione, segue che
\[
    C_{1}11C_{2}11 \cdots C_{n - 1} 11 C_{n}
\]
sono tutte le transizioni, descrivendo difatti l'intera MT, ove 11 funge da separatore tra le transizioni.

\begin{Example*}
    Sia \(M = (\Set{q_{1}, q_{2}, q_{3}}, \Set{0, 1}, \Set{0, 1, B}, \delta, q_{1}, B, \Set{q_{2}})\). Stabilire la codifica di \(M\) sapendo che le transizioni sono le seguenti.
    \[\begin{aligned}
            T_{1} = \delta(q_{1}, 1) & = (q_{3}, 0, R) \\
            T_{2} = \delta(q_{3}, 0) & = (q_{1}, 1, R) \\
            T_{3} = \delta(q_{3}, 1) & = (q_{2}, 0, R) \\
            T_{4} = \delta(q_{3}, B) & = (q_{3}, 1, L) \\
        \end{aligned}\]
    \begin{Solution*}
        Siano \(q_{1} = 0^{1}, q_{2} = 0^{2}, q_{3} = 0^{3}\), siano \(0 = 0^{1}, 1 = 0^{2}\) inoltre siano \(B = 0^{3}, L = 0^{1}, R = 0^{2}\), separando ogni parte della transizione con un uno,
        e ogni transizione con due uno segue che
        \[
            M = \underbrace{0100100010100}_{T_{1}} 11  \underbrace{0001010100100}_{T_{2}} 11 \underbrace{00010010010100}_{T_{3}} 11 \underbrace{0001000100010010}_{T_{4}}
        \]
    \end{Solution*}
\end{Example*}
\end{document}