\documentclass{subfiles}
\begin{document}\label{sec:8}
\begin{Definition*}
    Si definisce \emph{macchina di Turing}, (o MT), un modello formale di macchina capace di eseguire algoritmi,
    composti da un numero di passi elementari di calcolo.
\end{Definition*}

\subsection{Notazione per le MT}
\subfile{Sotto Sezioni/Sottosezione 8.1 - Notazione per le MT.tex}

\subsection{Istantanea di una MT}
\subfile{Sotto Sezioni/Sottosezione 8.2 - Istantanea di una MT.tex}

\subsection{Tesi di Turing-Church e codifica binaria di una MT}
\subfile{Sotto Sezioni/Sottosezione 8.3 - Tesi di Turing-Church & codifica di una MT.tex}

\subsection{Linguaggio diagonale e Linguaggio universale}
\subfile{Sotto Sezioni/Sottosezione 8.4 - Linguaggio diagonale e linguaggio universale.tex}
\end{document}