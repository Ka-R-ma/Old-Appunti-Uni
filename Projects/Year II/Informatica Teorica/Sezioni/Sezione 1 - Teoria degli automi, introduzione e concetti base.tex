\documentclass{subfiles}
\begin{document}
Si consideri il caso di un interruttore. Grazie agli automi è possibile rappresentare facilmente il passaggio tra i due stati come mostrato in \ref{fig:1}.
\subfile{Figure/Figura 1 - Switch come DFA.tex} \label{sec:1}

\noindent Dando una breve definizione di automa: questi è un sistema automatico, rappresentato da un grafo i cui nodi rappresentano gli stati e gli archi le transizioni tra stati.
\\ \\
L'utilizzo degli automi è da ricercare nello studio dei limiti computazionali, cui si legano
\begin{enumerate}
    \item lo studio della decidibilità, che stabilisce cosa possa fare un computer in assoluto;
    \item lo studio della trattabilità che stabilisce cosa possa fare un compute efficientemente.
\end{enumerate}

\noindent Agli automi sono inoltre legati due importanti nozioni, quali le grammatiche e le espressioni regolari, che si discuteranno nelle successive sezioni.

\subsection{Concetti centrali}
\subfile{Sotto Sezioni/Sottosezione 1.1 - Concetti centrali.tex}
\end{document}