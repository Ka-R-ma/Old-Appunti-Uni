\documentclass{subfiles}
\begin{document}
La \emph{teoria della complessità} si occupa di studiare la complessità computazionale di un problema.
\noindent Cioè, dato \(P\) un problema, posto che questi sia trattabile, quale funzione di complessità caratterizza l'algoritmo che risolve \(P\)?
\\ \\
Come anticipato sopra, esistono problemi \emph{trattabili} e conseguentemente problemi \emph{non trattabili}.
\begin{Definition*}
    Sia \(P\) un problema. Si dirà che \(P\) è trattabile se è possibile dimostrare che lo stesso è risolvibile da una MT deterministica.
    \noindent Si dirà \(P\) non trattabile altrimenti.
\end{Definition*}

\subsection{Riduzione polinomiale}
\subfile{Sotto Sezioni/Sottosezione 9.1 - Riduzione polinomiale.tex}

\subsection{Problemi P e NP}
\subfile{Sotto Sezioni/Sottosezione 9.2 - Problemi P e NP.tex}
\end{document}