\documentclass{subfiles}
\begin{document}
\begin{Definition*}
    Sia \(L\) un linguaggio. Questi si definisce regolare se accettato da un DFA.
\end{Definition*}

\noindent I linguaggi regolari sono chiusi, cioè rimangono regolari, rispetto operazioni quali
\begin{itemize}
    \item \emph{intersezione};
    \item \emph{unione};
    \item \emph{complemento};
    \item \emph{Kleene};
    \item \emph{croce}.
\end{itemize}

\subsection{Chiusura per intersezione}
\subfile{Sotto Sezioni/Sottosezione 3.1 - Chiusura per intersezione.tex}

\subsection{Chiusura per unione}
\subfile{Sotto Sezioni/Sottosezione 3.2 - Chiusura per unione.tex}

\begin{Note*}
    Similarmente si dimostra anche la chiusura per complemento, Kleene, croce.
\end{Note*}
\end{document}