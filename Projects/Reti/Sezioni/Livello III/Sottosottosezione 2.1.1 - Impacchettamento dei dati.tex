\documentclass{subfiles}
\begin{document}
Sia considerata la struttura protocollare mostrata in \emph{Figura \emph{fig:1}}.
Un messaggio $m$ prodotto dal livello 5 è trasmesso al livello 4.
Giunto al livello 4 ad $m$ si aggiunge un \emph{header}\footnotemark[1] e si trasferisce il messaggio al livello 3.

Poiché spesso si impone un limite alle dimensioni del messaggio nei livelli più bassi, in genere il messaggio è diviso in pacchetti:
si tratta di pezzi del messaggio originale ai quali, per permettere una corretta ricostruzione del messaggio, si aggiunge un ulteriore header.

Giunto al destinatario, il messaggio si muove dall'alto in basso, rimuovendo di volta in volta gli header.
I peer dello stesso livello, pertanto modellano concettualmente la loro comunicazione come se fosse “orizzontale”,
anche se in realtà comunicano mediante livelli inferiori attraverso le interfacce.

\subfile{../../Figure/Tikz Figure/Figure 1 - Esempio di impacchettamento dati.tex}
\footnotetext[1]{Si tratta di informazioni che permettono al livello corrispettivo del destinatario di leggere il messaggio correttamente.}
\end{document}