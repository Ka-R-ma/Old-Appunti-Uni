\documentclass{subfiles}
\begin{document}
Esiste una ``legge'' che lega velocità di trasferimento e larghezza di banda: la cosiddetta \emph{condizione di Nyquist}.
Questa stabilisce che
\begin{equation}
    C_{MAX} = 2H \Log{V}[2] \, \Frac*{bit}{s}
\end{equation}
ove $C_{MAX}$ rappresenta la velocità di trasferimento massima, $H$ la larghezza del canale e $V$ il numero di livelli del segnale. \\ \\
Supposto che il canale presenti del rumore $N$, la \eqref{eq:2} diventa
$$
    C_{MAX} = H \Log{\Frac{1 + S}{N}}[2] \, \Frac*{bit}{s}
$$
detta \emph{condizione di Shannon}. Sempre per Shannon si ha che, poiché con la distanza il segnale tenda ad affievolirsi,
anche il rapporto segnale-rumore tende a diminuire.
Per quanto appena detto la condizione di Shannon può essere approssimata come
$$
    C_{MAX} = H \Log{\frac{S}{N}}[2] \, \Frac*{bit}{s}
$$
Ponendo pertanto $S = S_{0}e^{-kx}$, cioè scrivendo $S$ in funzione della distanza dalla sorgente, ove $k$ è l'attenuazione del mezzo, segue
$$\begin{aligned}
        C_{MAX} & = H \left[\Frac{\Log*{S_{0}e^{-kx}} - \Log*{N}}{\Log*{2}}\right]       \\
                & = H \left[\Frac{kx  + \Log*{S_{0}e^{-kx}} - \Log*{N}}{\Log*{2}}\right] \\
                & = (Hc_{1})(c_{2} + kx)
    \end{aligned}$$
da cui assunti $c_{1} = c_{2} = const$ si ha
$$
    kx = const - \Frac{C}{H} const
$$
concludendo che la distanza servibile è inversamente proporzionale alla velocità di trasmissione.
\end{document}