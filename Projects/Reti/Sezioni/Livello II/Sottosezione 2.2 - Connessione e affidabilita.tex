\documentclass{subfiles}
\begin{document}
Ciascuno dei livelli di un protocollo fornisce diversi livelli di affidabilità e due tipologie di servizi:
\begin{enumerate}
    \item orientati alla connessione: i due utenti devono prima stabilire una connessione, utilizzarla e, una volta terminata, rilasciarla.

    \item non orientati alla connessione: tra i due host non si stabilisce una connessione, e ciascun messaggio è trasmesso indipendentemente dall'altro.
          Si verifica così la possibilità che i messaggi non giungano nell'ordine corretto.
\end{enumerate}

Relativamente l'affidabilità, ciascun servizio, sia questi orientato alla connessione che non, deve essere affidabile:
ossia deve permettere di trasmettere e ricevere i dati per intero e nel giusto ordine.
Per garantire l'affidabilità in generale si fa in modo che il ricevente confermi la corretta recezione del messaggio.
Tale operazione però appesantisce la comunicazione, che ne risulta dunque rallentata.
\\ \\
Analizzando un servizio, questi è formalmente specificato da un insieme di primitive.
\begin{MarginNote}
    servizi orientati alla connessione utilizzano primitive diverse da quelli non orientati alla connessione.
\end{MarginNote}


\end{document}