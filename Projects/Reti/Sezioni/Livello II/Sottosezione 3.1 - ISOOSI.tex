\documentclass{subfiles}
\begin{document}
\'E un modello di riferimento strutturato in 7 livelli: i tre più in alto sono dedicati al contenuto del messaggio, gli ultimi quattro alla rete.

I livelli sono:
\begin{enumerate}
    \item \textbf{applicazione:} si occupa di gestire la comunicazione tra le applicazioni;
    \item \textbf{presentazione:} gestisce la sintassi di dati con codifiche diverse;
    \item \textbf{sessione:} gestisce le sessioni tra utenti su macchine diverse;
    \item \textbf{trasporto:} principale obbiettivo è gestire il trasferimento dei dati da sorgente a destinatario.
          Come ulteriori compiti ha quello di gestire gli errori e prevenire la congestione;
    \item \textbf{rete:} gestisce l'instradamento dei pacchetti, della congestione e della qualità del servizio (QoS);
    \item \textbf{data link:} si occupa della creazione dei frame per la trasmissione tra elementi della rete locale, e del controllo degli errori;
    \item \textbf{fisico:} rappresenta le interfacce meccaniche ed elettriche per la trasmissione di bit.
\end{enumerate}
\end{document}