\documentclass{subfiles}
\begin{document}
La topologia di una rete identifica l'area che questa copre; in tal senso si distinguono le seguenti classi.
\begin{itemize}
    \item \textbf{reti PAN:} si tratta di reti a corto raggio. Esempio di queste è la rete bluetooth.

    \item \textbf{reti LAN:} reti la cui estensione non supera va oltre un edificio, permette la connessione di personal computer tra loro.
          Tra le LAN si distinguono il caso cablato, in cui ogni host è connesso ad uno switch che si occupa dello smistamento dei pacchetti;
          il caso wireless (le WLAN), in cui ogni host è connesso tramite un \emph{access point}.

    \item \textbf{reti MAN:} sono reti che coprono un'intera città.

    \item \textbf{reti WAN:} sono le reti di massima estensione in cui ogni sua sotto-rete, direttamente o meno,
          è connessa ad un altra tramite uno o più router.
\end{itemize}
\end{document}