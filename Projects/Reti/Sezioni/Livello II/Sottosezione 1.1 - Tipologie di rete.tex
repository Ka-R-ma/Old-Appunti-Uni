\documentclass{subfiles}
\begin{document}
Le reti, oltre che per quanto detto, sono distinte anche dal tipo di informazione a cui queste permettono di accedere.
In tal senso si hanno
\begin{itemize}
    \item \textbf{reti di accesso a banda larga e mobili:} reti che forniscono ai singoli utenti (host), accedendo a computer remoti,
          di accedere a informazioni e/o comunicare con altre persone. Nate come reti cablate, negli ultimi anni anche wireless.

    \item \textbf{reti per la distribuzione di contenuti:} realizzate per trasmettere grandi mole di dati,
          sfruttano le \emph{content delivery network} per assicurare una distribuzione del contenuto quanto più rapidamente possibile.

    \item \textbf{reti di transito (backbone):} reti di collegamento tra varie sotto-reti.
          In particolare collegano sotto-reti relative \emph{internet service provider (ISP)} diversi.
\end{itemize}
\end{document}