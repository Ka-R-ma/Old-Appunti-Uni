\documentclass{subfiles}
\begin{document}
Capita spesso che quando si ha a che fare con un problema, questi è descrivibile in termini di strutture matematiche elementari, ad esempio gli insiemi.
Per tale ragione in questa sezione si analizzeranno le operazioni elementari sugli insiemi, e sulle strutture adatte alla loro implementazione.
\\ \\
Circa le operazioni sugli insiemi, di interesse al corso sono le seguenti.
\begin{enumerate}
    \item MEMBER(a, S): verifica se a appartiene o meno ad S.
    \item INSERT(a, S): aggiunge a ad S, se a non è già presente.
    \item DELETE(a, S): rimuove a da S, se a è presente.
    \item UNION(S, T, U): gli insiemi S e T sono uniti, e la loro unione è assegnata ad U.
\end{enumerate}

\subsection{Hashing}
\subfile{Sotto sezioni/Sottosezione 5.1 - Hashing.tex}

\subsection{Optimal BST}
\subfile{Sotto sezioni/Sottosezione 5.2 - Optimal binary search tree.tex}
\clearpage

\subsection{Problema della Union-Find con struttura ad albero}
\subfile{Sotto sezioni/Sottosezione 5.3 - Problema della Union-Find con struttura ad albero.tex}
\clearpage

\subsection{Alberi bilanciati: alberi 2-3}
\subfile{Sotto sezioni/Sottosezione 5.4 - Alberi bilanciati: alberi 2-3.tex}
\clearpage

\subsection{Dizionari e code con priorità}
\subfile{Sotto sezioni/Sottosezione 5.5 - Dizionari e code con priorita.tex}
\clearpage

\subsection{Mergeable heaps}
\subfile{Sotto sezioni/Sottosezione 5.6 - Mergeable heaps.tex}
\clearpage

\subsection{Code concatenabili}
\subfile{Sotto sezioni/Sottosezione 5.7 - Code concatenabili.tex}

\end{document}