\documentclass{subfiles}
\begin{document}
Il problema di soddisfacibilità di espressioni booleane, o SAT, è il primo fra i problemi che si studieranno.
Il problema è il seguente: data un'espressione booleana, esiste una combinazione di ``VERO'' e ``FALSO'' delle sue variabili, tale che questa risulti vera?

Primo ad occuparsi del problema fu Cook, che col suo teorema dimostro che SAT fosse un NP-completo.

\begin{Theorem*}[di Cook]
    il problema di determinare se un'espressione booleane è soddisfacibile, è un problema \(\mathcal{NP}-completo\).

    \begin{Proof*}
        In quanto si suppone che SAT sia \(\mathcal{NP}-completo\), si deve prima dimostrare essere \(\mathcal{NP}\),
        per poi dimostrare che ogni altro linguaggio in \(\mathcal{NP}\), è polinomialmente trasformabile in SAT.
        \\ \\
        Partendo col dimostrare che SAT \(\in \mathcal{NP}\): sia \(L\) il linguaggio di tutte le stringhe soddisfacenti espressioni booleane.
        Si assuma unicamente che \(L \in \mathcal{NP}\). Un'algoritmo non deterministico tale da accettare \(L\),
        dovrebbe procedere ``indovinando'' un'assegnazione di 1 e 0 delle variabili della stringa in input, se questa esiste.
        \\
        Supponendo che questa esista, dovrebbe procedere col sostituire gli 0 e gli 1 alle variabili, ovunque esse compaiano nella stringa di input:
        valutando così l'espressione e verificando se essa assuma valore 1. Cio di fatti può essere effettuato in tempo proporzionale alla lunghezza con vari algoritmi di parsing.
        Pertanto esiste una MT non deterministica che accetta espressioni booleana soddisfacibili in tempo polinomiale, da cui il SAT si dimostra essere \(\mathcal{NP}\).
        \\ \\
        Dimostrato SAT \(\in \mathcal{NP}\), si procede col dimostrare che lo stesso è \(\mathcal{NP}-completo\).
        Sia \(M\) un NDTM di tempo polinomiale che accetta un qualsiasi linguaggio \(L \in \mathcal{NP}\), sia \(\omega\) una stringa di input per \(M\).
        Da \(M \text{e} \omega\) si costruisce \(\omega_{0}\): espressione booleana che è soddisfatta se e solo se \(\omega\) è accettata da \(M\).
        \'E noto che se un linguaggio è in \(\mathcal{NP}\), questi è accettato da una NDTM polinomiale.
        Da ciò si può assumere \(M\) a singolo nastro. Sia supposto che \(M\) abbia stati \(q_{0}, \ldots, q_{s}\), simboli di nastro \(X_{1}, \ldots, X_{m}\),
        sia infine \(p(n)\) la complessità di \(M\).
        \\ \\
        Sia \(\omega\) un'input di \(M\) di lunghezza \(n\). Se \(\omega\) è accettata da \(M\), ciò è effettuato in \(p(n)\) mosse.
        Se ciò avviene necessariamente esiste una sequenza \(Q_{0}, \ldots, Q_{q}\) di istantanee, tali che \(Q_{0}\) sia l'istantanea iniziale e \(Q_{q}\) quella accettante,
        \(q \le p(n)\), con nessuna delle istantanee aventi più di \(p(n)\) celle.
        Considerando dunque \(\omega_{0}\) questa rappresenta al più una sequenza di istantanee di \(M\), possibilmente non valida.
        Segue che \(\omega_{0}\) sarà 1 se e solo se l'assegnazione alle variabili, rappresenta un sequenza \(Q_{0}, \ldots, Q_{q}\) di istantanee che portano ad accettazione.
        Quindi come anticipato: \(\omega_{0} = 1 \iff M \text{accetta} \omega\).
        \clearpage

        Sia \(\omega_{0}\) definita per mezzo delle seguenti variabili proporzionali.
        \begin{itemize}
            \item \(C(i, j, t)\): questa è 1 se e solo se l'i-esima cella di \(M\) contiene il j-esimo simbolo al tempo t-esimo.
            \item \(S(k, t)\): è 1 se e solo se \(M\) è nel k-esimo stato al tempo t-esimo.
            \item \(H(i, t)\): posta ad 1 se e solo se al tempo t-esimo si sta leggendo la i-esima cella.
        \end{itemize}

        \noindent Si anno cosi \(\order{p^{2}(n)}\) variabili\footnote[12]{Per il resto della dimostrazione si suppone che le variabili siano rappresentate da un'unico bit, ciò non comporta una perdita di generalità.},
        che rappresentate in binario richiedono \(c \log n\) bit, con \(c \text{dipendente da} p\).
        Nella costruzione di \(\omega_{0}\) si deve tenere conto del seguente predicato: \(U(X_{1}, \ldots, X_{r})\); questi sarà vero se uno è uno solo degli \(X_{i}\) è vero.
        Con
        \[
            U(X_{1}, \ldots, Q_{q}) = (X_{1} + \ldots + X_{r})\left(\prod\limits_{i, j, i \ne j}{\lnot X_{i} + \lnot X_{j}}\right)
        \]

        \noindent Se \(M \text{accetta} \omega\), vi è una sequenza accettante di istantanee \(Q_{0}, \ldots, Q_{q}\) prodotta da \(M \text{processando} \omega\).
        Si assuma che \(M\) sia stata modificata in modo tale che, se questa termina la computazione prima delle \(p(n)\) mosse, continuerà a svolgere mosse ``a vuoto'',
        finché si siano raggiunte le \(p(n)\) mosse. Si può asserire che stabile se \(Q_{0}, \ldots, Q_{q}\) sia una sequenza accettante,
        sia equivalente ad asserire che le seguenti espressioni sono contemporaneamente verificate.
        \begin{itemize}
            \item Sia \(A\) l'espressione che stabilisce che ad ogni unità di tempo, \(M\) sta leggendo esattamente una cella.
                  Posto \(A_{t}\) il fatto che al tempo t sia letto un solo simbolo, allora \(A = A_{0} A_{1} \cdots A_{p(n)}\), con
                  \[
                      A_{t} = U(H(1, t), \ldots, H(p(n), t))
                  \]
            \item Sia \(B\) l'espressione che asserisce che per ogni nastro, un solo simbolo è presente in ciascuna cella.
                  Posto \(B_{t}\) il fatto che al tempo t il nastro contiene un solo simbolo, allora
                  \[
                      B = \prod\limits_{i, t}{B_{it}}
                  \]
                  con \(B_{it} = U(C(i, 1, t), \ldots, C(i, p(n), t))\).

            \item Sia \(C\) l'espressione che stabilisce che per ogni \(t, M\) sia in unico stato. Segue
                  \[
                      C = \prod\limits_{0 \le t \le p(n)}{U(S(1, t)), \ldots, S(p(n), t)}
                  \]
            \item Sia \(D\) l'espressione che impone che al più una cella per volta può essere modificata, per ogni t. Si ha
                  \[
                      D = \prod\limits_{i, j, t}{(C(i,j, t) \equiv C(i,j, t + 1)) + H(i, t)}
                  \]
                  L'espressione \((C(i,j, t) \equiv C(i,j, t + 1)) + H(i, t)\) stabilisce che
                  \begin{itemize}
                      \item la cella i è letta al tempo t, o che
                      \item il simbolo j è nella cella i al tempo t + 1, se e solo se lo era al tempo t.
                  \end{itemize}

            \item Sia \(E\) l'espressione che stabilisce che ciascuna ID successiva, sia ottenuta dalla precedente per una mossa consentita dalla funzione di transizione.
                  Siano \(E_{ijkt}\) espressioni che stabiliscono una delle seguenti proposizioni
                  \begin{itemize}
                      \item la i-esima cella non contiene il simbolo j-esimo al tempo t;
                      \item la testina non sta leggendo la cella i-esima al tempo t;
                      \item \(M\) non è nel k-esimo stato al tempo t;
                      \item l'istantanea successiva è ottenuta dalla precedente, per una mossa consentita dalla funzione di transizione di \(M\).
                  \end{itemize}
                  Allora
                  \[
                      E = \prod\limits_{i,j,k,t}{E_{ijkt}}
                  \]
                  con
                  \[
                      E_{ijkt} = \lnot C(i,j,t) + \lnot H(i,t) + \lnot S(k, t) + \sum\limits_{i}{C(i, j_{t}, t + 1) S(k_{t}, t + 1) H(k_{t}, t + 1)}
                  \]

            \item Sia \(F\) l'espressione che asserisce che le condizioni iniziali, siano soddisfatte
                  \[
                      F = S(1, 0)H(1, 0) \prod\limits_{1 \le i \le n}{C(i, j_{i}, 0)} \prod\limits_{n \le i \le p(n)}{C(i, B, 0)}
                  \]

            \item Sia \(G\) l'espressione che stabilisce che \(M\) rimane nello stato accettante, terminata la computazione.
                  Allora \(G = (S(s, p(n)))\), assunto \(q_{s}\) stato finale.
        \end{itemize}

        \noindent L'espressione \(\omega_{0}\) è dunque uguale a \(ABCDEFG\). Poiché ciascun fattore richiede al più \(\order{p^{3}(n)}\) simboli,
        \(\omega_{0} \text{stesso ha} \order{p^{3}(n)}\) simboli. Per quanto detto sui simboli, segue che \(\omega_{0}\) ha lunghezza \(\order{p^{3}(n)\log n}\).
        \\
        Da ciò si conclude che data una sequenza di ID \(Q_{0}, \ldots, Q_{q}\) accettante, è possibile trovare un'assegnazione di 0 e 1 tali da rendere vera \(\omega_{0}\);
        viceversa data un'assegnazione che rende \(\omega_{0}\) vera, si può trovare una sequenza di istantanee accettante. Indi per cui \(\omega_{0}\) soddisfatta se e solo se \(M \text{accetta} \omega\).
        Pertanto dimostrando che SAT sia \(\mathcal{NP}-completo\).
    \end{Proof*}
\end{Theorem*}

\begin{Theorem*}[3-SAT] Il problema di soddisfacibilità di espressioni booleane, con variabili a tre letterali è \(\mathcal{NP}-completo\).
    \begin{Proof*}
        Dato un prodotto di somme, si rimpiazza ciascuna delle somma \((x_{1} + x_{2} + \ldots + x_{k}), k \ge 4\) con
        \[\begin{aligned}
                (x_{1} + x_{2} + y_{1})(x_{3} + \overline{y_{1}} + y_{2}) & (x_{4} + \overline{y_{2}} + y_{3}) \cdots                                                \\
                                                                          & (x_{k - 2} + \overline{y_{k - 4}} + y_{k - 3})(x_{k - 1} + x_{k} + \overline{y_{k - 3}})
            \end{aligned}\]
        per nuove variabili \(y_{1}, \ldots, y_{k-3}\).
        Vi è un'assegnazione delle nuove variabili che rende l'espressione sostitutiva vera se e solo se uno dei letterali \(x_{1}, \ldots, x_{k}\) è vero,
        ossia se l'espressione originale è vera. Si assuma \(x_{i} = 1\) allora \(y_j = 1 \text{per} j \le i - 2, 0\) altrimenti.
        Viceversa si assuma che qualche assegnazione di \(y_{i}\) renda vera l'espressione. Se \(y_{0} = 0\), allora uno tra \(x_{1} \text{e} x_{2}\) ha valore 1.
        In ogni modo qualche \(x_{i} = 1\). Per ogni CNF \(E\) si può trovare una 3-CNF \(E'\) tale che quest'ultima sia vera se e solo se \(E\) è vera.
        Inoltre, ciò può essere effettuato in tempo lineare poiché la trasformazione si applica direttamente.
    \end{Proof*}

\end{Theorem*}
\end{document}