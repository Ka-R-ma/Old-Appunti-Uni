\documentclass{subfiles}
\begin{document}
Un MT non deterministica (NDTM) è definita tramite la medesima settupla della sua corrispondente deterministica, da quest'ultima differisce per la funzione di transizione,
che nel caso non deterministico è definita come
\[
    \delta : Q \cp T^{k} \to Q \cp (T \cp \Set{L, R, S})^{k}
\]
Invariata riamane anche la definizione di istantanea.

\begin{Definition*}
    Dicasi che una NDTM ha complessità \(T(n)\) se, per ogni stringa accettata di taglia \(n\), esiste una sequenza accettante di al più \(T(n)\) mosse.
    Dicasi che invece una NDTM ha complessità di spazio \(S(n)\) se, per ogni stringa accettata di taglia \(n\), esiste una sequenza di mosse per cui si ha accettazione,
    nella quale al più \(S(n)\) celle sono lette da un qualsiasi nastro.
\end{Definition*}
\end{document}