\documentclass{subfiles}
\begin{document}
\begin{Definition*}
    Siano \(G \text{e} H\) due grafi. Dicasi che \(G \text{e} H\) sono isomorfi se hanno lo stesso numero di vertici, archi e si mantiene la connettività degli archi.
    Cioè esiste una funzione biettiva \(f\) tale che, considerati due vertici \(u, v \in G\), questi sono adiacenti se e solo se \(f(u), f(v)\) sono adiacenti in \(H\).
\end{Definition*}

\begin{Theorem}
    Il problema del clique può essere polinomialmente ridotto al problema di Isomorfismo tra sotto-grafi. Quindi quest'ultimo è \(\mathcal{NP}-completo\).
\end{Theorem}
\end{document}