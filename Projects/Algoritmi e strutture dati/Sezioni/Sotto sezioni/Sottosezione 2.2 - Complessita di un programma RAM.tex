\documentclass{subfiles}
\begin{document}
Il modello RAM sinora descritto è definito come \emph{modello RAM con costo uniforme}: in questo caso ogni istruzione richiede un'unità di tempo e ogni registro un'unità di spazio.
Un'ulteriore modello è il \emph{modello RAM con costo logaritmico}, analizzato nel paragrafo seguente.

\subsubsection{RAM con costo logaritmico}
Nel modello con costo logaritmico si tiene conto della dimensione finita della memoria, nella fattispecie della dimensione limitata di una WORD.
In tal senso, sia \(l(i)\) la seguente funzione che stabilisce il numero di bit per rappresentare l'operando
\[
    l(i) = \begin{cases}
        \floor{\log \abs{i}} + 1, \quad & \text{se } i \neq 0 \\
        1, \quad                        & \text{se} i = 0     \\
    \end{cases}
\]
Si ha quindi che il costo per accedere agli operandi è quanto segue
\[\begin{aligned}
        =i: & \quad  l(i)                       \\
        i:  & \quad l(i) + l(c(i))              \\
        *i: & \quad l(i) + l(c(i)) + l(c(c(i))) \\
    \end{aligned}\]
\noindent Analizzando, a titolo di esempio, l'istruzione \lstinline[language = RAM]{ADD *i}, da quanto sopra si necessita \(l(i) + l(c(i)) + l(c(c(i)))\) per accedere ad *i,
a cui aggiungere il costo di accesso all'accumulatore pari a \(l(c(0))\).
\end{document}