\documentclass{subfiles}
\begin{document}
Il problema del Vertex cover è il seguente: dato un grafo non orientato \(G = (V, E)\), esiste \(S \subseteq V\) tale che ogni arco di \(G\) sia incidente ad un vertice in \(S\)?
\begin{Theorem}
    Il clique problem è polinomialmente riducibile al problema del vertex cover. Quindi Vertex cover è \(\mathcal{NP}-completo\).

    \begin{Proof*}
        Dato un grafo \(G = (V, E)\) si consideri \(\overline{G} = (V, \overline{E})\) con
        \[
            \overline{E} = \set{(v, w)}{v, w \in V, v \ne w, (v, w) \notin E}
        \]
        Si dichiara che \(S \subseteq V\) è clique in \(G\) se e solo se \(V \setminus S\) è una vertex cover per \(\overline{G}\).
        Si ha infatti che se \(S\) è clique per \(G\), nessun arco in \(\overline{G}\) connette due vertici in \(S\).
        Dunque ogni arco in \(\overline{G}\) è incidente ad almeno un vertice in \(V \setminus S\), quindi \(V \setminus S\) è una vertex cover per \(\overline{G}\).
        Viceversa, se \(V \setminus S\) è una vertex cover per \(\overline{G}\), allora ogni arco di \(\overline{G}\) è incidente ad almeno un arco in \(V \setminus S\).
        Conseguentemente, nessun arco in \(\overline{G}\) connette due vertici in \(S\).
        Indi per cui, ogni coppia di vertici in \(S\) è connesso in \(G\), quindi \(S\) è clique per \(G\).
        \\ \\
        Per dimostrare se esiste o meno un clique di taglia \(q\), si costruisce \(\overline{G}\) e si determina se questi ha una vertex cover di taglia \(\norm{V} - k\).
        Data una rappresentazione standard di \(G \text{e} k\), si può sicuramente trovare \(\overline{G} \text{e} \norm{V} - k\) in tempo polinomiale rispetto la lunghezza della rappresentazione di \(G \text{e} k\).
    \end{Proof*}
\end{Theorem}
\end{document}