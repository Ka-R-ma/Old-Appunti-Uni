\documentclass{subfiles}
\begin{document}
\begin{Definition*}
    Dato un grafo pesato e orientato \(G\), dicasi costo di un cammino \(p\), la somma dei costi degli archi in \(p\) stesso.
\end{Definition*}

\noindent Legati al costo di un cammino vi sono due problemi, quali
\begin{itemize}
    \item chiusura transitiva;
    \item single-source problem.
\end{itemize}

\subsubsection{Chiusura transitiva}
\noindent Sia \(G* = (V, E*)\) un grafo ottenuto da un grafo orientato \(G = (V, E)\), tale che \((v, w) \in E*\) se e solo se esiste in \(G\) un cammino da \(v \text{a} w\).
\begin{Definition*}
    Un sistema \((S, +, \cdot, 0, 1)\) è detto \emph{semi-anello chiuso}, con \(S\) insieme di elementi, \(+, \cdot\) operazioni binarie tali da soddisfare le seguenti proprietà.
    \begin{itemize}
        \item \((S, +, 0)\) è un monoide, ossia:
              \begin{itemize}
                  \item \(\forall a, b \in S, a + b \in S\);
                  \item \(\forall a, b, c \in S, a + (b + c) = (a + b) + c\);
                  \item \(\forall a \in S, a + 0 = 0 + a = a\).
              \end{itemize}
              discorso analogo per \((S, \cdot, 1)\).
        \item \(+\) è commutativo;
        \item \(\cdot\) è distributivo rispetto \(+\);
        \item se \(a_{1}, \ldots, a_{i}, \ldots\) è una sequenza di elementi in \(S\), allora \(a_{1} + \ldots + a_{i} + \ldots\) esiste ed è unico in \(S\).
        \item \(\cdot\) distribuisce su somme infinite numerabili, come su quelle finite.
    \end{itemize}
\end{Definition*}

\noindent Dalle proprietà quattro e cinque, segue
\[
    \left(\sum\limits_{i}{a_{i}}\right) \cdot \left(\sum\limits_{j}{b_{j}}\right) = \sum\limits_{i}\left(\sum\limits_{j}{(a_{i} \cdot b_{j})}\right)
\]

\begin{Example*}
    Sia \(S_{1} = (\Set{0, 1}, +, \cdot, 0, 1)\), con \(+, \cdot\) definite come segue.
    \begin{center}
        \begin{tabular}{c|c c}
            + & 0 & 1 \\
            \hline
            0 & 0 & 1 \\
            1 & 1 & 1 \\
        \end{tabular}
        \hspace{25pt}
        \begin{tabular}{c|c c}
            \(\cdot\) & 0 & 1 \\
            \hline
            0         & 0 & 0 \\
            1         & 0 & 1 \\
        \end{tabular}
    \end{center}
    Le prime tre proprietà sono dimostrate banalmente, circa ultime due si osserva che una somma è \(0\) se tutti gli elementi sono \(0\).
\end{Example*}

\begin{Example*}
    Sia \(S_{2} = (\R^{+}, MIN, +, +\infty, 0)\). Risulta ovvio che \(+\infty\) sia elemento neutro per \(MIN \text{e} 0 \text{per} +\).
\end{Example*}

\begin{Example*}
    Sia \(S_{3} = (F_{\Sigma}, \cup, \cdot, \emptyset, \Set{\varepsilon})\) dove \(F_{\Sigma}\) è la famiglia di stringhe su un alfabeto \(\Sigma\).
    Le prime tre proprietà sono banalmente dimostrate, le ultime due sono definite se si definisce
    \[
        x \in (A_{1} \cup \ldots \cup A_{i} \cup \ldots) \iff x \in A_{i}, \text{per qualche}i
    \]
\end{Example*}

\noindent Se \((S, +, \cdot, 0, 1)\) è un semi-anello chiuso, inoltre \(a \in S\), si definisce \(*\) come l'operazione di chiusura, nel seguente modo.
\[
    \sum\limits_{i = 0}^{+\infty}{a^{i}}
\]

\noindent Sia \(G = (V, E)\) un grafo orientato, in cui ogni arco è etichettato da un elemento di un semi-anello \((S, + , \cdot, 0, 1)\).
Per ogni \((v, w)\) si definisce \(c((v, w))\): la somma delle etichette di tutti i cammini tra \(v, w\).
La computazione di \(c((v, w))\) è effettuata col seguente algoritmo.
\subfile{../Figure/Figura 6.4 - Algoritmo per il calcolo di c.tex}

si osservi che \(l(v_{i}, v_{j})\) è 0 se non esiste un arco \((v_{i}, v_{j}) \in E\), 1 altrimenti.
Inoltre se il semi-anello considerato è \(S_{1}\), l'operazione di chiusura \((C^{k - 1}_{kk})*\) può essere omessa.
\clearpage

\subsubsection{Single-Source problem}
Il \emph{single-source problem} può essere espresso come segue: dato \(G = (V, E)\) un grafo orientato e pesato, fissato un vertice \(s \in V\) detto \emph{sorgente},
calcolare il cammino ottimo (cioè il cammino con costo minimo) tra \(s \text{e} v, \forall v \in V\).
\\ \\
Informalmente, il \emph{single source shortest path} cerca di stabilire se dato un grafo pesato e orientato,
esiste un cammino da un vertice a tutti gli altri, nel qual caso lo calcola.
\\ \\
Un primo algoritmo per la risoluzione del problema risulta essere il \emph{Bellman-Ford, Figura \ref{Fig:6.5}}.
Realizzato con programmazione dinamica, questi calcola il costo minimo per andare da \(s \text{a} v_{k}\) iterativamente, in termini di vertici precedenti.
\subfile{../Figure/Figura 6.5 - BellmanFord.tex}

\noindent nel caso dell'algoritmo di cui sopra, i pesi degli archi possono assumere valori anche negativi. \\
Nel caso di grafi in cui i pesi risultano tutti maggiori di o uguali a 0, è possibile applicare l'algoritmo di \emph{Dijkstra}.
\subfile{../Figure/Figura 6.6 - Dijkstra.tex}

\noindent Circa il costo computazionale: Bellman-Ford risulta essere \(\order{\norm{V} + \norm{E}}\), Dijkstra risulta invece \(\order{n^{2}}\) se D è implementata come array,
\(\order{\norm{E} \log n}\) se implementata come coda con priorità.
\end{document}