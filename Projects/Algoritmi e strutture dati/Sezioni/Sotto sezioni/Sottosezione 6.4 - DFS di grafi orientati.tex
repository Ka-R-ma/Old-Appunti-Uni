\documentclass{subfiles}
\begin{document}
Nel caso in cui il grafo su cui si deve effettuare la DFS sia un grafo orientato, se si applica opportune modifiche, si può sfruttare l'algoritmo sin ora visto.
Nel caso di grafi orientati infatti, la DFS produce una Spanning-Forest i cui archi sono di quattro tipi.
\begin{itemize}
    \item Tree edge: collegano vertici ancora non visitati.
    \item Back edge: collegano discendente e antenato di un vertice.
    \item Forward edge: collegano l'antenato e il discendente di un vertice, se questi non è Tree edge.
    \item Cross edge: archi che collegano vertici tra i quali non vie relazione di discendenza.
\end{itemize}

\begin{Lemma}
    Se \((v, w)\) è un Cross edge, allora \(DFNUMBER[v] > DFNUMBER[w]\).
\end{Lemma}
\end{document}