\documentclass{subfiles}
\begin{document}
\begin{Definition*}
    Si definisce \emph{P-space} la classe dei linguaggi tali per cui esiste una DTM che li riconosce in spazio polinomiale.
\end{Definition*}

\noindent Si ha infatti che una MT che impiega tempo polinomiale, necessariamente non può che impiegare un numero di celle polinomiale.
\'E altresì definita la classe \emph{NP-space}. Segue dunque la domanda \(P-space = NP-space\)?
Da quanto studiato sin'ora, \(NP-space \subseteq P-space\); per il Teorema di Savich si dimostra che in realtà i due spazi coincidono.
\\ \\
Ultima classe di problemi è la classe \emph{EXP-space}: questa contiene tutti quei problemi che, indipendentemente se risolti con una DTM o una NDTM,
impiegano tempo almeno esponenziale.
\end{document}